% \documentclass[paper=a4, fontsize=11pt]{scrartcl} % A4 paper and 11pt font size
\documentclass[11pt, a4paper]{book}
\usepackage[a4paper, total={5.5in, 8.75in}]{geometry}
\usepackage[T1]{fontenc} % Use 8-bit encoding that has 256 glyphs
\usepackage[utf8]{inputenc}
\usepackage{fourier} % Use the Adobe Utopia font for the document - comment this line to return to the LaTeX default
\usepackage{listings} % para insertar código con formato similar al editor
\usepackage[spanish, es-tabla]{babel} % Selecciona el español para palabras introducidas automáticamente, p.ej. "septiembre" en la fecha y especifica que se use la palabra Tabla en vez de Cuadro
\usepackage{url} % ,href} %para incluir URLs e hipervínculos dentro del texto (aunque hay que instalar href)
\usepackage{graphics,graphicx, float} %para incluir imágenes y colocarlas
\usepackage[gen]{eurosym} %para incluir el símbolo del euro
\usepackage{cite} %para incluir citas del archivo <nombre>.bib
\usepackage{enumerate}
\usepackage{hyperref}
\usepackage{graphicx}
\usepackage{adjustbox}
\usepackage{booktabs}

\usepackage{tabularx}
\newcolumntype{L}{>{\raggedright\arraybackslash}X}
\newcolumntype{R}{>{\raggedleft\arraybackslash}X}

\usepackage[table,xcdraw]{xcolor}
\hypersetup{
	colorlinks=true,	% false: boxed links; true: colored links
	linkcolor=black,	% color of internal links
	urlcolor=cyan,		% color of external links
}

\renewcommand{\familydefault}{\sfdefault}
\usepackage{fancyhdr} % Custom headers and footers
\pagestyle{fancyplain} % Makes all pages in the document conform to the custom headers and footers
\fancyhead[L]{} % Empty left header
\fancyhead[C]{} % Empty center header
\fancyhead[R]{Ángel Gómez Martín} % My name
\fancyfoot[L]{} % Empty left footer
\fancyfoot[C]{} % Empty center footer
\fancyfoot[R]{\thepage} % Page numbering for right footer
%\renewcommand{\headrulewidth}{0pt} % Remove header underlines
\renewcommand{\footrulewidth}{0pt} % Remove footer underlines
\setlength{\headheight}{13.6pt} % Customize the height of the header
\setlength{\parskip}{1ex}

\usepackage{titlesec, blindtext, color}
\definecolor{gray75}{gray}{0.75}
\newcommand{\hsp}{\hspace{20pt}}
\titleformat{\chapter}[hang]{\Huge\bfseries}{\thechapter\hsp\textcolor{gray75}{|}\hsp}{0pt}{\Huge\bfseries}
\setcounter{secnumdepth}{4}
\usepackage[Lenny]{fncychap}

\definecolor{gray97}{gray}{.97}
\definecolor{gray75}{gray}{.75}
\definecolor{gray45}{gray}{.45}
\definecolor{gray30}{gray}{.94}

\lstset{
	frame=Ltb,
    framerule=0.5pt,
    aboveskip=0.5cm,
    framextopmargin=3pt,
    framexbottommargin=3pt,
    framexleftmargin=0.1cm,
    framesep=0pt,
    rulesep=.4pt,
    backgroundcolor=\color{gray97},
    rulesepcolor=\color{black},
    stringstyle=\ttfamily,
    showstringspaces = false,
    basicstyle=\scriptsize\ttfamily,
    commentstyle=\color{gray45},
    keywordstyle=\bfseries,
    numbers=left,
    numbersep=6pt,
    numberstyle=\tiny,
    numberfirstline = false,
    breaklines=true,
}
 

\begin{document}
	\begin{titlepage}
\newlength{\centeroffset}
\setlength{\centeroffset}{-0.5\oddsidemargin}
\addtolength{\centeroffset}{0.5\evensidemargin}
\thispagestyle{empty}

\noindent\hspace*{\centeroffset}\begin{minipage}{\textwidth}

\centering
\includegraphics[width=0.9\textwidth]{logos/logo_ugr.jpg}\\[1.4cm]

\textsc{ \Large TRABAJO FIN DE MÁSTER\\[0.2cm]}
\textsc{ GRADO EN INGENIERÍA INFORMÁTICA}\\[1cm]

{\Huge\bfseries Matroos \\}
\noindent\rule[-1ex]{\textwidth}{3pt}\\[3.5ex]
{\large\bfseries Creación, configuración y despliegue de bots en \textit{Discord} }
\end{minipage}

\vspace{2.5cm}
\noindent\hspace*{\centeroffset}
\begin{minipage}{\textwidth}
\centering

\textbf{Autor}\\ {Ángel Gómez Martín}\\[2.5ex]
\textbf{Director}\\ {Juan Julián Merelo Guervós}\\[2cm]
\includegraphics[width=0.3\textwidth]{logos/etsiit_logo.png}\\[0.1cm]
\textsc{Escuela Técnica Superior de Ingenierías Informática y de Telecomunicación}\\
\textsc{---}\\
Granada, 7 de Julio de 2022
\end{minipage}
\end{titlepage}

	\thispagestyle{empty}

\begin{center}
{\large\bfseries Matroos \\ Creación, configuración y despliegue de bots en \textit{Discord}. }\\
\end{center}
\begin{center}
Ángel Gómez Martín
\end{center}


\vspace{0.5cm}
\noindent{\textbf{Palabras clave}: software libre, \textit{Discord}, bot, \textit{API REST}, despliegue, \textit{backend}, \textit{frontend}, \textit{worker}
\vspace{0.7cm}

\noindent{\textbf{Resumen}\\

La creación y uso de bots se ha popularizado mucho en los últimos años, siendo extraño no encontrarlos integrados en multitud de sistemas. De entre estos destaca \textit{Discord}, una plataforma de mensajería instantánea utilizada principalmente por jóvenes que también ha ganado gran relevancia recientemente y donde los bots son ampliamente usados. Aunque los bots puedan parecer algo sencillo, los procesos que conllevan crearlos, configurarlos y desplegarlos pueden ser bastante complejos.

No obstante estos se pueden simplificar y unificar. En este proyecto se ha desarrollado una solución que aúna todos esos procesos en un puesto centralizado compuesto por un \textit{backend}, un \textit{frontend} y una serie de servicios llamados \textit{workers}, siendo un conjunto que permite facilitar y agilizar el desarrollo de estas tareas. Además ofrece modularidad de sus características, pudiendo adecuar las funcionalidades de cada bot adecuándose a las necesidades requeridas en cada situación. También ofrece una \textit{API REST}, la cual permite la comunicación con otro tipo de aplicaciones.
	

\cleardoublepage

\begin{center}
{\large\bfseries Matroos \\ Creation, configuration and deployment of bots in \textit{Discord}. }\\
\end{center}
\begin{center}
	Ángel Gómez Martín
\end{center}
\vspace{0.5cm}
\noindent{\textbf{Keywords}: \textit{open source}, \textit{Discord}, bot, \textit{REST API}, \textit{deployment}, \textit{backend}, \textit{frontend}, \textit{worker}
\vspace{0.7cm}

\noindent{\textbf{Abstract}\\

The creation and use of bots has become very popular in recent years, and it is strange not to find them integrated into a multitude of systems. Among these, \textit{Discord}, an instant messaging platform used mainly by young people, has also gained great relevance recently and where bots are widely used. While bots may seem straightforward, the processes involved in creating, configuring and deploying them can be quite complex.

However, these can be simplified and unified. In this project, a solution has been developed that brings together all these processes in a centralised position composed of a backend, a frontend and a series of services called ``workers''; a set that facilitates and speeds up the development of these tasks. It also offers modularity of its characteristics, being able to adapt the functionalities of each bot to suit the needs required in each situation. It also offers a REST API, which allows communication with other types of applications.

\cleardoublepage

\thispagestyle{empty}

\noindent\rule[-1ex]{\textwidth}{2pt}\\[4.5ex]

D. \textbf{Juan Julián Merelo Guervós}, Profesor del Departamento de Arquitectura y Tecnología de Computadores de la Universidad de Granada.

\vspace{0.5cm}

\textbf{Informa:}

\vspace{0.5cm}

Que el presente trabajo, titulado \textit{\textbf{Matroos}}, ha sido realizado bajo mi supervisión por \textbf{Ángel Gómez Martín}, y autorizo la defensa de dicho trabajo ante el tribunal que corresponda.

\vspace{0.5cm}

Y para que conste, expiden y firman el presente informe en Granada a 7 de Julio de 2022.

\vspace{1cm}

\textbf{El director:}

\vspace{5cm}

\noindent Fdo: Juan Julián Merelo Guervós



\chapter*{Agradecimientos}

A mi tutor, JJ, por ofrecerme su ayuda, conocimientos y acertados comentarios para la realización de este proyecto.

A mis padres, Elia y Ángel, por ser un pilar fundamental y por empujarme a seguir aprendiendo cosas nuevas y a superarme cada día.

A mi hermana Cristina, por apoyarme y echarme una mano siempre que lo he necesitado.

Y a Paula, por aguantarme más que nadie todo este tiempo.


	\newpage
	\tableofcontents

	\newpage
	\listoffigures

	\listoftables 
	\newpage

	\chapter{Introducción}

Los bots permiten automatizar tareas en sistemas conversacionales. Sin embargo, su creación en muchas ocasiones es compleja y no está al alcance de todos los usuarios. Existen tareas que son sencillas de realizar de manera manual, pero que pueden requerir un tiempo y dedicación considerable, y sería mucho más cómodo automatizarlas de alguna manera. Por otra parte, cuando hay un gran volumen de datos sobre el que realizar estas tareas, automatizarlas otorgaría grandes beneficios ya que no dependerían de la disponibilidad de una persona para realizarlas. Además, se eliminaría la necesidad de realizar en un gran número de ocasiones un proceso que podría ser corto pero muy repetitivo. Estos procesos pueden ser de muchos tipos y sobre muchos ámbitos diferentes dependiendo del entorno en que sean necesarios, y como al fin y al cabo son procesos específicos, el sistema de automatización también debería ser así, pero desarrollarlo requiere de una experiencia técnica determinada. Esto no esta al alcance de todos.

En concreto, en la plataforma \textit{Discord}\cite{discord}, que se ha hecho popular recientemente, este problema no es una excepción. En \textit{Discord} existen comunidades construidas en torno a multitud de ámbitos, como videojuegos, deportes, cine o incluso en el mundo profesional, donde la plataforma se ha hecho un hueco. Estas comunidades pueden crecer mucho, y del mismo modo crecen las necesidades de moderación y funcionalidades adicionales que pueden ser útiles para facilitar ciertas tareas en función del ámbito de la comunidad. Un ejemplo de esto podría ser la automatización de mensajes para anunciar los estrenos de cartelera semanal de un cine en concreto o la monitorización de sistemas en una una pequeña empresa.

Resolver este problema puede ayudar a todos aquellos usuarios y comunidades que buscan automatizar tareas específicas de manera individual sin tener que recurrir a terceros con los conocimientos específicos y con los que quizás se tendría que compartir información sensible. Por otro lado, tener que recurrir a conocimiento experto externo conlleva una serie de costes que en algunos casos puede ser inadmisible. Este problema suele ser general para todas las herramientas que permiten la integración con bots, pero dado que en cada plataforma los bots se crean con tecnologías y procedimientos diferentes, este proyecto se centra en Discord pues es la plataforma que probablemente podría beneficiarse más de esta nueva herramienta dada la gran cantidad de usuarios que hace uso de ella.

\section{Motivación}

La principal motivación para realizar este proyecto es que crear bots para \textit{Discord} con funcionalidades muy concretas con las herramientas actuales es una tarea compleja. Si bien existen librerías para lenguajes de programación, las herramientas que han surgido para crear estos bots de forma sencilla se centran en aspectos muy básicos que no tienen mucha cabida en ámbitos más concretos.

No existen \textit{frameworks} o sistemas que permitan crear fácilmente comandos que se ajusten a necesidades específicas, y menos que sean reproducibles o configurables de alguna manera con distintos datos. Además, cuando se trata de crear y alojar distintos bots en un mismo sistema, las actuales soluciones requieren que cada uno de estos bots sea una instancia independiente. Esto imposibilita la configuración de todos estos bots de una manera sencilla dentro de un mismo sistema. De nuevo, algunos de estos conceptos requieren de conocimientos experto extra necesarios para crear estas herramientas no están al alcance de todos.

\section{Problema y solución}

Como se indicaba en la sección anterior, usuarios con necesidades específicas (como por ejemplo podría ser un administrador de sistemas de una pequeña empresa, o incluso un usuario entusiasta al que le gusta automatizar tareas en su comunidad de \textit{Discord}) que quieren hacer uso de este software y de su sistema de bots se ven obligados a crear distintos bots muy específicos y en ocasiones poco reutilizables. Si bien se podrían programar comandos más concretos en un único bot, sería una tarea tediosa la reutilización entre diferentes ámbitos. Por ejemplo, dos comunidades de juegos de mesa, donde se comparten muchas funcionalidades, pero dependiendo del juego los detalles son distintos.

Incidiendo en el aspecto de la creación de bots forma más técnica, se observa que actualmente hay tres maneras de usar bots en \textit{Discord}:

\begin{itemize}
	\item \textbf{Usar bots ya existentes}. El bot ya se encuentra creado, configurado y desplegado, y solo es necesario agregarlo al servidor para poder disfrutar de sus funcionalidades.
	\item \textbf{Crear un bot a alto nivel}. Este caso es similar al anterior, ya que se trata de un bot genérico, que es configurable en cierta medida para cada servidor. Esta configuración se hace a través de algún tipo de herramienta (generalmente una aplicación web) que permite configurar los comandos deseados. Por otro lado, ya que están pensados para un público general, las posibilidades de configuración son escasas. Suelen tener algunas plantillas de comandos básicos, como temporizadores o respuestas automáticas.
	\item \textbf{Crear un bot a bajo nivel}. En este caso el bot se crea haciendo uso de las diferentes \textit{API} que ofrece \textit{Discord} para ello y la personalización es máxima. En cambio, es más tedioso, y requiere conocimientos extra que algunos usuarios pueden no tener (como programación). Además hay que tener en cuenta que el bot debe ser desplegado manualmente, por lo que requiere un esfuerzo extra.
\end{itemize}

Como se observa, en el primer caso la capacidad de configuración del bot es nula y en el segundo las funcionalidades son muy reducidas. En el último caso, aunque permite realizar cualquier configuración, se pierde el aspecto de la administración central, ya que se crean bots independientes. 

Por tanto, se pretende desarrollar un \textit{framework} para crear bots de \textit{Discord} completamente configurables, y que además permita la creación sencilla de comandos. También, se busca también la centralización de procesos.

	\chapter{Estado del arte}

En este capítulo se hace un repaso de las distintas soluciones que existen actualmente para la creación de bots de \textit{Discord} y la configuración de sus comandos.

\section{Contexto y definiciones previas}

Los bots actúan como un usuario más dentro de un servidor de \textit{Discord}. En cambio, debido a su naturaleza, deben ser agregados a los servidores por usuarios humanos. Se pueden crear infinidad de bots, por lo que han surgido webs (como \href{https://top.gg/}{top.gg} o \href{https://bots.ondiscord.xyz/}{Bots on Discord}) que permiten que los usuarios encuentren bots ya creados y desplegados listos para ser utilizados en sus servidores. Estos cuentan con funcionalidades específicas, por lo que no pueden modificarse por los usuarios.

Además de este tipo de bots y webs, han surgido otras plataformas que permiten que los usuarios creen y configuren sus propios bots. Estas son las que se desarrollan en este capítulo. A continuación se enumeran algunos conceptos que son interesantes conocer ya que se utilizan tanto en este capítulo como en el resto del documento.

Conceptos técnicos, que definen la manera en la que se estructuran, configuran y definen los bots:

\begin{itemize}
	\item \textbf{Comandos predefinidos}. Comandos cuya funcionalidad está ya definida (y por tanto programada) y no puede ser modificada por un usuario.
	\item \textbf{Comandos personalizados}. Comandos que se pueden crear a partir de comandos predefinidos que pueden utilizarse con distintos parámetros para modificar su comportamiento.
	\item \textbf{Comandos reutilizables}. Comandos personalizados que una vez configurados pueden utilizarse en distintos bots.
	\item \textbf{Despliegue de un bot}. Todas aquellas tareas y actividades que hacen que un bot se encuentre disponible para ser usado por usuarios.
	\item \textbf{Control del despliegue}. Capacidad de controlar y monitorizar el despliegue de bots en un sistema.
\end{itemize}

Conceptos relacionados con funcionalidades de \textit{Discord}:

\begin{itemize}
    \item \textbf{Comandos de moderación}. Comandos cuya funcionalidad se centra en la moderación de usuarios en los servidores de \textit{Discord}.
\end{itemize}

\section{Soluciones actuales}

Las soluciones actuales se podrían dividir en dos grupos, las herramientas \textit{no-code} y aquellas herramientas que hacen uso de programación. En ambas la interacción con el sistema se hace a través de una aplicación web y, además, suelen tener una apariencia muy similar, siendo las \textit{no-code} algo más complejas de usar. Las primeras se centran en usuarios con menor conocimiento técnico, abstrayendo todos los detalles de este tipo. En cambio, las segundas son utilizadas por aquellos usuarios con un conocimiento informático más amplio.

En las siguientes secciones se incluyen las herramientas con características más interesantes. Por otro lado, para cada herramienta se incluye una tabla como la siguiente, donde se resumen las características más importantes de cada una de ellas.

\begin{table}[H]
    \centering
    \def\arraystretch{1.25}
    \begin{adjustbox}{max width=\textwidth}
    \begin{tabularx}{325px}{|l|L|}
    \hline
        \multicolumn{2}{|c|}{\textbf{Nombre de la herramienta}} \\ \hline
    \hline
        \textbf{Tipo de comandos} & El tipo de comandos que provee la herramienta, y la cantidad de comandos personalizados que se pueden crear. \\ \hline
        \textbf{Comandos reutilizables} & Si los comandos se pueden reutilizar entre bots o no. \\ \hline
        \textbf{Control del despliegue} & Si el usuario tiene control del despliegue del bot o no. \\ \hline
        \textbf{Número de bots} & La cantidad de bots que el usuario puede crear y configurar. \\ \hline
        \textbf{Experiencia} & La experiencia de usuario a la hora de utilizar las herramientas. \\ \hline
        \textbf{Personalización extra} & Si la herramienta permite una personalización más avanzada, y si es de pago o no. \\ \hline
        \textbf{Características} & Las funcionalidades configurables en los bots. \\ \hline
        \textbf{Logs} & Si permiten que los usuarios tengan acceso a los logs de los bots. \\ \hline
        \textbf{Premium} & Coste de los planes de pago de la herramienta. \\ \hline
    \end{tabularx}
    \end{adjustbox}
\end{table}

\subsection{Herramientas \textit{no-code}}

Estas herramientas permiten la creación de bots sin hacer uso de recursos de programación o similares. Cuentan con un repertorio de comandos predefinidos, de los cuales se pueden crear comandos personalizados modificando los parámetros de estos. Esto se hace a través de interfaces gráficas sencillas compuestas por formularios y otros elementos.

Por ejemplo, el comando predefinido más común sirve para enviar mensajes. A partir de este comando se pueden crear distintos comandos personalizados, cambiando la palabra clave que los usuarios deben introducir (en los canales de texto) para activar el comando y el mensaje que el bot debe enviar.

En general la mayoría tienen una serie de funcionalidades gratuitas, teniendo que suscribirse a un plan de pago mensual para obtener funcionalidades extra. Estas suscripciones suelen tener un coste aproximado de cinco dólares mensuales, pudiendo comprar suscripciones anuales o incluso de por vida y suelen ofrecer un repertorio de comandos más amplio, conexiones con redes sociales, estadísticas de uso e incluso \textit{logging}.

En la mayoría de casos estas plataformas cuentan con un único bot que se debe agregar al servidor de \textit{Discord} deseado. Esto implica que el uso de estos servicios no permite por tanto agregar varios bots con distintas funcionalidades a un mismo servidor. Si un usuario utiliza \textit{ProBot} (servicio que se explica en detalle a continuación), entonces el bot que proporciona \textit{ProBot} sólo podrá utilizarse una sola vez por servidor.

En ellas se puede observar también uno de los principales problemas ya mencionados, la muy reducida personalización y reutilización de los comandos. En estos sistemas no se puede crear un comando específico con una funcionalidad concreta, sino que se basan en funcionalidades predefinidas no reutilizables.

Estas funcionalidades predefinidas son en su mayoría de moderación de usuarios y envío de mensajes, y no se pueden reutilizar entre distintos bots. Además, debido a la estructura y arquitectura de estas plataformas, los usuarios no pueden ampliar el repertorio de comandos, teniendo que conformarse con los existentes.

\subsubsection{ProBot}

\href{https://probot.io/}{\textit{ProBot}} es sin duda la más interesante de las herramientas \textit{no-code} debido a que permite crear ilimitados comandos personalizados, siendo la principal desventaja que estos comandos son predefinidos, y no se puede cambiar su funcionalidad. Los comandos predefinidos se centran en moderación y mensajes automáticos, por lo que las posibilidades no son muy amplias.

A favor de esta herramienta también destaca que es sencilla de utilizar, la interfaz web intenta imitar a la de \textit{Discord} y es intuitiva. Por contra, es bastante intrusiva la modalidad \textit{premium}, ya que muchas secciones sugieren la compra de esta modalidad. Además es imposible controlar el despliegue del bot, y no es posible reutilizar comandos.

Sus características son:

\begin{table}[H]
    \centering
    \def\arraystretch{1.25}
    \begin{adjustbox}{max width=\textwidth}
    \begin{tabularx}{325px}{|l|L|}
    \hline
        \multicolumn{2}{|c|}{\textbf{\textit{ProBot}}} \\ \hline
    \hline
        \textbf{Tipo de comandos} & Predefinidos (ilimitados) \\ \hline
        \textbf{Comandos reutilizables} & No \\ \hline
        \textbf{Control del despliegue} & No \\ \hline
        \textbf{Número de bots} & 1, único \\ \hline
        \textbf{Experiencia} & Sencilla \\ \hline
        \textbf{Personalización extra} & Requiere \textit{premium} (mensualidades) \\ \hline
        \textbf{Características} & · Moderación\linebreak · Estadísticas\linebreak · Mensajes automáticos\linebreak · Música \\ \hline
        \textbf{Logs} & No \\ \hline
        \textbf{Premium} & \$60 al año \\ \hline
    \end{tabularx}
    \end{adjustbox}
    \caption{Características de \textit{ProBot}.}
\end{table}

\begin{figure}[H]
	\centering
	\includegraphics[width=1\textwidth]{img/probot.png}
	\caption{Interfaz web de \textit{ProBot}.}
\end{figure}

\subsubsection{Mee6}

\href{https://mee6.xyz/}{\textit{Mee6}} es otra herramienta muy similar a la anterior, siendo la principal diferencia que en este caso los comandos personalizados se limitan a 5. Por contra, tiene un mayor catálogo de funcionalidades.

De nuevo no es posible controlar el despliegue del bot, como tampoco es posible crear otro bot y agregarlo a un mismo servidor, o reutilizar comandos.

Sus características son:

\begin{table}[H]
    \centering
    \def\arraystretch{1.25}
    \begin{adjustbox}{max width=\textwidth}
    \begin{tabularx}{325px}{|l|L|}
    \hline
        \multicolumn{2}{|c|}{\textbf{\textit{Mee6}}} \\ \hline
    \hline
        \textbf{Tipo de comandos} & Predefinidos (muy limitados, 5) \\ \hline
        \textbf{Comandos reutilizables} & No \\ \hline
        \textbf{Control del despliegue} & No \\ \hline
        \textbf{Número de bots} & 1, único \\ \hline
        \textbf{Experiencia} & Sencilla \\ \hline
        \textbf{Personalización extra} & Requiere \textit{premium} (mensualidades) \\ \hline
        \textbf{Características} & · Moderación\linebreak · Estadísticas\linebreak · Mensajes automáticos\linebreak · Música\linebreak · Temporizadores\linebreak · \textit{Quiz} / \textit{Trivia} \\ \hline
        \textbf{Logs} & No \\ \hline
        \textbf{Premium} & \$50 al año / \$90 de por vida  \\ \hline
    \end{tabularx}
    \end{adjustbox}
    \caption{Características de \textit{Mee6}.}
\end{table}

\begin{figure}[H]
	\centering
	\includegraphics[width=1\textwidth]{img/mee6.png}
	\caption{Interfaz web de \textit{Mee6}.}
\end{figure}


\subsubsection{BotGhost}

\href{https://botghost.com/}{\textit{BotGhost}} es un híbrido entre \textit{ProBot} y \textit{Mee6}, ya que tiene características comunes de ambos. La principal característica de esta herramienta es que permite crear comandos personalizados haciendo uso de una serie de módulos que se pueden interconectar para definir el ciclo de vida de un comando.

Esta característica es muy interesante, pero está muy limitada y las funcionalidades que permite realizar se resumen en envío de mensajes y tareas de moderación de usuarios muy básicas. El plan \textit{premium} sería necesario en este caso para poder sacarle partido a esta funcionalidad.

Otro aspecto interesante es que se pueden crear distintos bots, hasta 50 distintos si se opta por la opción \textit{premium}.

Sus características son:

\begin{table}[H]
    \centering
    \def\arraystretch{1.25}
    \begin{adjustbox}{max width=\textwidth}
    \begin{tabularx}{325px}{|l|L|}
    \hline
        \multicolumn{2}{|c|}{\textbf{\textit{BotGhost}}} \\ \hline
    \hline
        \textbf{Tipo de comandos} & Predefinidos (muy limitados, 5) \\ \hline
        \textbf{Comandos reutilizables} & Sí \\ \hline
        \textbf{Control del despliegue} & No (Sólo encendido y apagado) \\ \hline
        \textbf{Número de bots} & 1, único (50 con \textit{premium}) \\ \hline
        \textbf{Experiencia} & Compleja \\ \hline
        \textbf{Personalización extra} & Requiere \textit{premium} (mensualidades) \\ \hline
        \textbf{Características} & · Moderación\linebreak · Estadísticas\linebreak · Mensajes automáticos\linebreak · Temporizadores\linebreak · Integración con videojuegos\linebreak · Meteorología\linebreak · Música\linebreak · \textit{Quiz} / \textit{Trivia} \\ \hline
        \textbf{Logs} & No \\ \hline
        \textbf{Premium} & \$60 al año / \$100 de por vida \\ \hline
    \end{tabularx}
    \end{adjustbox}
    \caption{Características de \textit{BotGhost}.}
\end{table}

\begin{figure}[H]
	\centering
	\includegraphics[width=1\textwidth]{img/botghost.png}
	\caption{Interfaz web de \textit{BotGhost}.}
\end{figure}

\subsection{Herramientas de programación}

Actualmente existen multitud de librerías para distintos lenguajes de programación que permiten interactuar con la \textit{API} de \textit{Discord} y por tanto crear un bot. Así mismo existen herramientas híbridas que permiten esta creación de una manera más sencilla.

\subsubsection{Autocode}

\href{https://autocode.com/}{Autocode} es sin duda la herramienta mas interesante que existe de esta modalidad híbrida. Realmente es una plataforma que facilita la creación y despliegue de aplicaciones y servicios web, bots, y tareas de automatización permitiendo que los usuarios escriban sólo una parte del código (\textit{JavaScript}) de estos. Esta plataforma provee al usuario con un editor de código sencillo y de un explorador de archivos, recursos con los cuales puede crear el código.

De este modo el usuario sólo tiene que preocuparse por el código de la aplicación (o bot en este caso) que quiere crear, ya que del despliegue se encarga \textit{Autocode}. En su plan gratuito se pueden crear hasta 50 aplicaciones distintas, y permite la integración entre si de los distintos recursos que el usuario crea en la plataforma.

Sus características son:

\begin{table}[H]
    \centering
    \def\arraystretch{1.25}
    \begin{adjustbox}{max width=\textwidth}
    \begin{tabularx}{325px}{|l|L|}
    \hline
        \multicolumn{2}{|c|}{\textbf{\textit{Autocode}}} \\ \hline
    \hline
        \textbf{Tipo de comandos} & Predefinidos + \textit{JS} \\ \hline
        \textbf{Comandos reutilizables} & No \\ \hline
        \textbf{Control del despliegue} & Sí (limitado) \\ \hline
        \textbf{Número de bots} & 50 gratis \\ \hline
        \textbf{Experiencia} & Algo complejo \\ \hline
        \textbf{Personalización extra} & Requiere \textit{premium} (mensualidades) \\ \hline
        \textbf{Especialidad} & Despliegue general de aplicaciones \\ \hline
        \textbf{Logs} & Sí (1-30 días) \\ \hline
        \textbf{Premium} & \$180 / \$1620 al año \\ \hline
    \end{tabularx}
    \end{adjustbox}
    \caption{Resumen de soluciones actuales.}
\end{table}

\begin{figure}[H]
	\centering
	\includegraphics[width=1\textwidth]{img/autocode.png}
	\caption{Interfaz web de \textit{Autocode}.}
\end{figure}

\subsubsection{Librerías de programación}

Las librerías de programación dan libertad total a la hora de crear un bot de \textit{Discord}, lo cual puede ser ideal en algunos casos. Las ventajas son obvias, ya que se puede crear cualquier tipo de comando y la reutilización es sencilla, pero en cambio, la gestión del despliegue puede ser compleja.

Por lo general todas las librerías permiten realizar casi las mismas funcionalidades, diferenciándose en aspectos como el rendimiento, la comunidad que las soporta o la facilidad de uso.

Algunos ejemplos de librerías son:

\begin{itemize}
	\item \textbf{\textit{C\#}}: \href{https://discordnet.dev/}{\textit{Discord.NET}}, \href{https://github.com/DSharpPlus/DSharpPlus}{\textit{DSharpPlus}}
	\item \textbf{\textit{Java}}: \href{https://github.com/DV8FromTheWorld/JDA}{\textit{JDA}}, \href{https://discord4j.com/}{\textit{Discord4J}}
	\item \textbf{\textit{C++}}: \href{https://dpp.dev/}{\textit{D++}}
	\item \textbf{\textit{JavaScript}}: \href{https://discord.js.org/}{\textit{discord.js}}
	\item \textbf{\textit{Golang}}: \href{https://github.com/bwmarrin/discordgo}{\textit{DiscordGo}}
	\item \textbf{\textit{Ruby}}: \href{https://github.com/shardlab/discordrb}{\textit{discordrb}}
\end{itemize}


\subsection{Comparativa de tiempos}

En esta sección se hace una comparativa del tiempo medio de desarrollo desde cero de un bot de \textit{Discord} usando las herramientas anterior mencionadas. Además se incluye tiempos de desarrollo usando tres lenguajes de programación: \textit{C\#}, \textit{JavaScript} y \textit{Python}.

Las mediciones incluyen todos los pasos necesarios para crear uno de estos bots con dos comandos personalizados. En el caso de las herramientas de programación se incluye desde la creación del proyecto hasta el despliegue (en local) de este.

Los dos comandos personalizados se han elegido al ser comunes en todas las plataformas mencionadas, además de sencillos de implementar. Son los siguientes:

\begin{itemize}
	\item Envío de un mensaje.
	\item Envío de un mensaje recurrente (cada cierto tiempo).
\end{itemize}

\begin{table}[H]
    \centering
    \def\arraystretch{1.25}
    \begin{adjustbox}{max width=\textwidth}
    \begin{tabularx}{200px}{|l|R|}
    \hline
        \textbf{Herramienta} & \textbf{Tiempo (en minutos)} \\ \hline
    \hline
        ProBot & 5 \\ \hline
        Mee6 & 5 \\ \hline
        BotGhost & 10 \\ \hline
    \hline
        Autocode (JS) & 45 \\ \hline
    \hline
        JS & 85 \\ \hline
        C\# & 100 \\ \hline
        Python & 80 \\ \hline
    \end{tabularx}
    \end{adjustbox}
    \caption{Comparativa de tiempos}
\end{table}

\section{Discusión}

Como se puede observar existen multitud de posibilidades a la hora de crear un bot de \textit{Discord}, y, aunque cumplen lo que prometen, se centran en aspectos muy concretos dejando otros bastante desatendidos.

En las herramientas \textit{no-code} los bots se centran principalmente en tareas de moderación, envío de mensajes, estadísticas e integración con videojuegos y redes sociales. Además, para sacarles partido es necesario el uso de los paquetes \textit{premium}, dejando de lado en el plan gratuito detalles específicos (y que serían ideales) como:

\begin{itemize}
	\item Reutilización de comandos.
	\item Creación de comandos con funcionalidad específica.
	\item Control del despliegue de los bots.
	\item Creación de distintos bots con distintas funcionalidades en un mismo sistema.
\end{itemize}

En el caso de las librerías de programación, aunque todas permiten el acceso a la \textit{API} de \textit{Discord}, cada una de ellas tiene una estructura distinta y los procedimientos para crear un bot o comandos son más o menos complejos. Si un usuario decidiese utilizarlas tendría flexibilidad completa a la hora de crear una estructura concreta, pero entonces tendría que dedicar en ese caso un tiempo necesario para diseñar algo funcional.

\textit{Autocode} es una buena alternativa a las soluciones anteriores, ya que se evita el tener que gestionar el despliegue de los bots y se eliminan algunas trabas de gestión del código, pero al igual que el uso de librerías toda la lógica recae en el usuario final. Esto puede ser útil en ciertos casos, pero no siempre.

En la comparativa de tiempos anterior se puede observar que las herramientas \textit{no-code} son las más rápidas. Esto se debe a que solo es necesario agregar el bot al servidor de \textit{Discord} deseado, y tras eso configurar de manera sencilla los comandos.

En menos de 10 minutos se puede incluir una gran cantidad de funcionalidad a un servidor de \textit{Discord} de manera gratuita, algo que puede ser muy útil para la basta mayoría de usuarios de \textit{Discord}, pero cuando se necesitan funcionalidades específicas entonces no es el sistema ideal.

En cambio, cuando se usan herramientas que hacen uso de código, el tiempo de implementación se incrementa considerablemente. No hace justicia la comparativa, ya que en este caso se tiene que desarrollar el software por completo, por lo que es obvio que el tiempo es mayor.

En definitiva, no existe ninguna herramienta sencilla que brinde lo mejor de ambas alternativas. Por un lado se quiere facilitar la creación de bots y comandos, y el despliegue de estos. Por otro se quiere poder ampliar el repertorio de comandos disponible de manera sencilla, sin tener que desarrollar una aplicación completa para ello.

	\chapter{Análisis}

En este capítulo se profundiza en el análisis del problema planteado en capítulos anteriores, describiendo los diferentes actores, casos de uso, historias de usuario y los \textit{user journeys} asociados a estas HU.

\section{Actores}

En el sistema hay un único actor, es el siguiente:

\begin{table}[H]
    \centering
    \def\arraystretch{1.25}
    \begin{adjustbox}{max width=\textwidth}
    \begin{tabularx}{\textwidth}{|l|L|}
    \hline
        \textbf{Actor} & \textbf{Administrador} \\ \hline
    \hline
        Descripción & Crea los bots y los comandos. Tiene poder de configuración y de despliegue. Puede ser también usuario del bot haciendo uso de sus comandos. Agrega los bots a los servidores de \textit{Discord}. \\ \hline
        Beneficio & Obtiene un software que le permite crear bots y comandos de forma sencilla sin tener que usar herramientas de programación, a la vez que puede ampliar los comandos disponibles usando estas herramientas. \\ \hline
    \end{tabularx}
    \end{adjustbox}
    \caption{Actor 1. Administrador.}
\end{table}



\section{Casos de uso}

En esta sección se presentan de manera detallada los diferentes casos de uso.

\begin{table}[H]
    \centering
    \def\arraystretch{1.25}
    \begin{adjustbox}{max width=\textwidth}
    \begin{tabularx}{\textwidth}{|l|L|}
    \hline
        \textbf{Caso de uso} & \textbf{01 - Crear un bot} \\ \hline
    \hline
        Tipo & Primario \\ \hline
        Propósito & Crear un bot para después configurarlos y desplegarlos en distintos \textit{workers}. \\ \hline
        Referencias & \hyperref[sec:hu01]{HU-01} \\ \hline
        Precondición & El usuario provee al sistema de los datos necesarios para crear el bot. El sistema se encuentra disponible, instalado en una máquina administrada por el usuario.\\ \hline
        Postcondición & El bot es creado y queda disponible para ser configurado y desplegado. \\ \hline
        Comentarios adicionales & El usuario ha instalado previamente 
    \end{tabularx}
    \end{adjustbox}
    \caption{Caso de uso 01. Crear un bot.}
\end{table}

\begin{table}[H]
    \centering
    \def\arraystretch{1.25}
    \begin{adjustbox}{max width=\textwidth}
    \begin{tabularx}{\textwidth}{|l|L|}
    \hline
        \textbf{Caso de uso} & \textbf{02 - Editar un bot} \\ \hline
    \hline
        Tipo & Primario \\ \hline
        Propósito & Editar los parámetros y comandos de un bot. \\ \hline
        Referencias & \hyperref[sec:hu03]{HU-03}\\ \hline
        Precondición & El usuario provee al sistema de los nuevos parámetros del bot, o de los comandos que quiere modificar. \\ \hline
        Postcondición & El bot queda modificado. \\ \hline
    \end{tabularx}
    \end{adjustbox}
    \caption{Caso de uso 02. Editar un bot.}
\end{table}

\begin{table}[H]
    \centering
    \def\arraystretch{1.25}
    \begin{adjustbox}{max width=\textwidth}
    \begin{tabularx}{\textwidth}{|l|L|}
    \hline
        \textbf{Caso de uso} & \textbf{03 - Eliminar un bot} \\ \hline
    \hline
        Tipo & Primario \\ \hline
        Propósito & Eliminar un bot cuando no es necesario. \\ \hline
        Referencias & \hyperref[sec:hu04]{HU-04} \\ \hline
        Precondición & El usuario provee al sistema de los datos necesarios del bot que quiere eliminar. \\ \hline
        Postcondición & El bot queda eliminado. \\ \hline
    \end{tabularx}
    \end{adjustbox}
    \caption{Caso de uso 03. Eliminar un bot.}
\end{table}

\begin{table}[H]
    \centering
    \def\arraystretch{1.25}
    \begin{adjustbox}{max width=\textwidth}
    \begin{tabularx}{\textwidth}{|l|L|}
    \hline
        \textbf{Caso de uso} & \textbf{04 - Crear un comando} \\ \hline
    \hline
        Tipo & Primario \\ \hline
        Propósito & Crear un comando para después configurarlo en un bot. \\ \hline
        Referencias & \hyperref[sec:hu05]{HU-05} \\ \hline
        Precondición & El usuario provee al sistema de los datos necesarios para crear el comando. \\ \hline
        Postcondición & El comando es creado y queda disponible para ser configurado en bots. \\ \hline
    \end{tabularx}
    \end{adjustbox}
    \caption{Caso de uso 04. Crear un comando.}
\end{table}

\begin{table}[H]
    \centering
    \def\arraystretch{1.25}
    \begin{adjustbox}{max width=\textwidth}
    \begin{tabularx}{\textwidth}{|l|L|}
    \hline
        \textbf{Caso de uso} & \textbf{05 - Editar un comando} \\ \hline
    \hline
        Tipo & Primario \\ \hline
        Propósito & Editar los parámetros de un comando. \\ \hline
        Referencias & \hyperref[sec:hu07]{HU-07} \\ \hline
        Precondición & El usuario provee al sistema de los nuevos parámetros del bot, o de los comandos que quiere modificar. \\ \hline
        Postcondición & El bot queda modificado. \\ \hline
    \end{tabularx}
    \end{adjustbox}
    \caption{Caso de uso 05. Editar un comando.}
\end{table}

\begin{table}[H]
    \centering
    \def\arraystretch{1.25}
    \begin{adjustbox}{max width=\textwidth}
    \begin{tabularx}{\textwidth}{|l|L|}
    \hline
        \textbf{Caso de uso} & \textbf{06 - Eliminar un comando} \\ \hline
    \hline
        Tipo & Primario \\ \hline
        Propósito & Eliminar un comando cuando no es necesario. \\ \hline
        Referencias & \hyperref[sec:hu08]{HU-08} \\ \hline
        Precondición & El usuario provee al sistema de los datos necesarios del comando que quiere eliminar. \\ \hline
        Postcondición & El comando queda eliminado. \\ \hline
    \end{tabularx}
    \end{adjustbox}
    \caption{Caso de uso 06. Eliminar un comando.}
\end{table}

\begin{table}[H]
    \centering
    \def\arraystretch{1.25}
    \begin{adjustbox}{max width=\textwidth}
    \begin{tabularx}{\textwidth}{|l|L|}
    \hline
        \textbf{Caso de uso} & \textbf{07 - Desplegar un bot} \\ \hline
    \hline
        Tipo & Primario \\ \hline
        Propósito & Desplegar un bot para que esté disponible en los servidores de \textit{Discord}. \\ \hline
        Referencias & \hyperref[sec:hu10]{HU-10} \\ \hline
        Precondición & El usuario provee al sistema de los datos necesarios del bot que quiere desplegar. Al menos un \textit{worker} se encuentra activo. \\ \hline
        Postcondición & El bot es desplegado en uno de los workers y queda disponible en los servidores de \textit{Discord}. \\ \hline
    \end{tabularx}
    \end{adjustbox}
    \caption{Caso de uso 07. Desplegar un bot.}
\end{table}

\begin{table}[H]
    \centering
    \def\arraystretch{1.25}
    \begin{adjustbox}{max width=\textwidth}
    \begin{tabularx}{\textwidth}{|l|L|}
    \hline
        \textbf{Caso de uso} & \textbf{08 - Cancelar el despliegue de un bot} \\ \hline
    \hline
        Tipo & Primario \\ \hline
        Propósito & Parar la ejecución de un bot para que deje de estar disponible en los servidores de \textit{Discord}. \\ \hline
        Referencias & \hyperref[sec:hu11]{HU-11} \\ \hline
        Precondición & El usuario provee al sistema de los datos necesarios del bot del que quiere cancelar el despliegue. \\ \hline
        Postcondición & El bot deja de ejecutarse y deja de estar disponible en los servidores de \textit{Discord}. \\ \hline
    \end{tabularx}
    \end{adjustbox}
    \caption{Caso de uso 08. Cancelar el despliegue de un bot.}
\end{table}

\begin{table}[H]
    \centering
    \def\arraystretch{1.25}
    \begin{adjustbox}{max width=\textwidth}
    \begin{tabularx}{\textwidth}{|l|L|}
    \hline
        \textbf{Caso de uso} & \textbf{09 - Interfaz de usuario} \\ \hline
    \hline
        Tipo & Primario \\ \hline
        Propósito & Realizar las tareas de gestión de bots y comandos mediante una interfaz de usuario. \\ \hline
        Referencias & \hyperref[sec:hu12]{HU-12} \\ \hline
        Precondición & El sistema provee a la interfaz de usuario de toda la información referente a los bots y comandos. \\ \hline
        Postcondición & El usuario realiza las tareas de gestión de estos bots y comandos. \\ \hline
    \end{tabularx}
    \end{adjustbox}
    \caption{Caso de uso 09. Cancelar el despliegue de un bot.}
\end{table}


\section{Personas}

\subsection{Administrador}
\label{sec:personaAdmin}

\begin{table}[H]
    \centering
    \def\arraystretch{1.25}
    \begin{adjustbox}{max width=\textwidth}
    \begin{tabularx}{\textwidth}{|l|L|}
    \hline
        \textbf{Nombre} & \textbf{David Infante} \\ \hline
    \hline
        Rol & Administrador \\ \hline
        Descripción & · 24 años.\linebreak · Disfruta de las tardes con sus amigos en \textit{Discord} jugando a sus videojuegos favoritos. \\ \hline
        Intereses & · \textit{Discord}, ya que le parece una herramienta muy potente.\linebreak · Videojuegos, le encantan los \textit{shooters}.\linebreak · Automatización de tareas repetitivas, ya que odia hacer lo mismo continuamente.\linebreak · Programación, ya que disfruta creando software para facilitar su día a día.\linebreak · Monitorización, le gusta saber que todo el software que despliega funciona correctamente.\linebreak · Creación de servidores de juegos, para jugar con sus amigos y no tener que depender de servidores de terceros. \\ \hline
        Formación & Ingeniero informático.\linebreak\linebreak Tiene conocimientos avanzados en:\linebreak · Configuración de \textit{Discord}.\linebreak · Administración de sistemas.\linebreak · Despliegue de software y sistemas. \\ \hline
        Frustraciones & Tener que realizar tareas repetitivas. \\ \hline
        Necesidades & Un software o sistema que le permita programar las tareas repetitivas de monitorización y automatización que tanto odia. Usa mucho \textit{Discord}, por lo que piensa que sería útil que el software estuviera integrado con esa herramienta. \\ \hline
    \end{tabularx}
    \end{adjustbox}
    \caption{Persona 1. Administrador.}
\end{table}


\subsection{Usuario de \textit{Discord}}
\label{sec:personaUsuarioDiscord}
\begin{table}[H]
    \centering
    \def\arraystretch{1.25}
    \begin{adjustbox}{max width=\textwidth}
    \begin{tabularx}{\textwidth}{|l|L|}
    \hline
        \textbf{Nombre} & \textbf{Jorge Pulido} \\ \hline
    \hline
        Rol & Usuario de \textit{Discord} \\ \hline
        Descripción & · 25 años.\linebreak · Amigo de Jorge Cancho. No tiene conocimientos de programación ni de temas relacionados con la ingeniería o la informática. Le gusta disfrutar de las tardes con sus amigos en \textit{Discord}. \\ \hline
        Intereses & · Videojuegos, dedica la mayor parte de su tiempo a jugar con sus amigos.\linebreak · La facilidad de las cosas, no le gusta complicarse la vida.\linebreak · \textit{Discord}, le parece una herramienta muy útil, ya que la usa con sus amigos y para temas laborales. \\ \hline
        Formación & Magisterio de educación primaria. \\ \hline
        Frustraciones & No le gusta nada tener que indagar en detalles técnicos al jugar a videojuegos con sus amigos. Entiende que en ocasiones es necesaria alguna configuración para poder jugar (como acceder a un servidor), pero quiere que ese proceso sea lo más fácil posible. No le gusta tener que recordar esos detalles. \\ \hline
        Necesidades & Una herramienta que le permita acceder a esos detalles sin preocuparse de recordarlos o de consultarlos de manera extraña. \\ \hline
    \end{tabularx}
    \end{adjustbox}
    \caption{Persona 2. Usuario de \textit{Discord}.}
\end{table}

\subsection{Miembro del tribunal}
\label{sec:personaMiembroTribunal}
\begin{table}[H]
    \centering
    \def\arraystretch{1.25}
    \begin{adjustbox}{max width=\textwidth}
    \begin{tabularx}{\textwidth}{|l|L|}
    \hline
        \textbf{Nombre} & \textbf{Blanca Casado} \\ \hline
    \hline
        Rol & Miembro del tribunal \\ \hline
        Descripción & · 52 años.\linebreak · Su conocimiento en informática es muy elevado, pero no tiene tanta destreza con las distintas aplicaciones de mensajería instantánea que han surgido en los últimos años. \\ \hline
        Intereses & · Procesamiento en segundo plano.\linebreak · Redes neuronales.\linebreak · Desarrollo ágil. \\ \hline
        Formación & Catedrática en informática. \\ \hline
        Frustraciones & No le gusta enfrentarse a documentaciones poco precisas o de dudosa credibilidad. \\ \hline
        Necesidades & Una documentación y una presentación acorde a los criterios de evaluación de TFM que le permita evaluar al estudiante. \\ \hline
    \end{tabularx}
    \end{adjustbox}
    \caption{Persona 3. Miembro del tribunal.}
\end{table}

\section{Historias de usuario}

Para la creación de las historias de usuarios se ha usado la siguiente estructura.

\begin{table}[H]
    \centering
    \def\arraystretch{1.25}
    \begin{adjustbox}{max width=\textwidth}
    \begin{tabularx}{\textwidth}{|l|L|}
    \hline
        \textbf{Sección} & \textbf{Significado} \\ \hline
    \hline
        Resumen & Breve resumen de la historia de usuario. \\ \hline
        Meta & Qué se quiere conseguir. \\ \hline
        Beneficio & El beneficio de la historia de usuario. \\ \hline
        Perfil de usuario & Perfil del usuario que genera la historia de usuario. \\ \hline
        Escenario & Escenario de la historia de usuario. Se deben especificar detalles más concretos.\linebreak · Dado …\linebreak · Cuando …\linebreak · Entonces … \\ \hline
        Notas funcionales & Notas adicionales de carácter funcional que ayudan a comprender mejor el alcance de la historia de usuario. \\ \hline
        Notas técnicas & Notas adicionales de carácter técnico que ayudan a comprender mejor este tipo de detalles a la hora de desarrollar la historia de usuario. \\ \hline
        Dependencias & Posibles dependencias que tenga la historia de usuario. Éstas pueden ser otras historias de usuario, tareas que se estén llevando a cabo, etc. \\ \hline
        Tareas de seguimiento & Una vez analizada la historia de usuario, las tareas que se deben realizar a continuación. \\ \hline
        Criterio de aceptación & Criterio por el cual se va a determinar que la historia de usuario ha sido completada con éxito y por tanto finalizada. \\ \hline
    \end{tabularx}
    \end{adjustbox}
    \caption{Resumen historias de usuario.}
\end{table}

\bigskip

Por otro lado, se han creado las siguientes historias de usuario. Estas se encuentran en el \href{https://github.com/harvestcore/matroos}{repositorio} de \textit{GitHub} del proyecto, en la sección \href{https://github.com/harvestcore/matroos/labels/US}{\textit{Issues}}.

\begin{enumerate}
	\item \href{https://github.com/harvestcore/matroos/issues/1}{Crear diferentes bots de \textit{Discord}}
	\item \href{https://github.com/harvestcore/matroos/issues/2}{Consultar datos de un bot}
	\item \href{https://github.com/harvestcore/matroos/issues/3}{Editar un bot}
	\item \href{https://github.com/harvestcore/matroos/issues/4}{Eliminar un bot}
	\item \href{https://github.com/harvestcore/matroos/issues/5}{Crear diferentes comandos de \textit{Discord}}
	\item \href{https://github.com/harvestcore/matroos/issues/6}{Consultar datos de un comando}
	\item \href{https://github.com/harvestcore/matroos/issues/7}{Editar un comando}
	\item \href{https://github.com/harvestcore/matroos/issues/8}{Eliminar un comando}
	\item \href{https://github.com/harvestcore/matroos/issues/9}{Lanzar bots}
	\item \href{https://github.com/harvestcore/matroos/issues/10}{Cancelar ejecución de bots}
	\item \href{https://github.com/harvestcore/matroos/issues/11}{Consultar estado de los despliegues}
	\item \href{https://github.com/harvestcore/matroos/issues/25}{Interfaz de usuario}
	\item \href{https://github.com/harvestcore/matroos/issues/39}{Criterios de evaluación}
\end{enumerate}


\subsection{HU-01 - Crear diferentes bots de \textit{Discord}}
\label{sec:hu01}

\begin{table}[H]
    \centering
    \def\arraystretch{1.25}
    \begin{adjustbox}{max width=\textwidth}
    \begin{tabularx}{\textwidth}{|l|L|}
    \hline
        \textbf{Sección} & \textbf{Contenido} \\ \hline
    \hline
        Resumen & Como usuario administrador quiero crear distintos bots de \textit{Discord} en el sistema para poder configurarlos con los comandos que yo quiera para que Estos realicen las tareas que deseo al ejecutar los comandos en los canales donde he agregado los bots. \\ \hline
        Meta & Creación de bots para que más tarde puedan ser configurados y desplegados y para que puedan ejecutar los comandos configurados en ellos. \\ \hline
        Beneficio & De este modo, cada bot puede tener una funcionalidad específica configurada, sin tener que compartir un bot con funcionalidades de distintos ámbitos. \\ \hline
        Perfil de usuario & \hyperref[sec:personaAdmin]{Administrador} \\ \hline
        Escenario & · Dado: que quiero crear un bot de \textit{Discord} y configurar sus funcionalidades,\linebreak · Cuando: proveo al sistema de una \textit{key} de bot de \textit{Discord} y de un nombre para el bot,\linebreak · Entonces: el sistema crea un bot para que pueda empezar a configurarlo. \\ \hline
        Notas funcionales & · No se debe permitir la creación de bots con el mismo nombre.\linebreak · No se debe permitir la creación de bots con la misma \textit{key}. \\ \hline
        Notas técnicas & Elementos necesarios para la creación del bot:\linebreak · \verb|key: str|. Key del bot, proporcionada en el panel de control de \textit{Discord}.\linebreak · \verb|name: str|. Nombre del bot.\linebreak \linebreak Parámetros extra necesarios para crear un bot:\linebreak · \verb|id: Guid|. Identificador único para cada bot. \\ \hline
        Dependencias & – \\ \hline
        Tareas de seguimiento & – \\ \hline
        Criterio de aceptación & · Los bots se pueden crear correctamente.\linebreak · El usuario es avisado en caso de error al crear un bot.\linebreak · Tests unitarios y integración son creados dentro de lo posible. \\ \hline
    \end{tabularx}
    \end{adjustbox}
    \caption{HU-01. Crear diferentes bots de \textit{Discord}.}
\end{table}


\subsection{HU-02 - Consultar datos de un bot}
\label{sec:hu02}

\begin{table}[H]
    \centering
    \def\arraystretch{1.25}
    \begin{adjustbox}{max width=\textwidth}
    \begin{tabularx}{\textwidth}{|l|L|}
    \hline
        \textbf{Sección} & \textbf{Contenido} \\ \hline
    \hline
        Resumen & Como usuario administrador quiero consultar los detalles de los bots de \textit{Discord} creados en el sistema, para poder ver sus características y su configuración. \\ \hline
        Meta & Obtener todos los datos de un bot en concreto o de todos los bots creados en el sistema para consultar sus características y configuración. \\ \hline
        Beneficio & Consultar la configuración única y específica de cada uno de los bots. \\ \hline
        Perfil de usuario & \hyperref[sec:personaAdmin]{Administrador} \\ \hline
        Escenario & · Dado: que quiero consultar los detalles de los bots creados en el sistema,\linebreak · Cuando: hago una petición al sistema para ello,\linebreak · Entonces: el sistema me devuelve los detalles y datos de los bots. \\ \hline
        Notas funcionales & · Se debe permitir obtener los detalles de todos los bots.\linebreak · Se debe permitir obtener los detalles de un bot en específico. \\ \hline
        Notas técnicas & Parámetros necesarios para obtener los detalles de un bot específico:\linebreak · \verb|id: Guid|. Identificador único para cada bot. \\ \hline
        Dependencias & – \\ \hline
        Tareas de seguimiento & – \\ \hline
        Criterio de aceptación & · Todos los datos de los bots son devueltos.\linebreak · El usuario es avisado en caso de error al obtener los detalles de los bots.\linebreak · Tests unitarios y integración son creados dentro de lo posible. \\ \hline
    \end{tabularx}
    \end{adjustbox}
    \caption{HU-02. Consultar datos de un bot.}
\end{table}

\subsection{HU-03 - Editar un bot}
\label{sec:hu03}

\begin{table}[H]
    \centering
    \def\arraystretch{1.25}
    \begin{adjustbox}{max width=\textwidth}
    \begin{tabularx}{\textwidth}{|l|L|}
    \hline
        \textbf{Sección} & \textbf{Contenido} \\ \hline
    \hline
        Resumen & Como usuario administrador quiero poder editar los detalles de un bot de \textit{Discord}, para configurarle comandos nuevos, eliminar existentes o cambiar sus parámetros. \\ \hline
        Meta & Permitir la edición de los parámetros de los bots (comandos, nombre, key, etc). \\ \hline
        Beneficio & Ampliar, reducir o modificar las funcionalidades específicas de los bots. \\ \hline
        Perfil de usuario & \hyperref[sec:personaAdmin]{Administrador} \\ \hline
        Escenario & · Dado: que quiero modificar los datos de un bot,\linebreak · Cuando: proveo al sistema del identificador del bot a editar y los datos que se deben modificar,\linebreak · Entonces: el sistema modifica los datos del bot. \\ \hline
        Notas funcionales & · No se debe permitir la existencia de bots con el mismo nombre.\linebreak · No se debe permitir la existencia de bots con la misma \textit{key}.\linebreak · En caso de agregar un comando a un bot, el comando debe estar creado previamente en el sistema.\linebreak · No se debe permitir la existencia comandos iguales en un mismo bot.\linebreak · En caso de que el bot se encuentre desplegado, debe reiniciarse el despliegue una vez la edición finaliza. \\ \hline
        Notas técnicas & Parámetros necesarios para obtener los detalles de un bot específico y para modificarlo:\linebreak · \verb|id: Guid|. Identificador único para cada bot.\linebreak · key: str. Key del bot, proporcionada en el panel de control de \textit{Discord}.\linebreak · \verb|name: str|. Nombre del bot.\linebreak · \verb|command_id: Guid|. Identificador único para cada comando. \\ \hline
        Dependencias & \hyperref[sec:hu01]{HU-01} \\ \hline
        Tareas de seguimiento & – \\ \hline
        Criterio de aceptación & · El bot es modificado correctamente.\linebreak · El usuario es avisado en caso de error al modificar los datos del bot.\linebreak · Tests unitarios y integración son creados dentro de lo posible. \\ \hline
    \end{tabularx}
    \end{adjustbox}
    \caption{HU-03. Editar un bot.}
\end{table}

\subsection{HU-04 - Eliminar un bot}
\label{sec:hu04}

\begin{table}[H]
    \centering
    \def\arraystretch{1.25}
    \begin{adjustbox}{max width=\textwidth}
    \begin{tabularx}{\textwidth}{|l|L|}
    \hline
        \textbf{Sección} & \textbf{Contenido} \\ \hline
    \hline
        Resumen & Como usuario administrador quiero poder eliminar un bot de \textit{Discord}. \\ \hline
        Meta & Permitir el borrado de bots. \\ \hline
        Beneficio & Cuando ya no es necesario un bot, se puede eliminar. \\ \hline
        Perfil de usuario & \hyperref[sec:personaAdmin]{Administrador} \\ \hline
        Escenario & · Dado: que quiero eliminar un bot,\linebreak · Cuando: proveo al sistema del identificador del bot a eliminar,\linebreak · Entonces: el sistema elimina el bot y sus datos asociados. \\ \hline
        Notas funcionales & ·  En caso de que el bot se encuentre desplegado, éste despliegue debe cancelarse. \\ \hline
        Notas técnicas & Parámetros necesarios para eliminar un bot:\linebreak · \verb|id: Guid|. Identificador único para cada bot. \\ \hline
        Dependencias & \hyperref[sec:hu01]{HU-01} \\ \hline
        Tareas de seguimiento & – \\ \hline
        Criterio de aceptación & · El bot es eliminado correctamente.\linebreak · El usuario es avisado en caso de error al eliminar los datos del bot.\linebreak · Tests unitarios y integración son creados dentro de lo posible. \\ \hline
    \end{tabularx}
    \end{adjustbox}
    \caption{HU-04. Eliminar un bot.}
\end{table}


\subsection{HU-05 - Crear diferentes comandos de \textit{Discord}}
\label{sec:hu05}

\begin{table}[H]
    \centering
    \def\arraystretch{1.25}
    \begin{adjustbox}{max width=\textwidth}
    \begin{tabularx}{\textwidth}{|l|L|}
    \hline
        \textbf{Sección} & \textbf{Contenido} \\ \hline
    \hline
        Resumen & Como usuario administrador quiero crear distintos comandos para los bots de \textit{Discord} para que Estos realicen las tareas que deseo tras ejecutarlos en los canales de \textit{Discord}. \\ \hline
        Meta & Creación de distintos comandos que posteriormente puedan ser asignados a bots. \\ \hline
        Beneficio & De este modo, cada comando puede tener una tarea específica configurada y accesible (y ejecutable) dentro de un servidor de \textit{Discord}. \\ \hline
        Perfil de usuario & \hyperref[sec:personaAdmin]{Administrador} \\ \hline
        Escenario & · Dado: que quiero crear un comando de \textit{Discord} y configurar sus funcionalidades,\linebreak · Cuando: proveo al sistema de un nombre para el comando y de un prefijo,\linebreak · Entonces: el sistema crea un comando para que pueda empezar a configurarlo. \\ \hline
        Notas funcionales & · No se debe permitir la creación de comandos con el mismo nombre. \\ \hline
        Notas técnicas & Elementos necesarios para la creación del comando:\linebreak · \verb|name: str|. Nombre del comando.\linebreak · \verb|prefix: str|. Prefijo del comando.\linebreak · \verb|type: CommandType|. El tipo de comando.\linebreak · \verb|params: obj|. Parámetros del comando. \\ \hline
        Dependencias & – \\ \hline
        Tareas de seguimiento & – \\ \hline
        Criterio de aceptación & · Los comandos se crean correctamente.\linebreak · El usuario es avisado en caso de error al crear un comando.\linebreak · Tests unitarios y integración son creados dentro de lo posible. \\ \hline
    \end{tabularx}
    \end{adjustbox}
    \caption{HU-05. Crear diferentes comandos de \textit{Discord}.}
\end{table}

\subsection{HU-06 - Consultar datos de un comando}
\label{sec:hu06}

\begin{table}[H]
    \centering
    \def\arraystretch{1.25}
    \begin{adjustbox}{max width=\textwidth}
    \begin{tabularx}{\textwidth}{|l|L|}
    \hline
        \textbf{Sección} & \textbf{Contenido} \\ \hline
    \hline
        Resumen & Como usuario administrador quiero consultar los detalles de los comandos creados en el sistema. \\ \hline
        Meta & Obtener todos los datos de un comando en concreto o de todos los comandos creados en el sistema. \\ \hline
        Beneficio & Consultar la configuración única y específica de cada uno de los comandos. \\ \hline
        Perfil de usuario & \hyperref[sec:personaAdmin]{Administrador} \\ \hline
        Escenario & · Dado: que quiero consultar los detalles de los comandos creados en el sistema,\linebreak · Cuando: hago una petición al sistema para ello,\linebreak · Entonces: el sistema me devuelve los datos de los comandos. \\ \hline
        Notas funcionales & · Se debe permitir obtener los detalles de todos los comandos.\linebreak · Se debe permitir obtener los detalles de un comando en específico. \\ \hline
        Notas técnicas & Parámetros necesarios para obtener los detalles de un comando específico:\linebreak · \verb|id: Guid|. Identificador único para cada comando. \\ \hline
        Dependencias & \hyperref[sec:hu05]{HU-05} \\ \hline
        Tareas de seguimiento & – \\ \hline
        Criterio de aceptación & · Todos los datos de los comandos son devueltos.\linebreak · El usuario es avisado en caso de error al obtener los detalles de los comandos.\linebreak · Tests unitarios y integración son creados dentro de lo posible. \\ \hline
    \end{tabularx}
    \end{adjustbox}
    \caption{HU-06. Consultar datos de un comando.}
\end{table}

\subsection{HU-07 - Editar un comando}
\label{sec:hu07}

\begin{table}[H]
    \centering
    \def\arraystretch{1.25}
    \begin{adjustbox}{max width=\textwidth}
    \begin{tabularx}{\textwidth}{|l|L|}
    \hline
        \textbf{Sección} & \textbf{Contenido} \\ \hline
    \hline
        Resumen & Como usuario administrador quiero poder editar los detalles de un comando para . \\ \hline
        Meta & La edición de los parámetros de los comandos (nombre, prefijo, parámetros, etc). \\ \hline
        Beneficio & Modificar la configuración de la tarea que ejecuta dicho comando. \\ \hline
        Perfil de usuario & \hyperref[sec:personaAdmin]{Administrador} \\ \hline
        Escenario & · Dado: que quiero modificar los datos de un comando,\linebreak · Cuando: proveo al sistema del identificador del comando a editar y los datos que se deben modificar,\linebreak · Entonces: el sistema modifica los datos del comando. \\ \hline
        Notas funcionales & · No se debe permitir la existencia de comandos con el mismo nombre.\linebreak · En caso de que el comando esté siendo usado por un bot que se encuentre desplegado, este debe ser reiniciado. \\ \hline
        Notas técnicas & Parámetros necesarios para obtener los detalles de un bot específico y para modificarlo:\linebreak · \verb|id: Guid|. Identificador único para cada bot.\linebreak · \verb|prefix: str|. Prefijo del comando.\linebreak · \verb|params: dict|. Parámetros adicionales del comando. \\ \hline
        Dependencias & \hyperref[sec:hu05]{HU-05} \\ \hline
        Tareas de seguimiento & – \\ \hline
        Criterio de aceptación & · El comando es modificado correctamente.\linebreak · El usuario es avisado en caso de error al modificar los datos del comando.\linebreak · Tests unitarios y integración son creados dentro de lo posible. \\ \hline
    \end{tabularx}
    \end{adjustbox}
    \caption{HU-07. Editar un comando.}
\end{table}

\subsection{HU-08 - Eliminar un comando}
\label{sec:hu08}

\begin{table}[H]
    \centering
    \def\arraystretch{1.25}
    \begin{adjustbox}{max width=\textwidth}
    \begin{tabularx}{\textwidth}{|l|L|}
    \hline
        \textbf{Sección} & \textbf{Contenido} \\ \hline
    \hline
        Resumen & Como usuario administrador quiero poder eliminar un comando para que deje de estar disponible en el sistema. \\ \hline
        Meta & El borrado de comandos en el sistema. \\ \hline
        Beneficio & Cuando ya no es necesario un comando, se puede eliminar. \\ \hline
        Perfil de usuario & \hyperref[sec:personaAdmin]{Administrador} \\ \hline
        Escenario & · Dado: que quiero eliminar un comando,\linebreak · Cuando: proveo al sistema del identificador del comando a eliminar,\linebreak · Entonces: el sistema elimina el comando y sus datos asociados. \\ \hline
        Notas funcionales & · En caso de que el comando esté en uso por un bot desplegado, éste debe reiniciarse. \\ \hline
        Notas técnicas & Parámetros necesarios para eliminar un bot:\linebreak · \verb|id: Guid|. Identificador único para cada bot. \\ \hline
        Dependencias & \hyperref[sec:hu05]{HU-05} \\ \hline
        Tareas de seguimiento & – \\ \hline
        Criterio de aceptación & · El comando es eliminado correctamente.\linebreak · El usuario es avisado en caso de error al eliminar los datos del comando.\linebreak · Tests unitarios y integración son creados dentro de lo posible. \\ \hline
    \end{tabularx}
    \end{adjustbox}
    \caption{HU-08. Eliminar un comando.}
\end{table}

\subsection{HU-09 - Lanzar bots}
\label{sec:hu09}

\begin{table}[H]
    \centering
    \def\arraystretch{1.25}
    \begin{adjustbox}{max width=\textwidth}
    \begin{tabularx}{\textwidth}{|l|L|}
    \hline
        \textbf{Sección} & \textbf{Contenido} \\ \hline
    \hline
        Resumen & Como usuario administrador quiero poder ejecutar los bots de \textit{Discord} que he creado y configurado previamente en el sistema para poder hacer uso de los comandos de los que disponen en los servidores de \textit{Discord}. \\ \hline
        Meta & Desplegar bots en los workers para que puedan usarse desde los servidores de \textit{Discord}. \\ \hline
        Beneficio & La ejecución de los bots permite que los comandos estén disponibles en los servidores de \textit{Discord} para los usuarios (una vez los bots se agreguen a esos servidores). \\ \hline
        Perfil de usuario & \hyperref[sec:personaAdmin]{Administrador} \\ \hline
        Escenario & · Dado: que quiero desplegar un bot de \textit{Discord},\linebreak · Cuando: proveo al sistema de un identificador de bot de \textit{Discord} existente,\linebreak · Entonces: el sistema despliega el bot automáticamente, lo que me permite empezar a hacer uso de sus comandos. \\ \hline
        Notas funcionales & · No se debe permitir el despliegue de un bot varias veces.\linebreak · No se debe permitir el despliegue de un bot que no tiene comandos configurados. \\ \hline
        Notas técnicas & Elementos necesarios para el despliegue del bot:\linebreak · \verb|id: Guid|. Identificador único para cada bot. \\ \hline
        Dependencias & \hyperref[sec:hu01]{HU-01}, \hyperref[sec:hu03]{HU-03}, \hyperref[sec:hu05]{HU-05}, \hyperref[sec:hu07]{HU-07} \\ \hline
        Tareas de seguimiento & – \\ \hline
        Criterio de aceptación & · El bot queda desplegado correctamente.\linebreak · El usuario es avisado en caso de error al desplegar un bot.\linebreak · Tests unitarios y integración son creados dentro de lo posible. \\ \hline
    \end{tabularx}
    \end{adjustbox}
    \caption{HU-09. Lanzar bots.}
\end{table}

\subsection{HU-10 - Cancelar ejecución de bots}
\label{sec:hu10}

\begin{table}[H]
    \centering
    \def\arraystretch{1.25}
    \begin{adjustbox}{max width=\textwidth}
    \begin{tabularx}{\textwidth}{|l|L|}
    \hline
        \textbf{Sección} & \textbf{Contenido} \\ \hline
    \hline
        Resumen & Como usuario administrador quiero terminar la ejecución de un bot de \textit{Discord} en el sistema para que deje de estar disponible. \\ \hline
        Meta & Terminar la ejecución de bots en los workers para que dejen de poder usarse desde los servidores de \textit{Discord}. \\ \hline
        Beneficio & Cuando sea necesario realizar mantenimiento a un bot (o cuando ya no sea necesario que esté en activo), se puede cancelar su ejecución. \\ \hline
        Perfil de usuario & \hyperref[sec:personaAdmin]{Administrador} \\ \hline
        Escenario & · Dado: que quiero cancelar el despliegue de un bot para realizar algún tipo de tarea de mantenimiento,\linebreak · Cuando: proveo al sistema de un identificador de bot de \textit{Discord} que se encuentre desplegado,\linebreak · Entonces: el sistema termina la ejecución del bot. \\ \hline
        Notas funcionales & · No se debe permitir cancelar el despliegue de un bot que no se encuentra desplegado. \\ \hline
        Notas técnicas & Parámetros necesarios para cancelar el despliegue de un bot:\linebreak · \verb|id: Guid|. Identificador único para cada bot. \\ \hline
        Dependencias & \hyperref[sec:hu01]{HU-01}, \hyperref[sec:hu03]{HU-03}, \hyperref[sec:hu05]{HU-05}, \hyperref[sec:hu07]{HU-07}, \hyperref[sec:hu09]{HU-09} \\ \hline
        Tareas de seguimiento & – \\ \hline
        Criterio de aceptación & · El despliegue del bot es cancelado.\linebreak · El usuario es avisado en caso de error al cancelar el despliegue del bot.\linebreak · Tests unitarios y integración son creados dentro de lo posible. \\ \hline
    \end{tabularx}
    \end{adjustbox}
    \caption{HU-10. Cancelar ejecución de bots.}
\end{table}

\subsection{HU-11 - Consultar estado de los despliegues}
\label{sec:hu11}

\begin{table}[H]
    \centering
    \def\arraystretch{1.25}
    \begin{adjustbox}{max width=\textwidth}
    \begin{tabularx}{\textwidth}{|l|L|}
    \hline
        \textbf{Sección} & \textbf{Contenido} \\ \hline
    \hline
        Resumen & Como usuario administrador quiero conocer el estado de los despliegues de los bots de \textit{Discord} en el sistema. \\ \hline
        Meta & Obtener detalles de los workers y de los bots que se encuentran desplegados en los workers. \\ \hline
        Beneficio & Consultar cuales de los bots están activos y cuales no, para tareas de monitorización del sistema. \\ \hline
        Perfil de usuario & \hyperref[sec:personaAdmin]{Administrador} \\ \hline
        Escenario & · Dado: que quiero conocer el estado de los despliegues de los bots,\linebreak · Cuando: hago una petición al sistema para ello,\linebreak · Entonces: el sistema me devuelve los datos de los bots que se encuentran desplegados. \\ \hline
        Notas funcionales & · No se deben devolver datos de configuración del bot.\linebreak · Se debe devolver información de los workers que ejecutan los bots. \\ \hline
        Notas técnicas & Datos a devolver:\linebreak · \verb|workers: []Worker|. Todos los workers que hay disponibles en el sistema.\linebreak \linebreak Worker:\linebreak · \verb|id: Guid|. Identificador único para cada worker.\linebreak · \verb|uptime: DateTime|. El tiempo de actividad del worker.\linebreak · \verb|location: str|. La URL donde se encuentra el worker.\linebreak · \verb|bots: []Bot|. Los bots que están desplegados en el worker.\linebreak \linebreak Bot:\linebreak · \verb|id: Guid|. Identificador único para cada bot.\linebreak · \verb|uptime: DateTime|. El tiempo de actividad del bot.\linebreak · ... \\ \hline
        Dependencias & \hyperref[sec:hu01]{HU-01}, \hyperref[sec:hu03]{HU-03}, \hyperref[sec:hu05]{HU-05}, \hyperref[sec:hu07]{HU-07}, \hyperref[sec:hu09]{HU-09}, \hyperref[sec:hu10]{HU-10} \\ \hline
        Tareas de seguimiento & – \\ \hline
        Criterio de aceptación & · Los datos de los bots de \textit{Discord} que se encuentran desplegados son devueltos al usuario.\linebreak · El usuario es avisado en caso de error al obtener los datos de despliegue.\linebreak · Tests unitarios y integración son creados dentro de lo posible. \\ \hline
    \end{tabularx}
    \end{adjustbox}
    \caption{HU-11. Consultar estado de los despliegues.}
\end{table}

\subsection{HU-12 - Interfaz de usuario}
\label{sec:hu12}

\begin{table}[H]
    \centering
    \def\arraystretch{1.25}
    \begin{adjustbox}{max width=\textwidth}
    \begin{tabularx}{\textwidth}{|l|L|}
    \hline
        \textbf{Sección} & \textbf{Contenido} \\ \hline
    \hline
        Resumen & Como usuario administrador quiero disponer de una interfaz gráfica para poder crear, configurar, desplegar y conocer el estado de los bots de \textit{Discord} que hay en el sistema. \\ \hline
        Meta & Disponer de una interfaz gráfica que permita realizar las tareas de creación, configuración y despliegue de bots de \textit{Discord}. \\ \hline
        Beneficio & Realizar todas las tareas de gestión de comandos y bots de manera sencilla en una interfaz de usuario, en lugar de hacerlas mediante el uso de una API. \\ \hline
        Perfil de usuario & \hyperref[sec:personaAdmin]{Administrador} \\ \hline
        Escenario & · Dado: que quiero administrar los bots de \textit{Discord},\linebreak · Cuando: accedo a la interfaz gráfica,\linebreak · Entonces: el sistema me permite realizar todas las tareas de administración de bots y comandos. \\ \hline
        Notas funcionales & – \\ \hline
        Notas técnicas & – \\ \hline
        Dependencias & \hyperref[sec:hu01]{HU-01}, \hyperref[sec:hu02]{HU-02}, \hyperref[sec:hu03]{HU-03}, \hyperref[sec:hu04]{HU-04}, \hyperref[sec:hu05]{HU-05}, \hyperref[sec:hu06]{HU-06}, \hyperref[sec:hu07]{HU-07}, \hyperref[sec:hu08]{HU-08}, \hyperref[sec:hu09]{HU-09}, \hyperref[sec:hu10]{HU-10}, \hyperref[sec:hu11]{HU-11} \\ \hline
        Tareas de seguimiento & – \\ \hline
        Criterio de aceptación & · La interfaz permite realizar las tareas de gestión de bots y comandos.\linebreak · El usuario es avisado en caso de producirse algún error.\linebreak · Tests unitarios y integración son creados dentro de lo posible. \\ \hline
    \end{tabularx}
    \end{adjustbox}
    \caption{HU-12. Interfaz de usuario.}
\end{table}

\subsection{HU-13 - Criterios de evaluación}
\label{sec:hu13}

\begin{table}[H]
    \centering
    \def\arraystretch{1.25}
    \begin{adjustbox}{max width=\textwidth}
    \begin{tabularx}{\textwidth}{|l|L|}
    \hline
        \textbf{Sección} & \textbf{Contenido} \\ \hline
    \hline
        Resumen & Como miembro del tribunal, quisiera disponer de una documentación, una presentación y un informe acordes a los criterios de evaluación para comprobar que Estos se han cumplido correctamente. \\ \hline
        Meta & Disponer de una documentación que recoja claramente toda la información referente al desarrollo del TFM. \\ \hline
        Beneficio & De este modo es más sencillo evaluar todo el trabajo que el alumno ha realizado para desarrollar el TFM. \\ \hline
        Perfil de usuario & \hyperref[sec:personaMiembroTribunal]{Miembro del tribunal} \\ \hline
        Escenario & · Dado: que quiero evaluar el trabajo realizado por el alumno,\linebreak · Cuando: éste me de acceso a dicha documentación acorde a los criterios de evaluación,\linebreak · Entonces: podré evaluar el trabajo del alumno. \\ \hline
        Notas funcionales & Los criterios de evaluación son:\linebreak \linebreak El estudiante…\linebreak · Utiliza fuentes de información variadas, válidas y fiables y selecciona la relevante para el objetivo del trabajo.\linebreak · Toma decisiones adecuadas al contexto y propone soluciones utilizando el conocimiento adquirido.\linebreak · Detecta y analiza oportunidades para hacer nuevas propuestas.\linebreak · Propone soluciones adecuadas y justifica las decisiones tomadas para resolver problemas complejos.\linebreak · Utiliza recursos formales e informales para documentar adecuadamente el proceso de desarrollo: concepción, planificación, análisis, diseño, implementación, pruebas, etc.\linebreak · Muestra claridad y comprensión en la redacción,organizando la información adecuadamente y utilizando los recursos adecuados para el discurso escrito. Muestra claridad y comprensión en la expresión oral, organizando la información adecuadamente y utilizando los recursos adecuados para el discurso oral. \\ \hline
        Notas técnicas & – \\ \hline
        Dependencias & – \\ \hline
        Tareas de seguimiento & – \\ \hline
        Criterio de aceptación & · La documentación cumple con los criterios de evaluación. \\ \hline
    \end{tabularx}
    \end{adjustbox}
    \caption{HU-13. Criterios de evaluación.}
\end{table}
 
\section{User journeys}

En esta sección se enumeran los \textit{user journeys} tanto para el uso del sistema, como para la ampliación del repertorio de comandos.

\subsection{Uso del sistema}

\subsubsection{Crear un bot}

\begin{enumerate}
	\item El usuario crea una aplicación de \textit{Discord} en el \href{https://discord.com/developers/applications}{portal de desarrolladores} y obtiene un \textit{token} para un bot.
	\item El usuario provee al sistema de este \textit{token} y de un nombre.
	\item[!] Si el nombre o el \textit{token} se encuentran en uso por otro bot, el sistema cancela la creación.
	\item El bot es creado en el sistema, quedando disponible para ser configurado con comandos o para ser desplegado.
\end{enumerate}

\subsubsection{Modificar un bot}

Se puede modificar un bot de dos maneras:

\begin{itemize}
	\item Parámetros del bot.
	\begin{enumerate}
		\item El usuario provee al sistema de un nombre o \textit{token} distinto a los actuales.
		\item[!] Si el nombre o el \textit{token} corresponden a otro bot, el sistema cancela la modificación.
		\item[!] Si el bot se encuentra desplegado, el sistema cancela el despliegue previa modificación.
		\item El sistema modifica los detalles del bot.
	\end{enumerate}
	
	\item Comandos del bot.
	\begin{enumerate}
		\item El usuario provee al sistema de los identificadores de los comandos que quiere agregar o eliminar del bot.
		\item[!] Si el bot se encuentra desplegado, el sistema cancela el despliegue previa modificación.
		\item El sistema modifica los comandos del bot.
	\end{enumerate}
\end{itemize}

\subsubsection{Eliminar un bot}

\begin{enumerate}
	\item El usuario provee al sistema del identificador del bot que quiere eliminar.
	\item[!] Si el bot se encuentra desplegado, el sistema cancela el despliegue previa eliminación.
	\item El sistema elimina todos los datos asociados al bot, no pudiendo volver a usarse.
\end{enumerate}

\subsubsection{Crear un comando}

\begin{enumerate}
	\item El usuario provee al sistema de un nombre, un prefijo, un tipo de comando y de los parámetros necesarios para ese tipo de comando.
	\item[!] Si el nombre se encuentra en uso, el sistema cancela la creación del comando.
	\item El sistema crea el comando.
\end{enumerate}

\subsubsection{Modificar un comando}

\begin{enumerate}
	\item El usuario provee al sistema de los datos que quiere modificar de un comando, estos incluyen el nombre, prefijo y parámetros.
	\item[!] Si los datos que el usuario provee (nombre y prefijo) corresponden a otro comando, el sistema cancela la modificación.
	\item[!] Si el comando se encuentra en uso por un bot que se encuentra desplegado, el sistema cancela el despliegue previa modificación.
	\item El sistema modifica el comando.
\end{enumerate}

\subsubsection{Eliminar un comando}

\begin{enumerate}
	\item El usuario provee al sistema del identificador del comando que quiere eliminar.
	\item[!] Si el comando se encuentra en uso por un bot que se encuentra desplegado, el sistema cancela el despliegue previa eliminación.
	\item El sistema elimina todos los datos asociados al comando, no pudiendo volver a agregarse a un bot.
\end{enumerate}

\subsubsection{Desplegar un bot}

\begin{enumerate}
	\item El usuario provee al sistema del identificador del bot que quiere desplegar, además del identificador del \textit{worker} donde quiere desplegarlo.
	\item[!] Si el bot ya se encuentra desplegado, el sistema cancela el despliegue.
	\item[!] Si el bot o el \textit{worker} no existen, el sistema cancela el despliegue.
	\item El sistema despliega el bot en el \textit{worker}.
\end{enumerate}

\subsubsection{Cancelar despliegue de un bot}

\begin{enumerate}
	\item El usuario provee al sistema del identificador del bot del que quiere cancelar el despliegue.
	\item[!] Si el bot no se encuentra desplegado, el sistema cancela la operación.
	\item[!] Si el bot no existe, el sistema cancela la operación.
	\item El sistema cancela el despliegue del bot.
\end{enumerate}


\subsection{Ampliación del repertorio de comandos}

Para la realización de este \textit{user journey} es necesario el uso de C\#.

\begin{enumerate}
	\item El usuario define el nuevo tipo de comando.
	\item El usuario define los parámetros que necesita ese comando.
	\item El usuario implementa la funcionalidad asociada al comando, esto es el código que se ejecuta cuando el comando es invocado.
	\item El comando queda disponible para ser usado por el sistema.
	\item[!] Es necesario reiniciar el sistema para que pueda utilizarse el nuevo comando.
\end{enumerate}

\section{Modelo de negocio}

La solución propuesta se caracteriza por ser software libre, pero los costos de desarrollo e implementación nunca son nulos.

Debido a la muy probable falta de financiación al inicio del desarrollo del software, los objetivos iniciales se centrarían en obtener la renta mínima para poder continuar con el desarrollo del proyecto. A medida que se supere este primer obstáculo y el software esté mejor establecido, el modelo de financiación cambiaría para lograr un mayor valor de mercado y ganancias.

Como modelo de negocio, teniendo en cuenta que se opta por una solución compuesta por software libre, con el fin de sufragar todos estos gastos se podría optar por un modelo de consultoría. En este, se ofrecería soporte personalizado y desarrollo de características personalizadas para cada uno de los clientes que contratase el servicio. Otra posible fuente de ingresos podría ser el \textit{hosting} de bots mediante suscripciones mensuales, ofreciendo la herramienta y los bots como servicio.

\subsection{Sociedad Limitada Nueva Empresa}

Una \textit{SLNE} es una buena opción, ya que permite crear una pequeña empresa con pocos recursos iniciales con la que iniciar el desarrollo de forma profesional el desarrollo. Además tiene bastantes beneficios frente a otros modelos:

\begin{itemize}
	\item Construcción rápida.
	\item No necesita registro de socios.
	\item Fraccionado y aplazamiento de retenciones del \textit{IRPF} y otras deudas y pagos fraccionados.
	\item Se puede cambiar la denominación social de forma gratuita.
\end{itemize}

Los gastos para poder desarrollar un software de las características descritas de forma profesional no son desorbitados, pero tampoco son bajos. En cuanto a gastos derivados de la empresa y burocráticos serían (al menos) los siguientes:

\begin{itemize}
	\item 3000 euros. El capital mínimo a aportar para crear una \textit{SLNE}.
	\item 1000 euros. Estimación de los distintos gastos burocráticos.
	\item 550 euros. Gastos derivados con el desarrollo de la actividad laboral, como por ejemplo un local. Al año supone al menos 6600 euros.
\end{itemize}

Además, hay que tener en cuenta el salario del trabajador, que en este caso sería uno solo, para intentar abaratar costes. Los datos de empleo de 2022 en el sector de la Informática y Telecomunicaciones indican que el salario medio de una persona con aproximadamente 3 años de experiencia laboral (como es mi caso) se sitúa en 36500 euros brutos, lo que se traduce aproximadamente en 3050 euros brutos al mes. De nuevo, a fin de reducir los costes al inicio de la actividad laboral de esta empresa, se podría fijar un salario inferior, 28000 euros brutos al año.

A todas estas cifras habría que sumar todos los gastos relacionados con el desarrollo del software en sí, como pueden ser servicios de alojamiento del código, integración continua, copias de seguridad o sistemas y equipos informáticos. En una etapa inicial se podrían utilizar las versiones gratuitas de algunos estos servicios, pero en ciertos casos no sería posible ya que pueden ser necesarias otras características adicionales.

A continuación se muestra un posible presupuesto del gasto anual teniendo en cuenta todos los aspectos anterior mencionados.

\begin{table}[H]
    \centering
    \def\arraystretch{1.25}
    \begin{adjustbox}{max width=\textwidth}
    \begin{tabularx}{\textwidth}{|L|r|r|r|}
    \hline
        \textbf{Concepto} & \textbf{Euros/Ud} & \textbf{Cantidad} & \textbf{Total (Euros)} \\ \hline
    \hline
        Capital inicial (SLNE) & 3000 & 1 & 3000 \\ \hline
        Burocracia & 1000 & 1 & 1000 \\ \hline
        Derivados & 550 & 12 & 6600 \\ \hline
        Salario & 2333 & 12 & 28000 \\ \hline
        Servicios \textit{Cloud} & 150 & 12 & 1800 \\ \hline
        Sistemas informáticos & 3000 & 1 & 3000 \\ \hline
    \hline
        \multicolumn{3}{|r|}{\textbf{Total}} & \textbf{43400} \\ \hline
    \end{tabularx}
    \end{adjustbox}
    \caption{Presupuesto anual como \textit{SLNE}.}
\end{table}

Se puede observar que el primer año de vida de esta empresa (que hasta el momento sólo tiene un empleado) costaría más de 43000 euros, una cifra bastante alta. En el caso de que se quisiera incluir a un nuevo empleado, también desarrollador con experiencia similar, el coste adicional ascendería a aproximadamente 33000 euros.

\subsection{Autónomo}

Ser autónomo es otra posible opción para comenzar a desarrollar el software profesionalmente. En este caso los costes pueden ser algo inferiores, y además existen deducciones en el caso de ser una primera alta, pero no son bajos.

La siguiente tabla muestra el posible presupuesto del gasto anual:

\begin{table}[H]
    \centering
    \def\arraystretch{1.25}
    \begin{adjustbox}{max width=\textwidth}
    \begin{tabularx}{\textwidth}{|L|r|r|r|}
    \hline
        \textbf{Concepto} & \textbf{Euros/Ud} & \textbf{Cantidad} & \textbf{Total (Euros)} \\ \hline
    \hline
        Cuota mínima & 294 & 12 & 3528 \\ \hline
        Cuota máxima & 711 & 12 & 8532 \\ \hline
    \hline
        Burocracia & 1000 & 1 & 1000 \\ \hline
        Servicios \textit{Cloud} & 150 & 12 & 1800 \\ \hline
        Sistemas informáticos & 3000 & 1 & 3000 \\ \hline
    \hline
        \multicolumn{3}{|r|}{\textbf{Total (cuota mínima)}} & \textbf{9328} \\ \hline
        \multicolumn{3}{|r|}{\textbf{Total (cuota máxima)}} & \textbf{14332} \\ \hline
    \end{tabularx}
    \end{adjustbox}
    \caption{Presupuesto anual como \textit{autónomo}.}
\end{table}

Para este caso se ha mantenido el salario objetivo que se marcaba en la sección anterior, 28000 euros divididos en 12 pagas. Además, entra en juego la cuota de autónomos, que en 2022 sitúa su mínimo en 294 euros. Este aspecto es importante, ya que esta es la aportación por la que se cotiza. Si bien en los primeros meses podría ser interesante reducir al máximo los gastos, no es lo ideal a largo plazo. Otro aspecto importante es que se deberían pagar impuestos trimestrales, como el IVA, lo que incrementa los gastos.

	\chapter{Planificación}

\section{Metodología de desarrollo}
\label{sec:metodologia}

La metodología de desarrollo se puede definir como el proceso disciplinado que busca ser eficiente a la hora de desarrollar un software. A lo largo del tiempo han surgido numerosas metodologías que buscan mejorarse unas a otras, haciendo hincapié en elementos como coste o calidad del desarrollo. De estas, destacan los principios ágiles, los cuales se usan en multitud de entornos laborales hoy en día.

Se ha utilizado \textit{GitHub} para la gestión de un repositorio para el código, además de para utilizar las herramientas que posee que facilitan el desarrollo del software.

Tras analizar el problema se han extraído una serie de casos de uso e historias de usuario, las cuales tienen especial importancia ya que de estas dependen las características del software final. En concreto, una vez establecidas las historias de usuario han surgido una serie de tareas, las cuales se han documentado en \textit{issues} en el \href{https://github.com/harvestcore/matroos}{repositorio}.

En el caso de este proyecto el desarrollo se ha dividido en diferentes hitos, los cuales están compuestos por las anteriores tareas e historias de usuario.

El proceso a seguir para el desarrollo es sencillo. Cuando es preciso trabajar en una tarea, se crea una rama de trabajo (usualmente nombrada con el identificador de la tarea, o un texto relevante). Sobre esta rama se publican una serie de \textit{commits} que solucionan el grueso de la tarea, y posteriormente se crea un \textit{pull request} (o \textit{PR}). Esta acción permite revisar lo que se quiere unir a la rama principal de desarrollo del software, con el fin de detectar errores o iniciar discusiones si fuese necesario.

\section{Temporización}

Como se menciona en la sección de \hyperref[sec:metodologia]{metodología}, el trabajo se divide en \textit{hitos} de duración variable, ya que cada uno puede requerir mayor o menor cantidad de tiempo.

TODOOOOOOOOOOOOOOOOOOOOOOOOOOOOOO
\textbf{Completar con el total de issues resueltas en el proyecto + información del desarrollo.}

\section{PMV y Milestones}

Un producto mínimamente viable, o \textit{PMV}, es un producto con las suficientes características capaz de atraer a los posibles clientes o usuarios tan pronto como sea posible.

Para la realización de este proyecto se ha propuesto la creación de los siguientes \textit{PMV} (o \textit{milestones}, como se llaman en \href{https://github.com/harvestcore/matroos/milestones}{GitHub}).

Los \textit{milestones} 0 a 6 son los principales del proyecto, y son los que se planea inicialmente realizar. Los \textit{milestones} 8 y 9 son adicionales, y completarían el desarrollo de todo el software incluyendo funcionalidad y características extra.

\subsection{Milestones principales}

\subsubsection{00 - Configuración del entorno, tests y CI}

Enlace en \href{https://github.com/harvestcore/matroos/milestone/3}{GitHub}.

\textbf{Versión objetivo: 0.0.1}

El \textit{PMV} incluirá:

\begin{itemize}
	\item La estructura del repositorio está definida e implementada.
	\item Los proyectos necesarios están creados y listos para continuar con el desarrollo de nuevas funcionalidades.
	\item \textit{CI} está listo para ejecutar los diferentes tests y pruebas implementadas en los distintos proyectos.
	\item La documentación hasta este punto del desarrollo está actualizada y disponible.
\end{itemize}

Decisiones técnicas y documentación adicional:

\begin{itemize}
	\item Lenguaje de programación y \textit{framework}.
	\item Integración continua.
	\item Arquitectura.
\end{itemize}

\subsubsection{01 - Modelado del dominio del problema y lógica de negocio}

Enlace en \href{https://github.com/harvestcore/matroos/milestone/12}{GitHub}.

\textbf{Versión objetivo: 0.0.2}

El \textit{PMV} incluirá:

\begin{itemize}
	\item El dominio del problema está modelado.
	\item La lógica de negocio en su forma más básica está definida.
	\item La documentación hasta este punto del desarrollo está actualizada y disponible.
\end{itemize}

Decisiones técnicas y documentación adicional:

\begin{itemize}
	\item Arquitectura.
	\item Comandos.
	\item Bots.
	\item Herramientas.
\end{itemize}

\subsubsection{02 - Gestión de comandos}

Enlace en \href{https://github.com/harvestcore/matroos/milestone/10}{GitHub}.

\textbf{Versión objetivo: 0.0.3}

Tras la finalización de este \textit{milestone}:

\begin{itemize}
	\item La estructura de los comandos está definida.
	\item El \textit{backend} cuenta con un servicio capaz de gestionar los comandos y su configuración.
	\item Los tipos de comandos básicos quedan definidos y se pueden crear nuevos comandos de estos tipos.
	\item La documentación hasta este punto del desarrollo está actualizada y disponible.
\end{itemize}

Decisiones técnicas y documentación adicional:

\begin{itemize}
	\item Arquitectura.
	\item Comandos.
\end{itemize}

\subsubsection{03 - Gestión de bots}

Enlace en \href{https://github.com/harvestcore/matroos/milestone/6}{GitHub}.

\textbf{Versión objetivo: 0.0.4}


El \textit{PMV} incluirá:

\begin{itemize}
	\item La estructura de los bots está definida.
	\item El \textit{backend} cuenta con un servicio capaz de gestionar los bots y su configuración.
	\item Es posible asociar comandos a bots.
	\item La documentación hasta este punto del desarrollo está actualizada y disponible.
\end{itemize}

Decisiones técnicas y documentación adicional:

\begin{itemize}
	\item Arquitectura.
	\item Bots.
\end{itemize}

\subsubsection{04 - Despliegue de bots en \textit{workers}}

Enlace en \href{https://github.com/harvestcore/matroos/milestone/5}{GitHub}.

\textbf{Versión objetivo: 0.0.5}


El \textit{PMV} incluirá:

\begin{itemize}
	\item Es posible desplegar (ejecutar) bots en los \textit{workers}.
	\item El \textit{backend} es capaz de comunicarse con los distintos \textit{workers}.
	\item La documentación hasta este punto del desarrollo está actualizada y disponible.
\end{itemize}

Decisiones técnicas y documentación adicional:

\begin{itemize}
	\item Lenguaje de programación y \textit{framework}.
	\item Arquitectura.
\end{itemize}

\subsubsection{05 - API REST}

Enlace en \href{https://github.com/harvestcore/matroos/milestone/7}{GitHub}.

\textbf{Versión objetivo: 0.1.0}


El \textit{PMV} incluirá:

\begin{itemize}
	\item La \textit{API Rest} está definida y los \textit{endpoints} están documentados.
	\item Es posible realizar las tareas de administración de bots y comandos haciendo uso de la API.
	\item La documentación hasta este punto del desarrollo está actualizada y disponible.
\end{itemize}

Decisiones técnicas y documentación adicional:

\begin{itemize}
	\item Lenguaje de programación y \textit{framework}.
	\item Arquitectura.
\end{itemize}

\subsubsection{06 - Despliegue en contenedores Docker}

Enlace en \href{https://github.com/harvestcore/matroos/milestone/2}{GitHub}.

\textbf{Versión objetivo: 0.2.0}


El \textit{PMV} incluirá:

\begin{itemize}
	\item El software es distribuible mediante contenedores Docker.
	\item El archivo \textit{Dockerfile} para el microservicio del \textit{backend} está disponible.
	\item El archivo \textit{Dockerfile} para el microservicio del \textit{worker} está disponible.
	\item El archivo \textit{Docker Compose} para orquestar los microservicios está disponible.
	\item La documentación hasta este punto del desarrollo está actualizada y disponible.
\end{itemize}

Decisiones técnicas y documentación adicional:

\begin{itemize}
	\item Despliegue en contenedores.
	\item Arquitectura.
\end{itemize}

\subsubsection{07 - Almacén de datos}

Enlace en \href{https://github.com/harvestcore/matroos/milestone/11}{GitHub}.

\textbf{Versión objetivo: 0.3.0}


El \textit{PMV} incluirá:

\begin{itemize}
	\item Tanto los bots como los comandos son almacenables en base de datos.
	\item La documentación hasta este punto del desarrollo está actualizada y disponible.
\end{itemize}

Decisiones técnicas y documentación adicional:

\begin{itemize}
	\item Herramientas (Base de datos).
	\item Arquitectura.
	\item Base de datos.
\end{itemize}

\subsection{Milestones adicionales}

\subsubsection{08 - Interfaz de usuario}

Enlace en \href{https://github.com/harvestcore/matroos/milestone/9}{GitHub}.

\textbf{Versión objetivo: 0.4.0}


El \textit{PMV} incluirá:

\begin{itemize}
	\item La interfaz de usuario está disponible y es capaz de realizar las tareas de creación y configuración de comandos y bots, además del despliegue de éstos en \textit{workers}.
	\item La interfaz de usuario es distribuible mediante contenedores \textit{Docker}.
	\item El archivo \textit{Dockerfile} para el microservicio está disponible.
	\item La documentación hasta este punto del desarrollo está actualizada y disponible.
\end{itemize}

Decisiones técnicas y documentación adicional:

\begin{itemize}
	\item \textit{Frontend}.
	\item Arquitectura.
\end{itemize}

	\chapter{Herramientas y tecnologías utilizadas}
 

	\input{secciones/06_implementacion}
	\input{secciones/07_conclusiones}
	
	\newpage
	\bibliography{biblio}
	\bibliographystyle{plain}
	
\end{document}

