% \documentclass[paper=a4, fontsize=11pt]{scrartcl} % A4 paper and 11pt font size
\documentclass[11pt, a4paper]{book}
\usepackage[T1]{fontenc} % Use 8-bit encoding that has 256 glyphs
\usepackage[utf8]{inputenc}
\usepackage{fourier} % Use the Adobe Utopia font for the document - comment this line to return to the LaTeX default
\usepackage{listings} % para insertar código con formato similar al editor
\usepackage[spanish, es-tabla]{babel} % Selecciona el español para palabras introducidas automáticamente, p.ej. "septiembre" en la fecha y especifica que se use la palabra Tabla en vez de Cuadro
\usepackage{url} % ,href} %para incluir URLs e hipervínculos dentro del texto (aunque hay que instalar href)
\usepackage{graphics,graphicx, float} %para incluir imágenes y colocarlas
\usepackage[gen]{eurosym} %para incluir el símbolo del euro
\usepackage{cite} %para incluir citas del archivo <nombre>.bib
\usepackage{enumerate}
\usepackage{hyperref}
\usepackage{graphicx}
\usepackage{tabularx}
\usepackage{booktabs}

\usepackage[table,xcdraw]{xcolor}
\hypersetup{
	colorlinks=true,	% false: boxed links; true: colored links
	linkcolor=black,	% color of internal links
	urlcolor=cyan		% color of external links
}

\renewcommand{\familydefault}{\sfdefault}
\usepackage{fancyhdr} % Custom headers and footers
\pagestyle{fancyplain} % Makes all pages in the document conform to the custom headers and footers
\fancyhead[L]{} % Empty left header
\fancyhead[C]{} % Empty center header
\fancyhead[R]{Me Llamo Así} % My name
\fancyfoot[L]{} % Empty left footer
\fancyfoot[C]{} % Empty center footer
\fancyfoot[R]{\thepage} % Page numbering for right footer
%\renewcommand{\headrulewidth}{0pt} % Remove header underlines
\renewcommand{\footrulewidth}{0pt} % Remove footer underlines
\setlength{\headheight}{13.6pt} % Customize the height of the header

\usepackage{titlesec, blindtext, color}
\definecolor{gray75}{gray}{0.75}
\newcommand{\hsp}{\hspace{20pt}}
\titleformat{\chapter}[hang]{\Huge\bfseries}{\thechapter\hsp\textcolor{gray75}{|}\hsp}{0pt}{\Huge\bfseries}
\setcounter{secnumdepth}{4}
\usepackage[Lenny]{fncychap}


\begin{document}
	\begin{titlepage}
\newlength{\centeroffset}
\setlength{\centeroffset}{-0.5\oddsidemargin}
\addtolength{\centeroffset}{0.5\evensidemargin}
\thispagestyle{empty}

\noindent\hspace*{\centeroffset}\begin{minipage}{\textwidth}

\centering
\includegraphics[width=0.9\textwidth]{logos/logo_ugr.jpg}\\[1.4cm]

\textsc{ \Large TRABAJO FIN DE MÁSTER\\[0.2cm]}
\textsc{ GRADO EN INGENIERÍA INFORMÁTICA}\\[1cm]

{\Huge\bfseries Matroos \\}
\noindent\rule[-1ex]{\textwidth}{3pt}\\[3.5ex]
{\large\bfseries Creación, configuración y despliegue de bots en \textit{Discord} }
\end{minipage}

\vspace{2.5cm}
\noindent\hspace*{\centeroffset}
\begin{minipage}{\textwidth}
\centering

\textbf{Autor}\\ {Ángel Gómez Martín}\\[2.5ex]
\textbf{Director}\\ {Juan Julián Merelo Guervós}\\[2cm]
\includegraphics[width=0.3\textwidth]{logos/etsiit_logo.png}\\[0.1cm]
\textsc{Escuela Técnica Superior de Ingenierías Informática y de Telecomunicación}\\
\textsc{---}\\
Granada, 7 de Julio de 2022
\end{minipage}
\end{titlepage}

	\thispagestyle{empty}

\begin{center}
{\large\bfseries Matroos \\ Creación, configuración y despliegue de bots en \textit{Discord}. }\\
\end{center}
\begin{center}
Ángel Gómez Martín
\end{center}


\vspace{0.5cm}
\noindent{\textbf{Palabras clave}: software libre, \textit{Discord}, bot, \textit{API REST}, despliegue, \textit{backend}, \textit{frontend}, \textit{worker}
\vspace{0.7cm}

\noindent{\textbf{Resumen}\\

La creación y uso de bots se ha popularizado mucho en los últimos años, siendo extraño no encontrarlos integrados en multitud de sistemas. De entre estos destaca \textit{Discord}, una plataforma de mensajería instantánea utilizada principalmente por jóvenes que también ha ganado gran relevancia recientemente y donde los bots son ampliamente usados. Aunque los bots puedan parecer algo sencillo, los procesos que conllevan crearlos, configurarlos y desplegarlos pueden ser bastante complejos.

No obstante estos se pueden simplificar y unificar. En este proyecto se ha desarrollado una solución que aúna todos esos procesos en un puesto centralizado compuesto por un \textit{backend}, un \textit{frontend} y una serie de servicios llamados \textit{workers}, siendo un conjunto que permite facilitar y agilizar el desarrollo de estas tareas. Además ofrece modularidad de sus características, pudiendo adecuar las funcionalidades de cada bot adecuándose a las necesidades requeridas en cada situación. También ofrece una \textit{API REST}, la cual permite la comunicación con otro tipo de aplicaciones.
	

\cleardoublepage

\begin{center}
{\large\bfseries Matroos \\ Creation, configuration and deployment of bots in \textit{Discord}. }\\
\end{center}
\begin{center}
	Ángel Gómez Martín
\end{center}
\vspace{0.5cm}
\noindent{\textbf{Keywords}: \textit{open source}, \textit{Discord}, bot, \textit{REST API}, \textit{deployment}, \textit{backend}, \textit{frontend}, \textit{worker}
\vspace{0.7cm}

\noindent{\textbf{Abstract}\\

The creation and use of bots has become very popular in recent years, and it is strange not to find them integrated into a multitude of systems. Among these, \textit{Discord}, an instant messaging platform used mainly by young people, has also gained great relevance recently and where bots are widely used. While bots may seem straightforward, the processes involved in creating, configuring and deploying them can be quite complex.

However, these can be simplified and unified. In this project, a solution has been developed that brings together all these processes in a centralised position composed of a backend, a frontend and a series of services called ``workers''; a set that facilitates and speeds up the development of these tasks. It also offers modularity of its characteristics, being able to adapt the functionalities of each bot to suit the needs required in each situation. It also offers a REST API, which allows communication with other types of applications.

\cleardoublepage

\thispagestyle{empty}

\noindent\rule[-1ex]{\textwidth}{2pt}\\[4.5ex]

D. \textbf{Juan Julián Merelo Guervós}, Profesor del Departamento de Arquitectura y Tecnología de Computadores de la Universidad de Granada.

\vspace{0.5cm}

\textbf{Informa:}

\vspace{0.5cm}

Que el presente trabajo, titulado \textit{\textbf{Matroos}}, ha sido realizado bajo mi supervisión por \textbf{Ángel Gómez Martín}, y autorizo la defensa de dicho trabajo ante el tribunal que corresponda.

\vspace{0.5cm}

Y para que conste, expiden y firman el presente informe en Granada a 7 de Julio de 2022.

\vspace{1cm}

\textbf{El director:}

\vspace{5cm}

\noindent Fdo: Juan Julián Merelo Guervós



\chapter*{Agradecimientos}

A mi tutor, JJ, por ofrecerme su ayuda, conocimientos y acertados comentarios para la realización de este proyecto.

A mis padres, Elia y Ángel, por ser un pilar fundamental y por empujarme a seguir aprendiendo cosas nuevas y a superarme cada día.

A mi hermana Cristina, por apoyarme y echarme una mano siempre que lo he necesitado.

Y a Paula, por aguantarme más que nadie todo este tiempo.


	\newpage
	\tableofcontents

	\newpage
	\listoffigures

	\listoftables 
	\newpage

	\chapter{Introducción}

Los bots permiten automatizar tareas en sistemas conversacionales. Sin embargo, su creación en muchas ocasiones es compleja y no está al alcance de todos los usuarios. Existen tareas que son sencillas de realizar de manera manual, pero que pueden requerir un tiempo y dedicación considerable, y sería mucho más cómodo automatizarlas de alguna manera. Por otra parte, cuando hay un gran volumen de datos sobre el que realizar estas tareas, automatizarlas otorgaría grandes beneficios ya que no dependerían de la disponibilidad de una persona para realizarlas. Además, se eliminaría la necesidad de realizar en un gran número de ocasiones un proceso que podría ser corto pero muy repetitivo. Estos procesos pueden ser de muchos tipos y sobre muchos ámbitos diferentes dependiendo del entorno en que sean necesarios, y como al fin y al cabo son procesos específicos, el sistema de automatización también debería ser así, pero desarrollarlo requiere de una experiencia técnica determinada. Esto no esta al alcance de todos.

En concreto, en la plataforma \textit{Discord}\cite{discord}, que se ha hecho popular recientemente, este problema no es una excepción. En \textit{Discord} existen comunidades construidas en torno a multitud de ámbitos, como videojuegos, deportes, cine o incluso en el mundo profesional, donde la plataforma se ha hecho un hueco. Estas comunidades pueden crecer mucho, y del mismo modo crecen las necesidades de moderación y funcionalidades adicionales que pueden ser útiles para facilitar ciertas tareas en función del ámbito de la comunidad. Un ejemplo de esto podría ser la automatización de mensajes para anunciar los estrenos de cartelera semanal de un cine en concreto o la monitorización de sistemas en una una pequeña empresa.

Resolver este problema puede ayudar a todos aquellos usuarios y comunidades que buscan automatizar tareas específicas de manera individual sin tener que recurrir a terceros con los conocimientos específicos y con los que quizás se tendría que compartir información sensible. Por otro lado, tener que recurrir a conocimiento experto externo conlleva una serie de costes que en algunos casos puede ser inadmisible. Este problema suele ser general para todas las herramientas que permiten la integración con bots, pero dado que en cada plataforma los bots se crean con tecnologías y procedimientos diferentes, este proyecto se centra en Discord pues es la plataforma que probablemente podría beneficiarse más de esta nueva herramienta dada la gran cantidad de usuarios que hace uso de ella.

\section{Motivación}

La principal motivación para realizar este proyecto es que crear bots para \textit{Discord} con funcionalidades muy concretas con las herramientas actuales es una tarea compleja. Si bien existen librerías para lenguajes de programación, las herramientas que han surgido para crear estos bots de forma sencilla se centran en aspectos muy básicos que no tienen mucha cabida en ámbitos más concretos.

No existen \textit{frameworks} o sistemas que permitan crear fácilmente comandos que se ajusten a necesidades específicas, y menos que sean reproducibles o configurables de alguna manera con distintos datos. Además, cuando se trata de crear y alojar distintos bots en un mismo sistema, las actuales soluciones requieren que cada uno de estos bots sea una instancia independiente. Esto imposibilita la configuración de todos estos bots de una manera sencilla dentro de un mismo sistema. De nuevo, algunos de estos conceptos requieren de conocimientos experto extra necesarios para crear estas herramientas no están al alcance de todos.

\section{Problema y solución}

Como se indicaba en la sección anterior, usuarios con necesidades específicas (como por ejemplo podría ser un administrador de sistemas de una pequeña empresa, o incluso un usuario entusiasta al que le gusta automatizar tareas en su comunidad de \textit{Discord}) que quieren hacer uso de este software y de su sistema de bots se ven obligados a crear distintos bots muy específicos y en ocasiones poco reutilizables. Si bien se podrían programar comandos más concretos en un único bot, sería una tarea tediosa la reutilización entre diferentes ámbitos. Por ejemplo, dos comunidades de juegos de mesa, donde se comparten muchas funcionalidades, pero dependiendo del juego los detalles son distintos.

Incidiendo en el aspecto de la creación de bots forma más técnica, se observa que actualmente hay tres maneras de usar bots en \textit{Discord}:

\begin{itemize}
	\item \textbf{Usar bots ya existentes}. El bot ya se encuentra creado, configurado y desplegado, y solo es necesario agregarlo al servidor para poder disfrutar de sus funcionalidades.
	\item \textbf{Crear un bot a alto nivel}. Este caso es similar al anterior, ya que se trata de un bot genérico, que es configurable en cierta medida para cada servidor. Esta configuración se hace a través de algún tipo de herramienta (generalmente una aplicación web) que permite configurar los comandos deseados. Por otro lado, ya que están pensados para un público general, las posibilidades de configuración son escasas. Suelen tener algunas plantillas de comandos básicos, como temporizadores o respuestas automáticas.
	\item \textbf{Crear un bot a bajo nivel}. En este caso el bot se crea haciendo uso de las diferentes \textit{API} que ofrece \textit{Discord} para ello y la personalización es máxima. En cambio, es más tedioso, y requiere conocimientos extra que algunos usuarios pueden no tener (como programación). Además hay que tener en cuenta que el bot debe ser desplegado manualmente, por lo que requiere un esfuerzo extra.
\end{itemize}

Como se observa, en el primer caso la capacidad de configuración del bot es nula y en el segundo las funcionalidades son muy reducidas. En el último caso, aunque permite realizar cualquier configuración, se pierde el aspecto de la administración central, ya que se crean bots independientes. 

Por tanto, se pretende desarrollar un \textit{framework} para crear bots de \textit{Discord} completamente configurables, y que además permita la creación sencilla de comandos. También, se busca también la centralización de procesos.

	\input{secciones/02_descripcion}
	\chapter{Estado del arte}

En este capítulo se hace un repaso de las distintas soluciones actuales que existen para la creación de bots de \textit{Discord}. Cada una de ellas destaca en aspectos concretos, pero ninguna reúne todas las características deseadas.

\section{Soluciones actuales}

En el ámbito de la creación de bots de \textit{Discord} existen una generosa cantidad de herramientas con este cometido. Como se comentaba en el capítulo anterior, estas herramientas en su basta mayoría son bastante básicas, permitiendo una creación y configuración de bots que en ciertos ámbitos puede ser insuficiente. Aún así, existen otras alternativas que son más interesantes y completas, pero que son complejas de utilizar.

Estos sistemas se podrían dividir en dos grupos, herramientas \textit{no-code} y herramientas que hacen uso de programación. En ambas la interacción con el sistema se hace a través de una aplicación web, y además suelen tener una apariencia muy similar. En las siguientes secciones se incluyen las herramientas con características más interesantes y las que dentro de lo que cabe son más completas.

\subsection{Herramientas \textit{no-code}}

Estas herramientas permiten la creación de bots sin hacer uso de recursos de programación o similares. En la mayoría de casos estas plataformas cuentan con un único bot que se debe agregar al servidor de \textit{Discord} deseado, y permiten la configuración de este bot de forma individual para cada servidor. En general la mayoría de herramientas de este tipo tienen una serie de funcionalidades gratuitas, teniendo que adquirir un plan de pago mensual para obtener funcionalidades extra.

En ellas se puede observar también uno de los principales problemas mencionados en el capítulo anterior, la muy reducida personalización y reutilización de los comandos. En estos sistemas no se puede crear un comando específico con una funcionalidad concreta, sino que se basan en funcionalidades predefinidas no reutilizables.

\subsubsection{ProBot}
\href{https://probot.io/}{\textit{ProBot}} es sin duda la más interesante de las herramientas no-code debido a que permite crear ilimitados comandos personalizados, siendo la principal desventaja que estos comandos son predefinidos, y no se puede cambiar su funcionalidad. Los comandos predefinidos se centran en moderación y mensajes automáticos, por lo que las posibilidades no son muy amplias.

A favor de esta herramienta también destaca que es sencilla de utilizar, la interfaz web intenta imitar a la de \textit{Discord} y es intuitiva. Por contra, es bastante intrusiva la modalidad \textit{premium}, ya que muchas secciones sugieren la compra de esta modalidad. Además es imposible controlar el despliegue del bot, y no es posible reutilizar comandos.

Sus características son:

\begin{table}[H]
    \centering
    \def\arraystretch{1.25}
    \begin{adjustbox}{max width=\textwidth}
    \begin{tabularx}{325px}{|l|L|}
    \hline
        \multicolumn{2}{|c|}{\textbf{\textit{ProBot}}} \\ \hline
    \hline
        \textbf{Tipo de comandos} & Predefinidos (ilimitados) \\ \hline
        \textbf{Comandos reutilizables} & No \\ \hline
        \textbf{Control del despliegue} & No \\ \hline
        \textbf{Número de bots} & 1, único \\ \hline
        \textbf{Experiencia} & Sencilla \\ \hline
        \textbf{Personalización extra} & Requiere \textit{premium} (mensualidades) \\ \hline
        \textbf{Características} & · Moderación\linebreak · Estadísticas\linebreak · Mensajes automáticos\linebreak · Música \\ \hline
        \textbf{Logs} & No \\ \hline
        \textbf{Premium} & \$60 al año \\ \hline
    \end{tabularx}
    \end{adjustbox}
    \caption{Características de \textit{ProBot}.}
\end{table}

\begin{figure}[H]
	\centering
	\includegraphics[width=1\textwidth]{img/probot.png}
	\caption{Interfaz web de \textit{ProBot}.}
\end{figure}

\subsubsection{Mee6}

\href{https://mee6.xyz/}{\textit{Mee6}} es otra herramienta muy similar a la anterior, siendo la principal diferencia que en este caso los comandos personalizados se limitan a 5. Por contra, tiene un mayor catálogo de funcionalidades.

De nuevo no es posible controlar el despliegue del bot, como tampoco es posible crear otro bot y agregarlo a un mismo servidor, o reutilizar comandos.

Sus características son:

\begin{table}[H]
    \centering
    \def\arraystretch{1.25}
    \begin{adjustbox}{max width=\textwidth}
    \begin{tabularx}{325px}{|l|L|}
    \hline
        \multicolumn{2}{|c|}{\textbf{\textit{Mee6}}} \\ \hline
    \hline
        \textbf{Tipo de comandos} & Predefinidos (muy limitados, 5) \\ \hline
        \textbf{Comandos reutilizables} & No \\ \hline
        \textbf{Control del despliegue} & No \\ \hline
        \textbf{Número de bots} & 1, único \\ \hline
        \textbf{Experiencia} & Sencilla \\ \hline
        \textbf{Personalización extra} & Requiere \textit{premium} (mensualidades) \\ \hline
        \textbf{Características} & · Moderación\linebreak · Estadísticas\linebreak · Mensajes automáticos\linebreak · Música\linebreak · Temporizadores\linebreak · \textit{Quiz} / \textit{Trivia} \\ \hline
        \textbf{Logs} & No \\ \hline
        \textbf{Premium} & \$50 al año / \$90 de por vida  \\ \hline
    \end{tabularx}
    \end{adjustbox}
    \caption{Características de \textit{Mee6}.}
\end{table}

\begin{figure}[H]
	\centering
	\includegraphics[width=1\textwidth]{img/mee6.png}
	\caption{Interfaz web de \textit{Mee6}.}
\end{figure}


\subsubsection{BotGhost}

\href{https://botghost.com/}{\textit{BotGhost}} es un híbrido entre \textit{ProBot} y \textit{Mee6}, ya que tiene las características comunes de ambos. La principal característica de esta herramienta es que permite crear comandos personalizados haciendo uso de una serie de módulos que se pueden interconectar para definir el ciclo de vida de un comando.

Esta característica es muy interesante, pero está muy limitada y las funcionalidades que permite realizar se resumen en envío de mensajes y tareas de moderación de usuarios muy básicas. El plan \textit{premium} sería necesario en este caso para poder sacarle partido a esta funcionalidad.

Otro aspecto interesante es que se pueden crear distintos bots, hasta 50 distintos si se opta por la opción \textit{premium}.

Sus características son:

\begin{table}[H]
    \centering
    \def\arraystretch{1.25}
    \begin{adjustbox}{max width=\textwidth}
    \begin{tabularx}{325px}{|l|L|}
    \hline
        \multicolumn{2}{|c|}{\textbf{\textit{BotGhost}}} \\ \hline
    \hline
        \textbf{Tipo de comandos} & Predefinidos (muy limitados, 5) \\ \hline
        \textbf{Comandos reutilizables} & Sí \\ \hline
        \textbf{Control del despliegue} & No (Sólo encendido y apagado) \\ \hline
        \textbf{Número de bots} & 1, único (50 con \textit{premium}) \\ \hline
        \textbf{Experiencia} & Compleja \\ \hline
        \textbf{Personalización extra} & Requiere \textit{premium} (mensualidades) \\ \hline
        \textbf{Características} & · Moderación\linebreak · Estadísticas\linebreak · Mensajes automáticos\linebreak · Temporizadores\linebreak · Integración con videojuegos\linebreak · Meteorología\linebreak · Música\linebreak · \textit{Quiz} / \textit{Trivia} \\ \hline
        \textbf{Logs} & No \\ \hline
        \textbf{Premium} & \$60 al año / \$100 de por vida \\ \hline
    \end{tabularx}
    \end{adjustbox}
    \caption{Características de \textit{BotGhost}.}
\end{table}

\begin{figure}[H]
	\centering
	\includegraphics[width=1\textwidth]{img/botghost.png}
	\caption{Interfaz web de \textit{BotGhost}.}
\end{figure}

\subsection{Herramientas de programación}

Actualmente existen multitud de librerías para distintos lenguajes de programación que permiten interactuar con la \textit{API} de \textit{Discord} y por tanto crear un bot. Así mismo existen herramientas híbridas que permiten esta creación de una manera más sencilla.

\subsubsection{Autocode}

\href{https://autocode.com/}{Autocode} es sin duda la herramienta mas interesante que se ha encontrado de esta modalidad híbrida. Realmente es una plataforma que permite la creación y despliegue de aplicaciones y servicios web, bots, y tareas de automatización haciendo uso de \textit{JavaScript} inyectado por los usuarios.

De este modo el usuario sólo tiene que preocuparse por el código de la aplicación (o bot en este caso) que quiere crear, ya que del despliegue se encarga \textit{Autocode}. En su plan gratuito se pueden crear hasta 50 aplicaciones distintas, y permite la integración entre si de los distintos recursos que el usuario crea en la plataforma.

Sus características son:

\begin{table}[H]
    \centering
    \def\arraystretch{1.25}
    \begin{adjustbox}{max width=\textwidth}
    \begin{tabularx}{325px}{|l|L|}
    \hline
        \multicolumn{2}{|c|}{\textbf{\textit{Autocode}}} \\ \hline
    \hline
        \textbf{Tipo de comandos} & Predefinidos + \textit{JS} \\ \hline
        \textbf{Comandos reutilizables} & No \\ \hline
        \textbf{Control del despliegue} & Sí (limitado) \\ \hline
        \textbf{Número de bots} & 50 gratis \\ \hline
        \textbf{Experiencia} & Algo complejo \\ \hline
        \textbf{Personalización extra} & Requiere \textit{premium} (mensualidades) \\ \hline
        \textbf{Especialidad} & Despliegue general de aplicaciones \\ \hline
        \textbf{Logs} & Sí (1-30 días) \\ \hline
        \textbf{Premium} & \$180 / \$1620 al año \\ \hline
    \end{tabularx}
    \end{adjustbox}
    \caption{Resumen de soluciones actuales.}
\end{table}

\begin{figure}[H]
	\centering
	\includegraphics[width=1\textwidth]{img/autocode.png}
	\caption{Interfaz web de \textit{Autocode}.}
\end{figure}

\subsubsection{Librerías de programación}

Las librerías de programación dan libertad a la hora de crear un bot de \textit{Discord}, lo cual puede ser ideal en algunos casos. Las ventajas son obvias, ya que se puede crear cualquier tipo de comando y la reutilización es sencilla, pero en cambio, la gestión del despliegue puede ser compleja.

Por lo general todas las librerías permiten realizar casi las mismas funcionalidades, diferenciándose en aspectos como el rendimiento, la comunidad que las soporta o la facilidad de uso.

Algunos ejemplos de librerías son:

\begin{itemize}
	\item \textbf{\textit{C\#}}: \href{https://discordnet.dev/}{\textit{Discord.NET}}, \href{https://github.com/DSharpPlus/DSharpPlus}{\textit{DSharpPlus}}
	\item \textbf{\textit{Java}}: \href{https://github.com/DV8FromTheWorld/JDA}{\textit{JDA}}, \href{https://discord4j.com/}{\textit{Discord4J}}
	\item \textbf{\textit{C++}}: \href{https://dpp.dev/}{\textit{D++}}
	\item \textbf{\textit{JavaScript}}: \href{https://discord.js.org/}{\textit{discord.js}}
	\item \textbf{\textit{Golang}}: \href{https://github.com/bwmarrin/discordgo}{\textit{DiscordGo}}
	\item \textbf{\textit{Ruby}}: \href{https://github.com/shardlab/discordrb}{\textit{discordrb}}
\end{itemize}


\subsection{Comparativa de tiempos}

En esta sección se hace una comparativa del tiempo medio de desarrollo desde cero de un bot de \textit{Discord} usando las herramientas anterior mencionadas. Además se incluye tiempos de desarrollo usando tres lenguajes de programación: \textit{C\#}, \textit{JavaScript} y \textit{Python}.

Las mediciones incluyen todos los pasos necesarios para crear uno de estos bots con dos comandos personalizados. En el caso de las herramientas de programación se incluye desde la creación del proyecto hasta el despliegue (en local) de este.

Los dos comandos personalizados se han elegido al ser comunes en todas las plataformas mencionadas, además de sencillos de implementar. Son los siguientes:

\begin{itemize}
	\item Envío de un mensaje.
	\item Envío de un mensaje recurrente.
\end{itemize}

\begin{table}[H]
    \centering
    \def\arraystretch{1.25}
    \begin{adjustbox}{max width=\textwidth}
    \begin{tabularx}{200px}{|l|R|}
    \hline
        \textbf{Herramienta} & \textbf{Tiempo (en minutos)} \\ \hline
    \hline
        ProBot & 5 \\ \hline
        Mee6 & 5 \\ \hline
        BotGhost & 10 \\ \hline
    \hline
        Autocode (JS) & 45 \\ \hline
    \hline
        JS & 85 \\ \hline
        C\# & 100 \\ \hline
        Python & 80 \\ \hline
    \end{tabularx}
    \end{adjustbox}
    \caption{Comparativa de tiempos}
\end{table}

\section{Discusión}

Como se puede observar existen multitud de posibilidades a la hora de crear un bot de \textit{Discord}, y aunque cumplen lo que prometen, se centran en aspectos muy concretos dejando otros bastante desatendidos.

En las herramientas \textit{no-code} los bots se centran principalmente en tareas de moderación, envío de mensajes, estadísticas e integración con videojuegos y redes sociales. Se enfocan también en los paquetes \textit{premium}, dejando de lado detalles específicos (y que serían ideales) como:

\begin{itemize}
	\item Reutilización de comandos.
	\item No es posible crear comandos con funcionalidad específica.
	\item No se tiene control del despliegue de los bots.
	\item No se pueden crear distintos bots.
\end{itemize}

En el caso de las librerías de programación, aunque todas permiten el acceso a la \textit{API} de \textit{Discord}, cada una de ellas tiene una estructura distinta y los procedimientos para crear un bot o comandos son más o menos complejos. Si un usuario decidiese utilizarlas tendría flexibilidad completa a la hora de crear una estructura concreta, pero entonces tendría que dedicar en ese caso un tiempo necesario para diseñar algo funcional.

\textit{Autocode} es una buena alternativa a las soluciones anteriores, ya que se evita el tener que gestionar el despliegue de los bots y se eliminan algunas trabas de gestión del código, pero al igual que el uso de librerías toda la lógica recae en el usuario final. Esto puede ser útil en ciertos casos, pero no siempre.

En la comparativa de tiempos anterior se puede observar que las herramientas \textit{no-code} son las más rápidas. Esto se debe a que solo es necesario agregar el bot al servidor de \textit{Discord} deseado, y tras eso configurar de manera sencilla los comandos.

En menos de 10 minutos se puede incluir una gran cantidad de funcionalidad a un servidor de manera gratuita, algo que puede ser muy útil para la basta mayoría de usuarios de \textit{Discord}, pero cuando se necesitan funcionalidades específicas entonces no son las ideales.

En cambio, cuando se usan herramientas que hacen uso de código, el tiempo de implementación se incrementa considerablemente. Debido a las características de lenguajes como \textit{JavaScript} o \textit{Python} este tiempo es algo menor que en \textit{C\#}, ya que este suele tener una curva de aprendizaje mayor. Aún así, teniendo en cuenta que hay que hacer ciertas consultas a la documentación de estas librerías y que hay que crear una estructura y una serie de recursos para poder comenzar a desarrollar el bot, el tiempo se incrementa.

En definitiva, no existe ninguna herramienta que brinde lo mejor de ambas alternativas. Por un lado se quiere facilitar la creación de bots y comandos, y el despliegue de estos. Por otro se quiere poder ampliar el repertorio de comandos disponible de manera sencilla, sin tener que desarrollar una aplicación completa para ello.

	\chapter{Planificación}

\section{Metodología de desarrollo}
\label{sec:metodologia}

La metodología de desarrollo se puede definir como el proceso disciplinado que busca ser eficiente a la hora de desarrollar un software. A lo largo del tiempo han surgido numerosas metodologías que buscan mejorarse unas a otras, haciendo hincapié en elementos como coste o calidad del desarrollo. De estas, destacan los principios ágiles, los cuales se usan en multitud de entornos laborales hoy en día.

Se ha utilizado \textit{GitHub} para la gestión de un repositorio para el código, además de para utilizar las herramientas que posee que facilitan el desarrollo del software.

Tras analizar el problema se han extraído una serie de casos de uso e historias de usuario, las cuales tienen especial importancia ya que de estas dependen las características del software final. En concreto, una vez establecidas las historias de usuario han surgido una serie de tareas, las cuales se han documentado en \textit{issues} en el \href{https://github.com/harvestcore/matroos}{repositorio}.

En el caso de este proyecto el desarrollo se ha dividido en diferentes hitos, los cuales están compuestos por las anteriores tareas e historias de usuario.

El proceso a seguir para el desarrollo es sencillo. Cuando es preciso trabajar en una tarea, se crea una rama de trabajo (usualmente nombrada con el identificador de la tarea, o un texto relevante). Sobre esta rama se publican una serie de \textit{commits} que solucionan el grueso de la tarea, y posteriormente se crea un \textit{pull request} (o \textit{PR}). Esta acción permite revisar lo que se quiere unir a la rama principal de desarrollo del software, con el fin de detectar errores o iniciar discusiones si fuese necesario.

\section{Temporización}

Como se menciona en la sección de \hyperref[sec:metodologia]{metodología}, el trabajo se divide en \textit{hitos} de duración variable, ya que cada uno puede requerir mayor o menor cantidad de tiempo.

TODOOOOOOOOOOOOOOOOOOOOOOOOOOOOOO
\textbf{Completar con el total de issues resueltas en el proyecto + información del desarrollo.}

\section{PMV y Milestones}

Un producto mínimamente viable, o \textit{PMV}, es un producto con las suficientes características capaz de atraer a los posibles clientes o usuarios tan pronto como sea posible.

Para la realización de este proyecto se ha propuesto la creación de los siguientes \textit{PMV} (o \textit{milestones}, como se llaman en \href{https://github.com/harvestcore/matroos/milestones}{GitHub}).

Los \textit{milestones} 0 a 6 son los principales del proyecto, y son los que se planea inicialmente realizar. Los \textit{milestones} 8 y 9 son adicionales, y completarían el desarrollo de todo el software incluyendo funcionalidad y características extra.

\subsection{Milestones principales}

\subsubsection{00 - Configuración del entorno, tests y CI}

Enlace en \href{https://github.com/harvestcore/matroos/milestone/3}{GitHub}.

\textbf{Versión objetivo: 0.0.1}

El \textit{PMV} incluirá:

\begin{itemize}
	\item La estructura del repositorio está definida e implementada.
	\item Los proyectos necesarios están creados y listos para continuar con el desarrollo de nuevas funcionalidades.
	\item \textit{CI} está listo para ejecutar los diferentes tests y pruebas implementadas en los distintos proyectos.
	\item La documentación hasta este punto del desarrollo está actualizada y disponible.
\end{itemize}

Decisiones técnicas y documentación adicional:

\begin{itemize}
	\item Lenguaje de programación y \textit{framework}.
	\item Integración continua.
	\item Arquitectura.
\end{itemize}

\subsubsection{01 - Modelado del dominio del problema y lógica de negocio}

Enlace en \href{https://github.com/harvestcore/matroos/milestone/12}{GitHub}.

\textbf{Versión objetivo: 0.0.2}

El \textit{PMV} incluirá:

\begin{itemize}
	\item El dominio del problema está modelado.
	\item La lógica de negocio en su forma más básica está definida.
	\item La documentación hasta este punto del desarrollo está actualizada y disponible.
\end{itemize}

Decisiones técnicas y documentación adicional:

\begin{itemize}
	\item Arquitectura.
	\item Comandos.
	\item Bots.
	\item Herramientas.
\end{itemize}

\subsubsection{02 - Gestión de comandos}

Enlace en \href{https://github.com/harvestcore/matroos/milestone/10}{GitHub}.

\textbf{Versión objetivo: 0.0.3}

Tras la finalización de este \textit{milestone}:

\begin{itemize}
	\item La estructura de los comandos está definida.
	\item El \textit{backend} cuenta con un servicio capaz de gestionar los comandos y su configuración.
	\item Los tipos de comandos básicos quedan definidos y se pueden crear nuevos comandos de estos tipos.
	\item La documentación hasta este punto del desarrollo está actualizada y disponible.
\end{itemize}

Decisiones técnicas y documentación adicional:

\begin{itemize}
	\item Arquitectura.
	\item Comandos.
\end{itemize}

\subsubsection{03 - Gestión de bots}

Enlace en \href{https://github.com/harvestcore/matroos/milestone/6}{GitHub}.

\textbf{Versión objetivo: 0.0.4}


El \textit{PMV} incluirá:

\begin{itemize}
	\item La estructura de los bots está definida.
	\item El \textit{backend} cuenta con un servicio capaz de gestionar los bots y su configuración.
	\item Es posible asociar comandos a bots.
	\item La documentación hasta este punto del desarrollo está actualizada y disponible.
\end{itemize}

Decisiones técnicas y documentación adicional:

\begin{itemize}
	\item Arquitectura.
	\item Bots.
\end{itemize}

\subsubsection{04 - Despliegue de bots en \textit{workers}}

Enlace en \href{https://github.com/harvestcore/matroos/milestone/5}{GitHub}.

\textbf{Versión objetivo: 0.0.5}


El \textit{PMV} incluirá:

\begin{itemize}
	\item Es posible desplegar (ejecutar) bots en los \textit{workers}.
	\item El \textit{backend} es capaz de comunicarse con los distintos \textit{workers}.
	\item La documentación hasta este punto del desarrollo está actualizada y disponible.
\end{itemize}

Decisiones técnicas y documentación adicional:

\begin{itemize}
	\item Lenguaje de programación y \textit{framework}.
	\item Arquitectura.
\end{itemize}

\subsubsection{05 - API REST}

Enlace en \href{https://github.com/harvestcore/matroos/milestone/7}{GitHub}.

\textbf{Versión objetivo: 0.1.0}


El \textit{PMV} incluirá:

\begin{itemize}
	\item La \textit{API Rest} está definida y los \textit{endpoints} están documentados.
	\item Es posible realizar las tareas de administración de bots y comandos haciendo uso de la API.
	\item La documentación hasta este punto del desarrollo está actualizada y disponible.
\end{itemize}

Decisiones técnicas y documentación adicional:

\begin{itemize}
	\item Lenguaje de programación y \textit{framework}.
	\item Arquitectura.
\end{itemize}

\subsubsection{06 - Despliegue en contenedores Docker}

Enlace en \href{https://github.com/harvestcore/matroos/milestone/2}{GitHub}.

\textbf{Versión objetivo: 0.2.0}


El \textit{PMV} incluirá:

\begin{itemize}
	\item El software es distribuible mediante contenedores Docker.
	\item El archivo \textit{Dockerfile} para el microservicio del \textit{backend} está disponible.
	\item El archivo \textit{Dockerfile} para el microservicio del \textit{worker} está disponible.
	\item El archivo \textit{Docker Compose} para orquestar los microservicios está disponible.
	\item La documentación hasta este punto del desarrollo está actualizada y disponible.
\end{itemize}

Decisiones técnicas y documentación adicional:

\begin{itemize}
	\item Despliegue en contenedores.
	\item Arquitectura.
\end{itemize}

\subsubsection{07 - Almacén de datos}

Enlace en \href{https://github.com/harvestcore/matroos/milestone/11}{GitHub}.

\textbf{Versión objetivo: 0.3.0}


El \textit{PMV} incluirá:

\begin{itemize}
	\item Tanto los bots como los comandos son almacenables en base de datos.
	\item La documentación hasta este punto del desarrollo está actualizada y disponible.
\end{itemize}

Decisiones técnicas y documentación adicional:

\begin{itemize}
	\item Herramientas (Base de datos).
	\item Arquitectura.
	\item Base de datos.
\end{itemize}

\subsection{Milestones adicionales}

\subsubsection{08 - Interfaz de usuario}

Enlace en \href{https://github.com/harvestcore/matroos/milestone/9}{GitHub}.

\textbf{Versión objetivo: 0.4.0}


El \textit{PMV} incluirá:

\begin{itemize}
	\item La interfaz de usuario está disponible y es capaz de realizar las tareas de creación y configuración de comandos y bots, además del despliegue de éstos en \textit{workers}.
	\item La interfaz de usuario es distribuible mediante contenedores \textit{Docker}.
	\item El archivo \textit{Dockerfile} para el microservicio está disponible.
	\item La documentación hasta este punto del desarrollo está actualizada y disponible.
\end{itemize}

Decisiones técnicas y documentación adicional:

\begin{itemize}
	\item \textit{Frontend}.
	\item Arquitectura.
\end{itemize}

	\input{secciones/05_analisis}
	\input{secciones/06_implementacion}
	\input{secciones/07_conclusiones}
	
	\newpage
	\bibliography{biblio}
	\bibliographystyle{plain}
	
\end{document}

