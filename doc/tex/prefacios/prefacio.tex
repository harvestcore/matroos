\thispagestyle{empty}

\begin{center}
{\large\bfseries Matroos \\ Creación, configuración y despliegue de bots en \textit{Discord}. }\\
\end{center}
\begin{center}
Ángel Gómez Martín
\end{center}


\vspace{0.5cm}
\noindent{\textbf{Palabras clave}: software libre, \textit{Discord}, bot, \textit{API REST}, despliegue, \textit{backend}, \textit{frontend}, \textit{worker}
\vspace{0.7cm}

\noindent{\textbf{Resumen}\\

La creación y uso de bots se ha popularizado mucho en los últimos años, siendo extraño no encontrarlos integrados en multitud de sistemas. De entre estos destaca \textit{Discord}, una plataforma de mensajería instantánea utilizada principalmente por jóvenes que también ha ganado gran relevancia recientemente y donde los bots son ampliamente usados. Aunque los bots puedan parecer algo sencillo, los procesos que conllevan crearlos, configurarlos y desplegarlos pueden ser bastante complejos.

No obstante estos se pueden simplificar y unificar. En este proyecto se ha desarrollado una solución que aúna todos esos procesos en un puesto centralizado compuesto por un \textit{backend}, un \textit{frontend} y una serie de servicios llamados \textit{workers}, siendo un conjunto que permite facilitar y agilizar el desarrollo de estas tareas. Además ofrece modularidad de sus características, pudiendo adecuar las funcionalidades de cada bot adecuándose a las necesidades requeridas en cada situación. También ofrece una \textit{API REST}, la cual permite la comunicación con otro tipo de aplicaciones.
	

\cleardoublepage

\begin{center}
{\large\bfseries Matroos \\ Creation, configuration and deployment of bots in \textit{Discord}. }\\
\end{center}
\begin{center}
	Ángel Gómez Martín
\end{center}
\vspace{0.5cm}
\noindent{\textbf{Keywords}: \textit{open source}, \textit{Discord}, bot, \textit{REST API}, \textit{deployment}, \textit{backend}, \textit{frontend}, \textit{worker}
\vspace{0.7cm}

\noindent{\textbf{Abstract}\\

The creation and use of bots has become very popular in recent years, and it is strange not to find them integrated into a multitude of systems. Among these, \textit{Discord}, an instant messaging platform used mainly by young people, has also gained great relevance recently and where bots are widely used. While bots may seem straightforward, the processes involved in creating, configuring and deploying them can be quite complex.

However, these can be simplified and unified. In this project, a solution has been developed that brings together all these processes in a centralised position composed of a backend, a frontend and a series of services called ``workers''; a set that facilitates and speeds up the development of these tasks. It also offers modularity of its characteristics, being able to adapt the functionalities of each bot to suit the needs required in each situation. It also offers a REST API, which allows communication with other types of applications.

\cleardoublepage

\thispagestyle{empty}

\noindent\rule[-1ex]{\textwidth}{2pt}\\[4.5ex]

D. \textbf{Juan Julián Merelo Guervós}, Profesor del Departamento de Arquitectura y Tecnología de Computadores de la Universidad de Granada.

\vspace{0.5cm}

\textbf{Informa:}

\vspace{0.5cm}

Que el presente trabajo, titulado \textit{\textbf{Matroos}}, ha sido realizado bajo mi supervisión por \textbf{Ángel Gómez Martín}, y autorizo la defensa de dicho trabajo ante el tribunal que corresponda.

\vspace{0.5cm}

Y para que conste, expiden y firman el presente informe en Granada a 7 de Julio de 2022.

\vspace{1cm}

\textbf{El director:}

\vspace{5cm}

\noindent Fdo: Juan Julián Merelo Guervós



\chapter*{Agradecimientos}

A mi tutor, JJ, por ofrecerme su ayuda, conocimientos y acertados comentarios para la realización de este proyecto.

A mis padres, Elia y Ángel, por ser un pilar fundamental y por empujarme a seguir aprendiendo cosas nuevas y a superarme cada día.

A mi hermana Cristina, por apoyarme y echarme una mano siempre que lo he necesitado.

Y a Paula, por aguantarme más que nadie todo este tiempo.
