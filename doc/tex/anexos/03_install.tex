\chapter{Instalación del sistema}

Una vez se ha clonado \href{https://github.com/harvestcore/matroos}{el repositorio} se puede instalar el software siguiendo los pasos que se describen a continuación.

\section{\textit{Backend} y \textit{workers}}


Para ejecutar estos servicios se recomienda el uso de \textit{Visual Studio} (abriendo los archivos de las soluciones (\textit{*.sln}), ya que facilita mucho este proceso. Aún así, se puede ejecutar mediante línea de comandos:

\begin{lstlisting}
cd matroos

// Compilar Backend.
cd backend
dotnet restore
dotnet build --no-restore -c Release
dotnet publish -c Release -o ./app --no-restore

// Compilar Worker.
cd worker
dotnet restore
dotnet build --no-restore -c Release
dotnet publish -c Release -o ./app --no-restore

// Ejecutar backend.
cd backend/app
dotnet Matroos.Backend.dll

// Ejecutar worker.
cd worker/app
dotnet Matroos.Worker.dll
\end{lstlisting}

En cambio, para ejecutar de una manera mucho más sencilla los servicios, se puede usar \textit{Docker}.

\begin{lstlisting}[language=sh]
cd matroos

// Construir imagenes.
docker-compose build

// Ejecutar contenedores.
docker-compose up
\end{lstlisting}

Además, previa ejecución de los servicios, se deben revisar las variables de entorno.




\section{Frontend}

Ejecución del \textit{frontend} instalando y construyendo el servicio de manera local:

\begin{lstlisting}[language=sh]
cd matroos/frontend

// Instalación de dependencias.
npm install

// Ejecución de un entorno de pruebas.
npm run start

// Construcción de una versión de producción.
npm run build
\end{lstlisting}

En el caso de utilizar una \textit{build} de producción se recomienda utilizar un servidor web como \textit{NGINX}, aunque puede usarse cualquier otro.

Por otro lado se puede ejecutar el \textit{frontend} haciendo uso de \textit{Docker}, para ello es necesario ejecutar:

\begin{lstlisting}[language=sh]
cd matroos/frontend

// Construir imagen.
docker-compose build

// Ejecutar contenedor.
docker-compose up
\end{lstlisting}

De cualquier modo, previa ejecución del \textit{frontend} debe configurarse la variable \textit{BACKEND\_URL}, la cual se encuentra en el archivo \textit{frontend/api/fetch.ts}. En un ambiente local puede definirse la variable de entorno con ese mismo nombre, en el caso de una \textit{build} de producción o en el uso de \textit{Docker} debe asignarse esa variable manualmente.
