\chapter{Introducción}

Los bots son software que, mediante la ejecución de algoritmos, permiten automatizar tareas repetitivas de manera autónoma en un sistema. El uso de estos se ha popularizado mucho recientemente, tanto es así que es raro no encontrarlos integrados en cualquier aplicación, ya sea de mensajería instantánea o de otros ámbitos.

Aunque parezca sencillo, todo lo que rodea a un bot puede llegar a ser intrincado. Sin ir mas lejos su creación en muchas ocasiones es compleja y no está al alcance de todos los usuarios. Esta creación implica una serie de procedimientos y conocimientos informáticos de los cuales no cualquier usuario dispone. Por otro lado, se requiere también de una infraestructura que puede ser compleja de implementar.

Existen herramientas que intentan simplificar este proceso para un público general, pero las características que ofrecen son demasiado básicas y en casos concretos sus características son insuficientes. Además, en la mayoría de casos no permiten reutilizar funcionalidad, y no se tiene control total de los bots utilizados.

Con la creciente popularidad de los bots y de las plataformas sociales, cada día hay mas gente que quiere hacer uso de estos bots, ya sea por temas de trabajo o por que quieren pasar el rato, pero como se indica, crear los bots es complejo. Por un lado un público general busca crear bots de manera sencilla sin tener que preocuparse por detalles técnicos, y por otro existen usuarios más avanzados buscan poder implementar funcionalidades más complejas de la manera más sencilla posible.

\section{Concretización del problema}

Previa introducción al problema concreto, en la siguiente sección se introducen todos aquellos conceptos y términos que se usan a lo largo de este documento, y que es útil conocer y comprender para la lectura.

\subsection{Conceptos básicos}

\subsubsection{Comandos}

Los comandos son órdenes o instrucciones concretas que se pueden ejecutar a voluntad de un usuario en cualquier momento. Se invocan haciendo uso de una palabra clave, aunque pueden ser también tareas fijas y establecidas ejecutadas de manera automática sin intervención de un usuario. Además, estos suelen admitir parámetros o argumentos, algo que permite modificar su comportamiento.

\subsubsection{Bots}

Como se indicaba antes, los bots son programas informáticos con funcionalidades definidas. En concreto, los bots están compuestos por distintos elementos:

\begin{itemize}
	\item Código fuente. Creado por un desarrollador. Define la funcionalidad interna y las capacidades que va a tener ese bot.
	\item Comandos. Una serie de estos que el bot puede ejecutar.
	\item Contexto de ejecución. El entorno que percibe el bot en el momento de ser invocada alguna de sus funcionalidades. En el caso de sistemas conversacionales, si un bot es invocado en un chat grupal el contexto es ese grupo; en el caso de una interacción aislada con el bot, el contexto es esa conversación privada.
\end{itemize}

En el caso de este proyecto un bot es la entidad que se compone de los elementos elementos anteriores, y que puede ser desplegada (ejecutada) de manera sencilla en el software creado. Por tanto, el software permite la ejecución de múltiples bots de una forma transparente para el usuario, teniendo solo que desplegar (en una plataforma \textit{cloud} por ejemplo) el software desarrollado.

\subsubsection{Sistemas de mensajería instantánea}

Permiten una comunicación basada en texto a través de la red en tiempo real entre usuarios. Además de mensajes, multitud de sistemas de este tipo permiten la comunicación a través de llamadas de voz o vídeo, y permiten también compartir documentos.

En general estos sistemas se organizan la comunicación entre usuarios de distinta manera, como las siguientes:

\begin{itemize}
	\item Conversaciones privadas. Chat privado entre dos usuarios.
	\item Grupos. Pequeñas comunidades de usuarios que permiten el intercambio de mensajes entre todos ellos.
	\item Canales. Similar a un grupo, pero sólo el dueño del canal puede enviar mensajes a los usuarios suscritos.
\end{itemize}

Los protocolos que soportan estos sistemas son varios, y cada aplicación utiliza los que mejor se adecúan a las necesidades concretas. Algunos ejemplos son \textit{WebRTC}, \textit{WebSocket}, \textit{IRC} o \textit{XMPP}.

En concreto, en cuanto a bots, es bastante usual la integración de estos ya que aportan funcionalidad extra a estas aplicaciones. Generalmente los bots se ejecutan a nivel de sistema operativo en  como un software más, por lo que no se ejecutan en los dispositivos de los usuarios, sino en servidores y/o máquinas remotas encargadas de alojar software de ese tipo. Aún así existen excepciones, y es posible ejecutar bots como plugins o extensiones de un software.


\subsubsection{\textit{Discord}}

Es una aplicación de mensajería instantánea multiplataforma semejante a otras herramientas que se usan en ámbitos similares, como \textit{Slack} o \textit{TeamSpeak}, y ofrece distintos servicios de comunicación, como mensajería instantánea, chat de voz y vídeo e integración con bots y videojuegos, haciendo uso del protocolo \textit{WebRTC}.

Esta integración con bots es bastante importante, ya que han hecho que \textit{Discord} gane una gran visibilidad en los últimos años y se haya hecho muy popular incluso en ámbitos profesionales. La herramienta destaca también en otros aspectos\cite{earthweb} frente a otras alternativas, como la sencilla integración con otras herramientas o los más de 150 millones de usuarios activos al mes, siendo una de las aplicaciones más utilizadas hoy en día.

En \textit{Discord} los usuarios interactúan en \textbf{servidores}, que son comunidades construidas en torno a multitud de ámbitos, como por ejemplo deportes, cine o incluso en el mundo profesional. En estos servidores existen dos tipos de canales: de texto, donde los usuarios pueden mandar mensajes; y de voz, donde los usuarios pueden conversar.

En el caso de esta aplicación, y de igual manera ocurre en muchas otras, existen distintas maneras de crear bots:

\begin{itemize}
	\item \textbf{Alto nivel}. Mediante el uso de herramientas \textit{no-code} que permite la creación sencilla de bots y comandos, obviando todos los detalles técnicos.
	\item \textbf{Bajo nivel}. Haciendo uso de recursos más complejos, como librerías y herramientas de programación, para la creación de los bots desde cero.
\end{itemize}

\subsubsection{\textit{Discord.NET}}

Es una librería del lenguaje de programación \textit{C\#} que permite crear bots que interactúan con la \textit{API} de \textit{Discord}. Es sencilla de utilizar y ofrece una mayor versatilidad frente a las alternativas existentes, además de ser la más usada por la comunidad de usuarios.

Permite realizar multitud de acciones, como enviar y recibir mensajes, escuchar y enviar audio en tiempo real, enviar contenido multimedia e incluso interaccionar con los mensajes que se envían en los distintos canales y chats de los servidores.


\subsection{Creación de bots en \textit{Discord}}

La mejor manera de ilustrar el problema es con un ejemplo. Supongamos una pequeña empresa que utiliza \textit{Discord} como método de comunicación interna. Además, quieren utilizar bots como herramienta de alertas y para el control de aspectos técnicos de los departamentos de la empresa.

Como se introduce en las secciones anteriores, se pueden crear bots para integrarlos en \textit{Discord} de dos maneras.

En el primer caso las herramientas son muy básicas y limitantes, no pudiendo adaptarse a las necesidades más complejas que podría tener un usuario más avanzado, como en el ejemplo. Además suelen basarse en planes de pago y no son nada ampliables, con comandos no reutilizables y poco concretos. En caso de necesitar ampliar las funcionalidades, como podría ser una ligera modificación de un comando existente, no es posible realizar ese cambio.

En el segundo caso, existen multitud de librerías para crear bots desde cero, para multitud de lenguajes de programación. Algunas tienen más o menos popularidad, y otras tienen mayor o menor cantidad de características. Sin embargo, estas herramientas son en ocasiones algo confusas e incluso ineficientes. Esto se debe a que cada una tiene una estructura distinta, e implementan los procedimientos de una manera concreta.

Por tanto, cuando esta pequeña empresa quiera incluir bots en sus servidores de \textit{Discord}, se van a enfrentar a herramientas que son por un lado bastante pobres, y por otro excesivamente amplias. Más información en el estado del arte.

\section{Solución propuesta}

En este proyecto se desarrolla un sistema que actúa como capa de abstracción sobre \textit{Discord.NET}, haciendo uso de \textit{C\#} como lenguaje de desarrollo, que:

\begin{itemize}
	\item Permite crear bots y comandos de \textit{Discord} configurables y reutilizables de manera agnóstica a las librerías y tecnologías que hay por debajo.
	\item Posibilita el despliegue de los bots de manera sencilla e independiente de los demás, pudiendo ejecutar distintos bots en el mismo sistema, o en remotos.
	\item Centraliza los procesos de creación y despliegue en un mismo sistema, a través de una interfaz de usuario.
	\item Facilita la ampliación del software fácilmente, para adaptarlo a cualquier ámbito. De esta manera se pueden crear distintos tipos de comandos reutilizables acordes a las necesidades de cada usuario.
	\item Permite las configuraciones de los bots y los comandos.
	\item Incluye una serie de comandos básicos listos para ser utilizados.
\end{itemize}

Resolver este problema puede ayudar a dos principales grupos de usuarios, aunque estos deben tener unos conocimientos mínimos previos de las herramientas que se utilizan.

\begin{itemize}
	\item Usuarios que no tienen un conocimiento técnico muy extenso pero que buscan una herramienta con la que crear bots de Discord fácilmente sin limitaciones.
	\item Usuarios con conocimiento más avanzado que buscan crear bots específicos y personalizados, sin tener que preocuparse por toda la infraestructura necesaria para crear y desplegar los bots.
\end{itemize}

\section{Motivación}

La principal motivación para realizar este proyecto es que crear bots con funcionalidades de las características mencionadas con las herramientas actuales es una tarea compleja que se puede simplificar. Crear una herramienta de este tipo puede favorecer la comunidad de desarrollo de bots y de herramientas de creación de bots, aparte de facilitar el acercamiento de los usuarios con menor conocimiento técnico a este proceso.

Aunque este problema suele ser general para todas las herramientas que permiten la integración con bots, este proyecto se centra en \textit{Discord}, pues es la plataforma que probablemente podría beneficiarse más de un software de este tipo dada la gran cantidad de usuarios que hace uso de ella y por el éxito\cite{enlyft} que está teniendo en multitud de ámbitos.

\section{Objetivos}

Este proyecto tiene los siguientes objetivos:

\begin{enumerate}
	\item Simplificar la creación de bots de \textit{Discord} para un usuario administrador, de modo que estos bots sean fácilmente extensibles y configurables frente al uso de la librería \textit{Discord.NET} y otros recursos de programación.
	\item Agilizar el proceso de despliegue de estos bots, de manera que se eliminen las trabas y procedimientos asociados al despliegue de estos sistemas.
	\item Facilitar la creación de comandos de \textit{Discord} de modo que estos sean sencillos de configurar, reutilizar y ampliar haciendo uso de herramientas de programación.
	\item Se publicarán versiones del software en los repositorios de imágenes de \textit{Docker} para facilitar la distribución del mismo.
	\item El tiempo de creación y despliegue de un bot de \textit{Discord} se reducirá al menos un 50\% frente al uso de herramientas de programación que permiten crear bots de este tipo.
\end{enumerate}

\section{Estructura del documento}

TBD
