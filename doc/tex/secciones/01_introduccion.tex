\chapter{Introducción}

Los bots permiten automatizar tareas en sistemas conversacionales. Sin embargo, su creación en muchas ocasiones es compleja y no está al alcance de todos los usuarios. Existen tareas que son sencillas de realizar de manera manual, pero que pueden requerir un tiempo y dedicación considerable, y sería mucho más cómodo automatizarlas de alguna manera. Por otra parte, cuando hay un gran volumen de datos sobre el que realizar estas tareas, automatizarlas otorgaría grandes beneficios ya que no dependerían de la disponibilidad de una persona para realizarlas. Además, se eliminaría la necesidad de realizar en un gran número de ocasiones un proceso que podría ser corto pero muy repetitivo. Estos procesos pueden ser de muchos tipos y sobre muchos ámbitos diferentes dependiendo del entorno en que sean necesarios, y como al fin y al cabo son procesos específicos, el sistema de automatización también debería ser así, pero desarrollarlo requiere de una experiencia técnica determinada. Esto no esta al alcance de todos.

En concreto, en la plataforma \textit{Discord}\cite{discord}, que se ha hecho popular recientemente, este problema no es una excepción. En \textit{Discord} existen comunidades construidas en torno a multitud de ámbitos, como videojuegos, deportes, cine o incluso en el mundo profesional, donde la plataforma se ha hecho un hueco. Estas comunidades pueden crecer mucho, y del mismo modo crecen las necesidades de moderación y funcionalidades adicionales que pueden ser útiles para facilitar ciertas tareas en función del ámbito de la comunidad. Un ejemplo de esto podría ser la automatización de mensajes para anunciar los estrenos de cartelera semanal de un cine en concreto o la monitorización de sistemas en una una pequeña empresa.

Resolver este problema puede ayudar a todos aquellos usuarios y comunidades que buscan automatizar tareas específicas de manera individual sin tener que recurrir a terceros con los conocimientos específicos y con los que quizás se tendría que compartir información sensible. Por otro lado, tener que recurrir a conocimiento experto externo conlleva una serie de costes que en algunos casos puede ser inadmisible. Este problema suele ser general para todas las herramientas que permiten la integración con bots, pero dado que en cada plataforma los bots se crean con tecnologías y procedimientos diferentes, este proyecto se centra en Discord pues es la plataforma que probablemente podría beneficiarse más de esta nueva herramienta dada la gran cantidad de usuarios que hace uso de ella.

\section{Motivación}

La principal motivación para realizar este proyecto es que crear bots para \textit{Discord} con funcionalidades muy concretas con las herramientas actuales es una tarea compleja. Si bien existen librerías para lenguajes de programación, las herramientas que han surgido para crear estos bots de forma sencilla se centran en aspectos muy básicos que no tienen mucha cabida en ámbitos más concretos.

No existen \textit{frameworks} o sistemas que permitan crear fácilmente comandos que se ajusten a necesidades específicas, y menos que sean reproducibles o configurables de alguna manera con distintos datos. Además, cuando se trata de crear y alojar distintos bots en un mismo sistema, las actuales soluciones requieren que cada uno de estos bots sea una instancia independiente. Esto imposibilita la configuración de todos estos bots de una manera sencilla dentro de un mismo sistema. De nuevo, algunos de estos conceptos requieren de conocimientos experto extra necesarios para crear estas herramientas no están al alcance de todos.

\section{Problema y solución}

Como se indicaba en la sección anterior, usuarios con necesidades específicas (como por ejemplo podría ser un administrador de sistemas de una pequeña empresa, o incluso un usuario entusiasta al que le gusta automatizar tareas en su comunidad de \textit{Discord}) que quieren hacer uso de este software y de su sistema de bots se ven obligados a crear distintos bots muy específicos y en ocasiones poco reutilizables. Si bien se podrían programar comandos más concretos en un único bot, sería una tarea tediosa la reutilización entre diferentes ámbitos. Por ejemplo, dos comunidades de juegos de mesa, donde se comparten muchas funcionalidades, pero dependiendo del juego los detalles son distintos.

Incidiendo en el aspecto de la creación de bots forma más técnica, se observa que actualmente hay tres maneras de usar bots en \textit{Discord}:

\begin{itemize}
	\item \textbf{Usar bots ya existentes}. El bot ya se encuentra creado, configurado y desplegado, y solo es necesario agregarlo al servidor para poder disfrutar de sus funcionalidades.
	\item \textbf{Crear un bot a alto nivel}. Este caso es similar al anterior, ya que se trata de un bot genérico, que es configurable en cierta medida para cada servidor. Esta configuración se hace a través de algún tipo de herramienta (generalmente una aplicación web) que permite configurar los comandos deseados. Por otro lado, ya que están pensados para un público general, las posibilidades de configuración son escasas. Suelen tener algunas plantillas de comandos básicos, como temporizadores o respuestas automáticas.
	\item \textbf{Crear un bot a bajo nivel}. En este caso el bot se crea haciendo uso de las diferentes \textit{API} que ofrece \textit{Discord} para ello y la personalización es máxima. En cambio, es más tedioso, y requiere conocimientos extra que algunos usuarios pueden no tener (como programación). Además hay que tener en cuenta que el bot debe ser desplegado manualmente, por lo que requiere un esfuerzo extra.
\end{itemize}

Como se observa, en el primer caso la capacidad de configuración del bot es nula y en el segundo las funcionalidades son muy reducidas. En el último caso, aunque permite realizar cualquier configuración, se pierde el aspecto de la administración central, ya que se crean bots independientes. 

Por tanto, se pretende desarrollar un \textit{framework} para crear bots de \textit{Discord} completamente configurables, y que además permita la creación sencilla de comandos. También, se busca también la centralización de procesos.
