\chapter{Introducción}

\textit{Discord}\cite{discord} es una aplicación de mensajería de uso tanto personal como en empresas. Es semejante a otras herramientas que se usan en ámbitos similares, como \textit{Slack}\cite{slack} o \textit{TeamSpeak}\cite{teamspeak}, y ofrece distintos servicios de comunicación, como mensajería instantánea, chat de voz e integración con bots y videojuegos.

Estos últimos elementos son bastante importantes, ya que la integración con videojuegos ha hecho que \textit{Discord} gane visibilidad en los últimos años; y la integración con bots ha hecho que la herramienta sea una herramienta mucho más interactiva que las otras mencionadas. Más información en el estado del arte.

En concreto, este proyecto se centra en los bots, que podrían definirse como herramientas que, haciendo uso de diferentes permisos, ayudan a automatizar tareas dentro de un servidor de \textit{Discord}. Las funciones de éstos son prácticamente ilimitadas, aunque las más populares son moderación de usuarios y mensajes, música, envío de contenido, noticias o encuestas.

Usuarios más avanzados (como por ejemplo, un administrador de sistemas, o alguien que controla distintos equipos) que quieren hacer uso de este software y de su sistema de bots se ven obligados a crear distintos bots muy específicos y en ocasiones poco reutilizables. Si bien se podrían programar comandos más específicos en un único bot, sería una tarea tediosa la reutilización entre diferentes ámbitos.

Incidiendo en el aspecto de la creación de bots forma más técnica, se observa que actualmente hay tres maneras de usar bots en \textit{Discord}:

\begin{itemize}
	\item \textbf{Usar bots ya existentes}. El bot ya se encuentra creado, configurado y desplegado, y solo es necesario agregarlo al servidor para poder disfrutar de sus funcionalidades.
	\item \textbf{Crear un bot a alto nivel}. Este caso es similar al anterior, ya que se trata de un bot genérico, que es configurable en cierta medida para cada servidor. Esta configuración se hace a través de algún tipo de herramienta (generalmente una aplicación web) que permite configurar los comandos deseados. Por otro lado, ya que están pensados para un público general, las posibilidades de configuración son escasas. Suelen tener algunas plantillas de comandos básicos, como temporizadores o respuestas automáticas.
	\item \textbf{Crear un bot a bajo nivel}. En este caso el bot se crea haciendo uso de las diferentes \textit{API} que ofrece \textit{Discord} para ello y la personalización es máxima. En cambio, es más tedioso, y requiere conocimientos extra que algunos usuarios pueden no tener (como programación). Además hay que tener en cuenta que el bot debe ser desplegado manualmente, por lo que requiere un esfuerzo extra.
\end{itemize}

Además, en el caso de que se quisieran desplegar distintos bots al mismo tiempo y administrarlos desde un mismo entorno, no sería posible, ya que cada uno de estos es una instancia distinta.

Por tanto, se pretende desarrollar un \textit{framework} para crear bots de \textit{Discord} configurables que sean capaces de conectar con diferentes sistemas. La principal motivación a la hora de desarrollar este sistema es encontrar la manera de centralizar la creación y la administración de bots de este tipo, además de resolver los diferentes problemas planteados:

\begin{itemize}
	\item Dificultad al crear bots con funcionalidades específicas.
	\item Poca configuración en aspectos más técnicos, como administración de sistemas.
	\item Dificultad de manejar el despliegue de distintos bots al mismo tiempo.
\end{itemize}

Es por tanto, un software pensado para usuarios mas avanzados, ya que los usuarios con menor conocimiento de las tareas administrativas sólo hacen uso de estos bots una vez ya desplegados a través de los servidores de \textit{Discord}.
