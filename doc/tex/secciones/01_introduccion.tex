\chapter{Introducción}

El uso de bots se ha popularizado mucho recientemente, tanto es así que es raro no encontrarlos integrados en cualquier aplicación, ya sea de mensajería o de otros ámbitos. Éstas herramientas permiten automatizar tareas reiterativas mediante el uso de una serie de comandos o funciones autónomas.

Los comandos son órdenes que puede ejecutar el bot a voluntad del usuario en cualquier momento, y que se ejecutan en el contexto actual donde se encuentre interactuando un usuario con el bot. Usualmente estos comandos se invocan haciendo uso de una palabra clave, aunque pueden ser también tareas fijas y establecidas ejecutadas de manera automática sin intervención de un usuario.

En el caso de los sistemas conversacionales existen comunidades construidas en torno a multitud de ámbitos, como por ejemplo videojuegos, deportes, cine o incluso en el mundo profesional. Estas comunidades pueden crecer mucho, y del mismo modo crecen las necesidades de moderación y funcionalidades adicionales que pueden ser útiles para facilitar ciertas tareas en cada comunidad. Sin embargo, su creación en muchas ocasiones es compleja y no está al alcance de todos los usuarios.

De estos sistemas conversacionales destaca \textit{Discord}\cite{earthweb}, una aplicación de mensajería instantánea multiplataforma. Es semejante a otras herramientas que se usan en ámbitos similares, como \textit{Slack} o \textit{TeamSpeak}, y ofrece distintos servicios de comunicación, como mensajería instantánea, chat de voz y vídeo e integración con bots y videojuegos.

Estos bots son bastante importantes, ya que junto con la integración con videojuegos, han hecho que \textit{Discord} gane visibilidad en los últimos años y se haya hecho muy popular incluso en ámbitos profesionales. 

En el caso de esta aplicación, y de igual manera ocurre en muchas otras, existen distintas maneras de crear bots:

\begin{enumerate}
	\item \textbf{Alto nivel}. Mediante el uso de herramientas \textit{no-code} que permite la creación sencilla de bots y comandos, obviando todos los detalles técnicos.
	\item \textbf{Bajo nivel}. Haciendo uso de recursos más complejos, como librerías y herramientas de programación, para la creación de los bots desde cero.
\end{enumerate}

En el primer caso las herramientas son correctas y cumplen lo que prometen, pero son muy básicas, no pudiendo adaptarse a las necesidades más complejas que podría tener un usuario más avanzado. Además suelen basarse en planes de pago, donde los gratuitos ofrecen funcionalidades muy reducidas y los de pago las herramientas extra.

Por otro lado, no son nada ampliables. En caso de necesitar ampliar las funcionalidades, como podría ser una ligera modificación de un comando existente, no es posible realizar ese cambio. En este caso sería preciso recurrir a las herramientas del segundo caso para crear un bot desde cero con las características deseadas.

Otro aspecto a tener en cuenta es que no se tiene el control de los despliegues con este sistema. Casi en la totalidad de casos estas herramientas ofrecen un único bot al que se le configuran una serie de comandos específicos para cada comunidad, por lo que no es posible crear distintos bots y menos reutilizar los comandos entre estos.

Un ejemplo de esto podría ser una pequeña empresa que utiliza \textit{Discord} como método de comunicación interna. Además, quieren utilizar bots como herramienta de alertas y para el control de aspectos técnicos de los departamentos de la empresa.

Usar las herramientas actuales es muy limitante, pues no cubre las necesidades del ámbito del ejemplo. Esta empresa tendría que usar bots únicos, con comandos no reutilizables y poco concretos. Asimismo, si quisieran aislar funcionalidades en distintos bots no sería posible, y si lo fuera no estarían bajo su control.

Volviendo al segundo punto de la lista anterior, existen multitud de librerías para crear bots, para multitud de lenguajes de programación. Algunas tienen más o menos popularidad, y otras tienen mayor o menor cantidad de características.

En general todas ellas proveen las herramientas necesarias para crear tanto bots como comandos, pero estas herramientas son en ocasiones algo confusas e incluso ineficientes. Esto se debe a que cada una tiene una estructura distinta, e implementan los procedimientos de una manera concreta. También, cuando los usuarios hacen uso de estas librerías y herramientas se comete un error muy habitual, ya que no reutiliza el código generado todo lo que se podría y debería.

En concreto, el lenguaje de programación C\# cuenta con dos principales librerías que permiten interactuar con la API de Discord, ambas actualizadas regularmente y con una comunidad de usuarios bastante importante. Sin embargo, estas dos herramientas se centran en la creación de bots únicos, un detalle que es importante, ya que como se ha visto puede ser un aspecto limitante. En el caso de este proyecto se utiliza Discord.NET, ya que es más sencilla de utilizar y porque ofrece mayor versatilidad, además de ser la más usada por la comunidad de usuarios.

Por tanto, en este proyecto se desarrolla un sistema que actúa como capa de abstracción sobre Discord.NET, haciendo uso de C\# como lenguaje de desarrollo, que:

\begin{itemize}
	\item Permite crear bots y comandos de \textit{Discord} configurables y reutilizables de manera agnóstica a las librerías y tecnologías que hay por debajo.
	\item Posibilita el despliegue de los bots de manera sencilla e independiente de los demás, pudiendo ejecutar distintos bots en el mismo sistema, o en remotos.
	\item Centraliza los procesos de creación y despliegue en un mismo sistema, a través de una interfaz de usuario.
	\item Facilita la ampliación del software fácilmente, para adaptarlo a cualquier ámbito. De esta manera se pueden crear distintos tipos de comandos reutilizables acordes a las necesidades de cada usuario.
	\item Permite las configuraciones de los bots y los comandos.
	\item Incluye una serie de comandos básicos listos para ser utilizados.
\end{itemize}

Resolver este problema puede ayudar a dos principales grupos de usuarios, aunque estos deben tener unos conocimientos mínimos previos de las herramientas que se utilizan.

\begin{itemize}
	\item Usuarios que no tienen un conocimiento técnico muy extenso pero que buscan una herramienta con la que crear bots de Discord fácilmente sin limitaciones.
	\item Usuarios con conocimiento más avanzado que buscan crear bots específicos y personalizados, sin tener que preocuparse por toda la infraestructura necesaria para crear y desplegar los bots.
\end{itemize}

\section{Motivación}

La principal motivación para realizar este proyecto es que crear bots con funcionalidades de las características mencionadas con las herramientas actuales es una tarea compleja que se puede simplificar. Crear una herramienta de este tipo puede favorecer la comunidad de desarrollo de bots y de herramientas de creación de bots, aparte de facilitar el acercamiento de los usuarios con menor conocimiento técnico a la creación de bots.

Aunque este problema suele ser general para todas las herramientas que permiten la integración con bots, este proyecto se centra en \textit{Discord}, pues es la plataforma que probablemente podría beneficiarse más de un software de este tipo dada la gran cantidad de usuarios que hace uso de ella y por el éxito\cite{enlyft} que está teniendo en multitud de ámbitos.

\section{Objetivos}

El desarrollo de este proyecto tiene los siguientes objetivos:

\begin{enumerate}
	\item Simplificar la creación de bots de \textit{Discord} para un usuario administrador, de modo que estos bots sean fácilmente extensibles y configurables frente al uso de la librería \textit{Discord.NET} y otros recursos de programación.
	\item Agilizar el proceso de despliegue de estos bots, de manera que se eliminen las trabas y procedimientos asociados al despliegue de estos sistemas.
	\item Facilitar la creación de comandos de \textit{Discord} de modo que estos sean sencillos de configurar, reutilizar y ampliar haciendo uso de herramientas de programación.
	\item Se publicarán versiones del software en los repositorios de imágenes de \textit{Docker} para facilitar la distribución del mismo.
	\item El tiempo de creación y despliegue de un bot de \textit{Discord} se reducirá al menos un 50\% frente al uso de herramientas de programación que permiten crear bots de este tipo.
\end{enumerate}

\section{Estructura del documento}

TBD
