\chapter{Introducción}

Los bots permiten automatizar tareas en sistemas conversacionales. Sin embargo, su creación en muchas ocasiones es compleja y no está al alcance de todos los usuarios. Existen tareas que son sencillas de realizar de manera manual, pero que pueden requerir un tiempo y dedicación considerable, y sería mucho más cómodo automatizarlas de alguna manera. Por otra parte, tomando como ejemplo un gran volumen de datos sobre el que realizar estas tareas, automatizarlas otorgaría grandes beneficios ya que no dependerían de la disponibilidad de una persona para realizarlas. Además, se eliminaría la necesidad de realizar en un gran número de ocasiones un proceso que podría ser corto pero muy repetitivo. Estos procesos pueden ser de muchos tipos y sobre muchos ámbitos diferentes dependiendo del entorno en que sean necesarios, y como al fin y al cabo son procesos específicos, el sistema de automatización también debería ser así, pero desarrollarlo requiere de una experiencia técnica determinada. Esto no esta al alcance de todos.

El uso de estos bots se ha popularizado mucho recientemente, tanto es así que es raro no encontrarlos integrados en cualquier aplicación, ya sea de mensajería o de otros ámbitos. En el caso de los sistemas conversacionales existen comunidades construidas en torno a multitud de ámbitos, como videojuegos, deportes, cine o incluso en el mundo profesional. Estas comunidades pueden crecer mucho, y del mismo modo crecen las necesidades de moderación y funcionalidades adicionales que pueden ser útiles para facilitar ciertas tareas en función del ámbito de la comunidad. Un ejemplo de esto podría ser la automatización de mensajes para anunciar los estrenos de cartelera semanal de un cine en concreto o la monitorización de sistemas en una una pequeña empresa. Para esto, se utilizan comandos, que son órdenes que puede ejecutar el bot a voluntad del usuario en cualquier momento.

Sin embargo, como se mencionaba anteriormente, la creación es compleja. Cada plataforma integra los bots de una manera única, utilizando procedimientos, tecnologías y herramientas diferentes. En ocasiones estas plataformas ofrecen recursos diversos para facilitar esta integración, como pueden librerías para distintos lenguajes de programación, pero estas requieren de unos conocimientos adicionales que la mayoría de usuarios no tienen. Para solucionar este problema han surgido herramientas que facilitan el proceso de creación y configuración de los bots, de sus comandos y de los distintos parámetros de configuración adicionales que se requieran.

Estas herramientas, aunque cumplen con lo que prometen, tienen una gran carencia, y es que las posibilidades de configuración de los bots son muy limitadas. Están pensadas para un público muy general, y las posibilidades son también muy generales, tendiendo a ser escasas, con funcionalidades muy simples que en ocasiones no son suficientes o no cumplen con las necesidades de comunidades que requieren funcionalidades más complejas.

Un ejemplo de esto son los bots de moderación de conocidas aplicaciones de mensajería instantánea, donde las funcionalidades incluyen una moderación de usuarios muy básica, pero no incluyen características extra para poder modificar las reglas de moderación. Este ejemplo sirve para ver el segundo gran problema de estas herramientas, los costes. En la gran mayoría de casos estas herramientas tienen una modalidad gratuita, con esas características muy básicas, y aparte cuentan con modalidades de pago o \textit{premium} que incluyen funcionalidades extendidas, pero que aún así pueden quedarse cortas en algunos ámbitos concretos.

Otro ejemplo podría ser una pequeña empresa que hace uso de alguna aplicación de mensajería interna, para la que existe una herramienta online que permite crear y desplegar bots con algunos comandos. Si bien la herramienta permite crear bots, sólo permite crear uno en su paquete gratuito, y los comandos son muy básicos, ya que sólo les permite realizar moderación de mensajes y de usuarios. Dado el caso en el que esta empresa quiera incluir un bot para obtener información de cierto departamento de la misma, la mencionada herramienta deja de ser útil. Como se mencionaba anteriormente es ahora necesario de un conocimiento experto extra para poder desarrollar e implementar ese bot concreto.

En resumen, cuando se trata de integrar bots con funcionalidades muy concretas, las posibilidades son escasas. Resolver este problema puede ayudar a todos aquellos usuarios y comunidades que buscan automatizar tareas específicas de manera individual sin tener que recurrir a terceros con los conocimientos específicos. Por otro lado, tener que recurrir a conocimiento experto externo conlleva una serie de costes que en algunos casos puede ser inadmisible.

Este problema suele ser general para todas las herramientas que permiten la integración con bots, pero dado que en cada plataforma los bots se crean con tecnologías y procedimientos diferentes, este proyecto se centra en \textit{Discord}\cite{discord} pues es la plataforma que probablemente podría beneficiarse más de esta nueva herramienta dada la gran cantidad de usuarios que hace uso de ella. \textit{Discord} es un servicio de mensajería instantánea gratuito de chat de voz, video y texto, que ha ganado mucha popularidad en los últimos años gracias a la integración que tiene con videojuegos y los ya mencionados bots.

\section{Motivación}

La principal motivación para realizar este proyecto es que crear bots con funcionalidades de estas características con las herramientas actuales es una tarea compleja. En concreto, para desarrollar bots \textit{Discord} existen una gran multitud de librerías para lenguajes de programación, pero las herramientas de creación de bots, y especialmente comandos, tienen las características anterior mencionadas, son demasiado básicas.

Por otro lado, cuando se trata de aspectos como reproducibilidad, las herramientas actuales apenas cuentan con esta característica, por lo que éste es otro motivo por el que desarrollar este proyecto. Para ilustrar este caso, se podría ser crear un comando básico (que por ejemplo devuelva un mensaje que ayude a comprender cómo usar el bot), y este comando configurarlo en distintos bots, estando cada bot desplegado de manera independiente a los demás.

En definitiva, no existen \textit{frameworks} o sistemas que permitan crear fácilmente comandos que se ajusten a necesidades específicas, y menos que sean reproducibles o configurables de alguna manera con distintos datos. Además, cuando se trata de crear y alojar distintos bots en un mismo sistema, las actuales soluciones requieren que cada uno de estos bots sea una instancia independiente. Esto imposibilita la configuración de todos estos bots de una manera sencilla dentro de un mismo sistema. De nuevo, algunos de estos conceptos requieren de conocimientos experto extra necesarios para crear estas herramientas no están al alcance de todos.


\section{Problema y solución}

Como se indicaba en la sección anterior, usuarios con necesidades específicas (como por ejemplo podría ser un administrador de sistemas de una pequeña empresa, o incluso un usuario entusiasta al que le gusta automatizar tareas en su comunidad de \textit{Discord}) que quieren hacer uso de este software y de su sistema de bots se ven obligados a crear distintos bots muy específicos y en ocasiones poco reutilizables. Si bien se podrían programar comandos más concretos en un único bot, sería una tarea tediosa la reutilización entre diferentes ámbitos. Por ejemplo, dos comunidades de juegos de mesa, donde se comparten muchas funcionalidades, pero dependiendo del juego los detalles son distintos.

Incidiendo en el aspecto de la creación de bots forma más técnica, se observa que actualmente hay tres maneras de usar bots en \textit{Discord}:

\begin{enumerate}
	\item \textbf{Usar bots ya existentes}. El bot ya se encuentra creado, configurado y desplegado, y solo es necesario agregarlo al servidor para poder disfrutar de sus funcionalidades.
	\item \textbf{Crear un bot a alto nivel}. Este caso es similar al anterior, ya que se trata de un bot genérico, que es configurable en cierta medida para cada servidor. Esta configuración se hace a través de algún tipo de herramienta (generalmente una aplicación web) que permite configurar los comandos deseados. Por otro lado, ya que están pensados para un público general, las posibilidades de configuración son escasas. Suelen tener algunas plantillas de comandos básicos, como temporizadores o respuestas automáticas.
	\item \textbf{Crear un bot a bajo nivel}. En este caso el bot se crea haciendo uso de las diferentes \textit{API} que ofrece \textit{Discord} para ello y la personalización es máxima. En cambio, es más tedioso, y requiere conocimientos extra que algunos usuarios pueden no tener (como programación). Además hay que tener en cuenta que el bot debe ser desplegado manualmente, por lo que requiere un esfuerzo extra.
\end{enumerate}

Como se observa, en el primer caso la capacidad de configuración del bot es nula y en el segundo las funcionalidades son muy reducidas. En el último caso, aunque permite realizar cualquier configuración, se pierde el aspecto de la administración central, ya que se crean bots independientes. 

Por tanto, en este proyecto se desarrolla un \textit{framework}, que podría situarse entre las categorías dos y tres, que permite:

\begin{itemize}
	\item Crear bots y comandos de \textit{Discord} completamente configurables y reutilizables de manera agnóstica a las librerías y tecnologías que hay por debajo.
	\item Desplegar los bots de manera sencilla e independiente de los demás, pudiendo ejecutar distintos bots en el mismo sistema, o en remotos.
	\item Centralizar los procesos de creación y despliegue en un mismo sistema, a través de una interfaz de usuario.
	\item La ampliación del software fácilmente, para adaptarlo a cualquier ámbito.
	\item Almacenar las configuraciones de los bots y los comandos.
\end{itemize}
