\chapter{Introducción}

\textit{Discord}\cite{discord} es una aplicación de mensajería de uso tanto personal como en empresas. Es semejante a otras herramientas que se usan en ámbitos similares, como \textit{Slack}\cite{slack} o \textit{TeamSpeak}\cite{teamspeak}, y ofrece distintos servicios de comunicación, como mensajería instantánea, chat de voz e integración con bots y videojuegos.

Estos últimos elementos son bastante importantes, ya que la integración con videojuegos ha hecho que \textit{Discord} gane visibilidad en los últimos años; y la integración con bots ha hecho que la aplicación sea una herramienta mucho más interactiva que las otras mencionadas. Más información en el estado del arte.

En concreto, este proyecto se centra en los bots, que podrían definirse en este ámbito como herramientas que, haciendo uso de diferentes permisos, ayudan a automatizar tareas dentro de un servidor de \textit{Discord}. Las funciones de éstos son prácticamente ilimitadas, aunque las más populares son moderación de usuarios y mensajes, música, envío de contenido, noticias o encuestas.

\section{Motivación}

La principal motivación para realizar este proyecto es que crear bots para Discord con funcionalidades complejas con las herramientas actuales es prácticamente imposible. Si bien existen librerías para lenguajes de programación, las herramientas que han surgido para crear estos bots de forma sencilla se centran en aspectos muy básicos que no tienen mucha cabida en entornos más profesionales.

No existen frameworks o sistemas que permitan crear fácilmente cualquier comando que uno quiera, y menos que sea reproducible o configurable con distintos parámetros. Un ejemplo de esto podría ser hacer una petición PING a una dirección remota. Además, cuando se trata de crear y alojar distintos bots en un mismo sistema, las actuales soluciones requieren que cada uno de estos bots sea una instancia independiente. Esto imposibilita la configuración de todos estos bots de una manera sencilla dentro de un mismo sistema.

\section{Problema y solución}

Como se indicaba en la sección anterior, usuarios más avanzados (como por ejemplo podría ser un administrador de sistemas de una pequeña empresa, o incluso un usuario entusiasta al que le gusta automatizar tareas) que quieren hacer uso de este software y de su sistema de bots se ven obligados a crear distintos bots muy específicos y en ocasiones poco reutilizables. Si bien se podrían programar comandos más específicos en un único bot, sería una tarea tediosa la reutilización entre diferentes ámbitos.

Incidiendo en el aspecto de la creación de bots forma más técnica, se observa que actualmente hay tres maneras de usar bots en \textit{Discord}:

\begin{itemize}
	\item \textbf{Usar bots ya existentes}. El bot ya se encuentra creado, configurado y desplegado, y solo es necesario agregarlo al servidor para poder disfrutar de sus funcionalidades.
	\item \textbf{Crear un bot a alto nivel}. Este caso es similar al anterior, ya que se trata de un bot genérico, que es configurable en cierta medida para cada servidor. Esta configuración se hace a través de algún tipo de herramienta (generalmente una aplicación web) que permite configurar los comandos deseados. Por otro lado, ya que están pensados para un público general, las posibilidades de configuración son escasas. Suelen tener algunas plantillas de comandos básicos, como temporizadores o respuestas automáticas.
	\item \textbf{Crear un bot a bajo nivel}. En este caso el bot se crea haciendo uso de las diferentes \textit{API} que ofrece \textit{Discord} para ello y la personalización es máxima. En cambio, es más tedioso, y requiere conocimientos extra que algunos usuarios pueden no tener (como programación). Además hay que tener en cuenta que el bot debe ser desplegado manualmente, por lo que requiere un esfuerzo extra.
\end{itemize}

Como se observa, en el primer caso la capacidad de configuración del bot es nula y en el segundo las funcionalidades son muy reducidas. En el último caso, aunque permite realizar cualquier configuración, se pierde el aspecto de la administración central, ya que se crean bots independientes. 

Por tanto, se pretende desarrollar un \textit{framework} para crear bots de \textit{Discord} completamente configurables, y que además permita la creación sencilla de comandos. Además, se busca también la centralización de procesos.

Es, por tanto, un software pensado para usuarios con conocimientos más avanzados en estas tecnologías, ya que los usuarios con menor conocimiento sólo hacen uso de estos bots y comandos una vez ya se encuentran desplegados y disponibles a través de los servidores de \textit{Discord}.
