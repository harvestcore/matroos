\chapter{Conclusiones y trabajos futuros}

Se pueden extraer una serie de conclusiones respecto todas las etapas de desarrollo de este proyecto, desde la definición los objetivos iniciales, hasta las etapas de investigación y creación del código del software. El objetivo principal del proyecto era simplificar el proceso de creación y uso de bots de \textit{Discord}, y el software creado cumple con creces tanto este como los objetivos secundarios.

La herramienta simplifica ampliamente la creación de bots y comandos de \textit{Discord}, además de agilizar el proceso de configuración y despliegue de los bots. Además, sirve como plataforma de desarrollo de nuevos tipos de comandos, ofreciendo las herramientas necesarias para ampliar el repertorio disponible de forma sencilla. Por otro lado, su distribución es sencilla y permite que se pueda iniciar el software, crear y configurar un bot en pocos minutos.

\section{Conclusiones}

\begin{enumerate}
	\item En este proyecto se proponía la simplificación de la creación de bots para la aplicación \textit{Discord}. Esta tarea, aunque puede parecer sencilla, una vez que se profundiza en ella no lo es. Sin embargo, proponer objetivos que pueden parecer muy exigentes y difíciles de alcanzar es una gran manera de forzar la búsqueda de la excelencia. Por tanto, con este proyecto se ha impulsado el interés por diseñar una estructura y arquitectura flexible, ambiciosa, y de gran potencial en este campo.
	
	\item Actualmente existen una cantidad de herramientas y recursos relacionados con los bots que es abrumadora, también existen para procesos como el despliegue del software. \textbf{Matroos} aúna todo ello en un mismo paquete, permitiendo crear y desplegar bots de una manera sencilla.

	\item \textit{Discord} es una herramienta que está creciendo mucho en muy poco tiempo. Sus usuarios requieren características concretas y complejas que la propia plataforma no ofrece. Software como el creado permite aunar de una forma sencilla muchos procesos internos, dejando disponibles a los usuarios aspectos como la ampliación del repertorio de comandos que permiten adaptar el software a cualquier situación.

	\item Crear bots desde cero implica contar con una serie de conocimientos tecnológicos de los que no todos los usuarios disponen. Herramientas como \textbf{Matroos}, que facilitan esta creación y configuración, son muy valoradas y necesitadas por estos usuarios. Uno de los objetivos era reducir el tiempo de creación y despliegue de un bot al menos  un 50\%, y se ha alcanzado con creces. El desarrollo de un bot de la forma tradicional se puede demorar hasta dos horas, mientras que usando \textbf{Matroos} el tiempo se reduce a no más de 10 minutos.

	\item Crear soluciones de \textit{software libre} es realmente beneficioso. La inmensa mayoría de librerías y recursos relacionados con el desarrollo de bots son de este tipo, y publicar cualquier contenido de esta manera no hace más que promover el crecimiento de las comunidades de desarrollo de software.
\end{enumerate}





\section{Trabajos futuros}

Hay algunos aspectos que podrían ser objetivo de mejora, son los siguientes:

\begin{itemize}
	\item Incrementar la cantidad de comandos predefinidos disponible. En este proyecto se han desarrollado cinco tipos de comandos disponibles para el uso de los usuarios. Si bien no es un mal comienzo ya que estos sirven como ejemplo, ampliar este repertorio es sin duda la principal rama de mejora del software.
	\item Interfaz de usuario. Aunque se ha intentado desarrollar una interfaz de usuario sencilla y fácil de utilizar, no es la más atractiva. Sin lugar a dudas es la parte principal que un usuario va a ver cuando utilice el software, por lo que su diseño y elementos visuales podrían mejorarse. Esto evitaría la pérdida de aquellos usuarios que principalmente se guían por el aspecto del software, en lugar de por las funcionalidades de este.
	\item Integración con otros sistemas y plataformas. Tanto \textit{Discord} como los frameworks de desarrollo de bots permiten la integración con redes sociales, videojuegos y otras plataformas. Sería interesante incluir utilidades internas en el software para facilitar esta conexión.
	\item \textit{Logs}. Estos son realmente importantes, ya que facilitan la trazabilidad de errores y problemas en un sistema. Aunque el software cuenta con un sistema de \textit{logging} básico, sería interesante ampliarlo para facilitar aún más la detección de errores.
\end{itemize}
