\chapter{Conclusiones y trabajos futuros}

Se pueden extraer una serie de conclusiones respecto todas las etapas de desarrollo de este proyecto, desde la definición los objetivos iniciales, hasta las etapas de investigación y creación del código del software. Siempre se ha tenido en cuenta la idea de simplificar los procesos existentes relacionados con la creación y uso de bots de \textit{Discord}, y de este aspecto es de donde más conclusiones se extraen.

El software creado cumple con creces los objetivos propuestos al inicio del proyecto. La herramienta simplifica ampliamente la creación de bots y comandos de \textit{Discord}, además de agilizar el proceso de configuración y despliegue de los bots. Además, sirve como plataforma de desarrollo de nuevos tipos de comandos, ofreciendo las herramientas necesarias para ampliar el repertorio disponible de forma sencilla. Por otro lado su distribución es sencilla y permite que se pueda iniciar el software, crear y configurar un bot en pocos minutos.

\section{Conclusiones}

\begin{itemize}
	\item Proponer objetivos que pueden parecer muy exigentes y difíciles de alcanzar es una gran manera de forzar la búsqueda de la excelencia. En el caso de este proyecto se quería simplificar la creación de bots de \textit{Discord}, una tarea que parece sencilla pero que una vez se profundiza en ella no lo es. Impulsa el interés por diseñar una estructura y arquitectura que permita ser lo más flexible posible en este campo.
	\item Actualmente existen una cantidad de herramientas y recursos relacionados con los bots que es abrumadora. En el caso de desarrollo de bots para \textit{Discord} las posibilidades son muy amplias, apareciendo cada día nuevos útiles y nuevos bots específicos que hacen que la comunidad crezca, teniendo que adaptarse lo actual a los nuevos cambios de una manera constante.
	\item \textit{Discord} es una herramienta que está creciendo mucho en muy poco tiempo. Sus usuarios requieren características concretas y complejas que la propia plataforma no ofrece, por lo que se usan bots. Toda aquella herramienta que permita facilitar la creación y configuración de estos es siempre bienvenida.
	\item Existen multitud de aplicaciones de mensajería que permiten la integración de bots. Aun así cuando se trata de funcionalidades muy concretas de estos bots, estos se encuentran en aplicaciones que tienen clientes de escritorio (como es el caso de \textit{Discord} o \textit{Slack}). También se tiene muy en cuenta por parte del usuario el precio a pagar por ellas, por lo que plataformas de pago suelen ser rechazadas por la vasta mayoría de usuarios.
	\item Crear bots desde cero implica contar con una serie de conocimientos tecnológicos de los que no todos los usuarios disponen. Herramientas que faciliten esta creación son muy valoradas y necesitadas por estos usuarios.
	\item Crear soluciones de \textit{software libre} es realmente beneficioso. La inmensa mayoría de librerías y recursos relacionados con el desarrollo de bots son de este tipo, y publicar cualquier contenido de esta manera no hace más que promover el crecimiento de las comunidades de desarrollo de software.
\end{itemize}





\section{Trabajos futuros}

Hay muchos aspectos que podrían ser objetivo de mejora.

\begin{itemize}
	\item Incrementar la cantidad de comandos disponible. En este proyecto se han desarrollado cinco tipos de comandos disponibles para el uso de los usuarios. Si bien no es un mal comienzo ya que estos sirven como ejemplo, ampliar este repertorio es sin duda la principal rama de mejora del software.
	\item Interfaz de usuario. Aunque se ha intentado desarrollar una interfaz de usuario sencilla y fácil de utilizar, no es la más atractiva. Sin lugar a dudas es la parte principal que un usuario va a ver cuando utilice el software, por lo que su diseño y elementos visuales podrían mejorarse. Esto evitaría la pérdida de aquellos usuarios que principalmente se guían por el aspecto del software, en lugar de por las funcionalidades de este.
	\item Integración con otros sistemas y plataformas. Tanto \textit{Discord} como los frameworks de desarrollo de bots permiten la integración con redes sociales, videojuegos y otras plataformas. Sería interesante incluir utilidades internas en el software para facilitar esta conexión.
	\item \textit{Logs}. Estos son realmente importantes, ya que facilitan la trazabilidad de errores y problemas en un sistema. Aunque el software cuenta con un sistema de \textit{logging} básico, sería interesante ampliarlo para facilitar aún más la detección de errores.
\end{itemize}
