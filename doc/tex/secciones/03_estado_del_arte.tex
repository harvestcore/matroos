\chapter{Estado del arte}

En este capítulo se hace un repaso de las distintas soluciones actuales que existen para la creación de bots de \textit{Discord}. Cada una de ellas destaca en aspectos concretos, pero ninguna reúne todas las características deseadas.

\section{Soluciones actuales}

En el ámbito de la creación de bots de \textit{Discord} existen una generosa cantidad de herramientas con este cometido. Como se comentaba en el capítulo anterior, estas herramientas en su basta mayoría son bastante básicas, permitiendo una creación y configuración de bots que en ciertos ámbitos puede ser insuficiente. Aún así, existen otras alternativas que son más interesantes y completas, pero que son complejas de utilizar.

Estos sistemas se podrían dividir en dos grupos, herramientas \textit{no-code} y herramientas que hacen uso de programación. En ambas la interacción con el sistema se hace a través de una aplicación web, y además suelen tener una apariencia muy similar. En las siguientes secciones se incluyen las herramientas con características más interesantes y las que dentro de lo que cabe son más completas.

\subsection{Herramientas \textit{no-code}}

Estas herramientas permiten la creación de bots sin hacer uso de recursos de programación o similares. En la mayoría de casos estas plataformas cuentan con un único bot que se debe agregar al servidor de \textit{Discord} deseado, y permiten la configuración de este bot de forma individual para cada servidor. En general la mayoría de herramientas de este tipo tienen una serie de funcionalidades gratuitas, teniendo que adquirir un plan de pago mensual para obtener funcionalidades extra.

En ellas se puede observar también uno de los principales problemas mencionados en el capítulo anterior, la muy reducida personalización y reutilización de los comandos. En estos sistemas no se puede crear un comando específico con una funcionalidad concreta, sino que se basan en funcionalidades predefinidas no reutilizables.

\subsubsection{ProBot}
\href{https://probot.io/}{\textit{ProBot}} es sin duda la más interesante de las herramientas no-code debido a que permite crear ilimitados comandos personalizados, siendo la principal desventaja que estos comandos son predefinidos, y no se puede cambiar su funcionalidad. Los comandos predefinidos se centran en moderación y mensajes automáticos, por lo que las posibilidades no son muy amplias.

A favor de esta herramienta también destaca que es sencilla de utilizar, la interfaz web intenta imitar a la de \textit{Discord} y es intuitiva. Por contra, es bastante intrusiva la modalidad \textit{premium}, ya que muchas secciones sugieren la compra de esta modalidad. Además es imposible controlar el despliegue del bot, y no es posible reutilizar comandos.

Sus características son:

\begin{table}[H]
    \centering
    \def\arraystretch{1.25}
    \begin{adjustbox}{max width=\textwidth}
    \begin{tabularx}{325px}{|l|L|}
    \hline
        \multicolumn{2}{|c|}{\textbf{\textit{ProBot}}} \\ \hline
    \hline
        \textbf{Tipo de comandos} & Predefinidos (ilimitados) \\ \hline
        \textbf{Comandos reutilizables} & No \\ \hline
        \textbf{Control del despliegue} & No \\ \hline
        \textbf{Número de bots} & 1, único \\ \hline
        \textbf{Experiencia} & Sencilla \\ \hline
        \textbf{Personalización extra} & Requiere \textit{premium} (mensualidades) \\ \hline
        \textbf{Características} & · Moderación\linebreak · Estadísticas\linebreak · Mensajes automáticos\linebreak · Música \\ \hline
        \textbf{Logs} & No \\ \hline
        \textbf{Premium} & \$60 al año \\ \hline
    \end{tabularx}
    \end{adjustbox}
    \caption{Características de \textit{ProBot}.}
\end{table}

\begin{figure}[H]
	\centering
	\includegraphics[width=1\textwidth]{img/probot.png}
	\caption{Interfaz web de \textit{ProBot}.}
\end{figure}

\subsubsection{Mee6}

\href{https://mee6.xyz/}{\textit{Mee6}} es otra herramienta muy similar a la anterior, siendo la principal diferencia que en este caso los comandos personalizados se limitan a 5. Por contra, tiene un mayor catálogo de funcionalidades.

De nuevo no es posible controlar el despliegue del bot, como tampoco es posible crear otro bot y agregarlo a un mismo servidor, o reutilizar comandos.

Sus características son:

\begin{table}[H]
    \centering
    \def\arraystretch{1.25}
    \begin{adjustbox}{max width=\textwidth}
    \begin{tabularx}{325px}{|l|L|}
    \hline
        \multicolumn{2}{|c|}{\textbf{\textit{Mee6}}} \\ \hline
    \hline
        \textbf{Tipo de comandos} & Predefinidos (muy limitados, 5) \\ \hline
        \textbf{Comandos reutilizables} & No \\ \hline
        \textbf{Control del despliegue} & No \\ \hline
        \textbf{Número de bots} & 1, único \\ \hline
        \textbf{Experiencia} & Sencilla \\ \hline
        \textbf{Personalización extra} & Requiere \textit{premium} (mensualidades) \\ \hline
        \textbf{Características} & · Moderación\linebreak · Estadísticas\linebreak · Mensajes automáticos\linebreak · Música\linebreak · Temporizadores\linebreak · \textit{Quiz} / \textit{Trivia} \\ \hline
        \textbf{Logs} & No \\ \hline
        \textbf{Premium} & \$50 al año / \$90 de por vida  \\ \hline
    \end{tabularx}
    \end{adjustbox}
    \caption{Características de \textit{Mee6}.}
\end{table}

\begin{figure}[H]
	\centering
	\includegraphics[width=1\textwidth]{img/mee6.png}
	\caption{Interfaz web de \textit{Mee6}.}
\end{figure}


\subsubsection{BotGhost}

\href{https://botghost.com/}{\textit{BotGhost}} es un híbrido entre \textit{ProBot} y \textit{Mee6}, ya que tiene las características comunes de ambos. La principal característica de esta herramienta es que permite crear comandos personalizados haciendo uso de una serie de módulos que se pueden interconectar para definir el ciclo de vida de un comando.

Esta característica es muy interesante, pero está muy limitada y las funcionalidades que permite realizar se resumen en envío de mensajes y tareas de moderación de usuarios muy básicas. El plan \textit{premium} sería necesario en este caso para poder sacarle partido a esta funcionalidad.

Otro aspecto interesante es que se pueden crear distintos bots, hasta 50 distintos si se opta por la opción \textit{premium}.

Sus características son:

\begin{table}[H]
    \centering
    \def\arraystretch{1.25}
    \begin{adjustbox}{max width=\textwidth}
    \begin{tabularx}{325px}{|l|L|}
    \hline
        \multicolumn{2}{|c|}{\textbf{\textit{BotGhost}}} \\ \hline
    \hline
        \textbf{Tipo de comandos} & Predefinidos (muy limitados, 5) \\ \hline
        \textbf{Comandos reutilizables} & Sí \\ \hline
        \textbf{Control del despliegue} & No (Sólo encendido y apagado) \\ \hline
        \textbf{Número de bots} & 1, único (50 con \textit{premium}) \\ \hline
        \textbf{Experiencia} & Compleja \\ \hline
        \textbf{Personalización extra} & Requiere \textit{premium} (mensualidades) \\ \hline
        \textbf{Características} & · Moderación\linebreak · Estadísticas\linebreak · Mensajes automáticos\linebreak · Temporizadores\linebreak · Integración con videojuegos\linebreak · Meteorología\linebreak · Música\linebreak · \textit{Quiz} / \textit{Trivia} \\ \hline
        \textbf{Logs} & No \\ \hline
        \textbf{Premium} & \$60 al año / \$100 de por vida \\ \hline
    \end{tabularx}
    \end{adjustbox}
    \caption{Características de \textit{BotGhost}.}
\end{table}

\begin{figure}[H]
	\centering
	\includegraphics[width=1\textwidth]{img/botghost.png}
	\caption{Interfaz web de \textit{BotGhost}.}
\end{figure}

\subsection{Herramientas de programación}

Actualmente existen multitud de librerías para distintos lenguajes de programación que permiten interactuar con la \textit{API} de \textit{Discord} y por tanto crear un bot. Así mismo existen herramientas híbridas que permiten esta creación de una manera más sencilla.

\subsubsection{Autocode}

\href{https://autocode.com/}{Autocode} es sin duda la herramienta mas interesante que se ha encontrado de esta modalidad híbrida. Realmente es una plataforma que permite la creación y despliegue de aplicaciones y servicios web, bots, y tareas de automatización haciendo uso de \textit{JavaScript} inyectado por los usuarios.

De este modo el usuario sólo tiene que preocuparse por el código de la aplicación (o bot en este caso) que quiere crear, ya que del despliegue se encarga \textit{Autocode}. En su plan gratuito se pueden crear hasta 50 aplicaciones distintas, y permite la integración entre si de los distintos recursos que el usuario crea en la plataforma.

Sus características son:

\begin{table}[H]
    \centering
    \def\arraystretch{1.25}
    \begin{adjustbox}{max width=\textwidth}
    \begin{tabularx}{325px}{|l|L|}
    \hline
        \multicolumn{2}{|c|}{\textbf{\textit{Autocode}}} \\ \hline
    \hline
        \textbf{Tipo de comandos} & Predefinidos + \textit{JS} \\ \hline
        \textbf{Comandos reutilizables} & No \\ \hline
        \textbf{Control del despliegue} & Sí (limitado) \\ \hline
        \textbf{Número de bots} & 50 gratis \\ \hline
        \textbf{Experiencia} & Algo complejo \\ \hline
        \textbf{Personalización extra} & Requiere \textit{premium} (mensualidades) \\ \hline
        \textbf{Especialidad} & Despliegue general de aplicaciones \\ \hline
        \textbf{Logs} & Sí (1-30 días) \\ \hline
        \textbf{Premium} & \$180 / \$1620 al año \\ \hline
    \end{tabularx}
    \end{adjustbox}
    \caption{Resumen de soluciones actuales.}
\end{table}

\begin{figure}[H]
	\centering
	\includegraphics[width=1\textwidth]{img/autocode.png}
	\caption{Interfaz web de \textit{Autocode}.}
\end{figure}

\subsubsection{Librerías de programación}

Las librerías de programación dan libertad a la hora de crear un bot de \textit{Discord}, lo cual puede ser ideal en algunos casos. Las ventajas son obvias, ya que se puede crear cualquier tipo de comando y la reutilización es sencilla, pero en cambio, la gestión del despliegue puede ser compleja.

Por lo general todas las librerías permiten realizar casi las mismas funcionalidades, diferenciándose en aspectos como el rendimiento, la comunidad que las soporta o la facilidad de uso.

Algunos ejemplos de librerías son:

\begin{itemize}
	\item \textbf{\textit{C\#}}: \href{https://discordnet.dev/}{\textit{Discord.NET}}, \href{https://github.com/DSharpPlus/DSharpPlus}{\textit{DSharpPlus}}
	\item \textbf{\textit{Java}}: \href{https://github.com/DV8FromTheWorld/JDA}{\textit{JDA}}, \href{https://discord4j.com/}{\textit{Discord4J}}
	\item \textbf{\textit{C++}}: \href{https://dpp.dev/}{\textit{D++}}
	\item \textbf{\textit{JavaScript}}: \href{https://discord.js.org/}{\textit{discord.js}}
	\item \textbf{\textit{Golang}}: \href{https://github.com/bwmarrin/discordgo}{\textit{DiscordGo}}
	\item \textbf{\textit{Ruby}}: \href{https://github.com/shardlab/discordrb}{\textit{discordrb}}
\end{itemize}


\subsection{Comparativa de tiempos}

En esta sección se hace una comparativa del tiempo medio de desarrollo desde cero de un bot de \textit{Discord} usando las herramientas anterior mencionadas. Además se incluye tiempos de desarrollo usando tres lenguajes de programación: \textit{C\#}, \textit{JavaScript} y \textit{Python}.

Las mediciones incluyen todos los pasos necesarios para crear uno de estos bots con dos comandos personalizados. En el caso de las herramientas de programación se incluye desde la creación del proyecto hasta el despliegue (en local) de este.

Los dos comandos personalizados se han elegido al ser comunes en todas las plataformas mencionadas, además de sencillos de implementar. Son los siguientes:

\begin{itemize}
	\item Envío de un mensaje.
	\item Envío de un mensaje recurrente.
\end{itemize}

\begin{table}[H]
    \centering
    \def\arraystretch{1.25}
    \begin{adjustbox}{max width=\textwidth}
    \begin{tabularx}{200px}{|l|R|}
    \hline
        \textbf{Herramienta} & \textbf{Tiempo (en minutos)} \\ \hline
    \hline
        ProBot & 5 \\ \hline
        Mee6 & 5 \\ \hline
        BotGhost & 10 \\ \hline
    \hline
        Autocode (JS) & 45 \\ \hline
    \hline
        JS & 85 \\ \hline
        C\# & 100 \\ \hline
        Python & 80 \\ \hline
    \end{tabularx}
    \end{adjustbox}
    \caption{Comparativa de tiempos}
\end{table}

\section{Discusión}

Como se puede observar existen multitud de posibilidades a la hora de crear un bot de \textit{Discord}, y aunque cumplen lo que prometen, se centran en aspectos muy concretos dejando otros bastante desatendidos.

En las herramientas \textit{no-code} los bots se centran principalmente en tareas de moderación, envío de mensajes, estadísticas e integración con videojuegos y redes sociales. Se enfocan también en los paquetes \textit{premium}, dejando de lado detalles específicos (y que serían ideales) como:

\begin{itemize}
	\item Reutilización de comandos.
	\item No es posible crear comandos con funcionalidad específica.
	\item No se tiene control del despliegue de los bots.
	\item No se pueden crear distintos bots.
\end{itemize}

En el caso de las librerías de programación, aunque todas permiten el acceso a la \textit{API} de \textit{Discord}, cada una de ellas tiene una estructura distinta y los procedimientos para crear un bot o comandos son más o menos complejos. Si un usuario decidiese utilizarlas tendría flexibilidad completa a la hora de crear una estructura concreta, pero entonces tendría que dedicar en ese caso un tiempo necesario para diseñar algo funcional.

\textit{Autocode} es una buena alternativa a las soluciones anteriores, ya que se evita el tener que gestionar el despliegue de los bots y se eliminan algunas trabas de gestión del código, pero al igual que el uso de librerías toda la lógica recae en el usuario final. Esto puede ser útil en ciertos casos, pero no siempre.

En la comparativa de tiempos anterior se puede observar que las herramientas \textit{no-code} son las más rápidas. Esto se debe a que solo es necesario agregar el bot al servidor de \textit{Discord} deseado, y tras eso configurar de manera sencilla los comandos.

En menos de 10 minutos se puede incluir una gran cantidad de funcionalidad a un servidor de manera gratuita, algo que puede ser muy útil para la basta mayoría de usuarios de \textit{Discord}, pero cuando se necesitan funcionalidades específicas entonces no son las ideales.

En cambio, cuando se usan herramientas que hacen uso de código, el tiempo de implementación se incrementa considerablemente. Debido a las características de lenguajes como \textit{JavaScript} o \textit{Python} este tiempo es algo menor que en \textit{C\#}, ya que este suele tener una curva de aprendizaje mayor. Aún así, teniendo en cuenta que hay que hacer ciertas consultas a la documentación de estas librerías y que hay que crear una estructura y una serie de recursos para poder comenzar a desarrollar el bot, el tiempo se incrementa.

En definitiva, no existe ninguna herramienta que brinde lo mejor de ambas alternativas. Por un lado se quiere facilitar la creación de bots y comandos, y el despliegue de estos. Por otro se quiere poder ampliar el repertorio de comandos disponible de manera sencilla, sin tener que desarrollar una aplicación completa para ello.
