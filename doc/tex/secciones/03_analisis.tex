\chapter{Análisis}

En este capítulo se profundiza en el análisis del problema planteado en capítulos anteriores, describiendo los diferentes actores, casos de uso, historias de usuario y los \textit{user journeys} asociados a estas HU.

\section{Actores}

En el sistema hay un único actor, es el siguiente:

\begin{table}[H]
    \centering
    \def\arraystretch{1.25}
    \begin{adjustbox}{max width=\textwidth}
    \begin{tabularx}{\textwidth}{|l|L|}
    \hline
        \textbf{Actor} & \textbf{Administrador} \\ \hline
    \hline
        Descripción & Crea los bots y los comandos. Tiene poder de configuración y de despliegue. Puede ser también usuario del bot haciendo uso de sus comandos. Agrega los bots a los servidores de \textit{Discord}. \\ \hline
        Beneficio & Obtiene un software que le permite crear bots y comandos de forma sencilla sin tener que usar herramientas de programación, a la vez que puede ampliar los comandos disponibles usando estas herramientas. \\ \hline
    \end{tabularx}
    \end{adjustbox}
    \caption{Actor 1. Administrador.}
\end{table}



\section{Casos de uso}

En esta sección se presentan de manera detallada los diferentes casos de uso.

\begin{table}[H]
    \centering
    \def\arraystretch{1.25}
    \begin{adjustbox}{max width=\textwidth}
    \begin{tabularx}{\textwidth}{|l|L|}
    \hline
        \textbf{Caso de uso} & \textbf{01 - Crear un bot} \\ \hline
    \hline
        Tipo & Primario \\ \hline
        Propósito & Crear un bot para después configurarlos y desplegarlos en distintos \textit{workers}. \\ \hline
        Referencias & \hyperref[sec:hu01]{HU-01} \\ \hline
        Precondición & El usuario provee al sistema de los datos necesarios para crear el bot. El sistema se encuentra disponible, instalado en una máquina administrada por el usuario.\\ \hline
        Postcondición & El bot es creado y queda disponible para ser configurado y desplegado. \\ \hline
        Comentarios adicionales & El usuario ha instalado previamente 
    \end{tabularx}
    \end{adjustbox}
    \caption{Caso de uso 01. Crear un bot.}
\end{table}

\begin{table}[H]
    \centering
    \def\arraystretch{1.25}
    \begin{adjustbox}{max width=\textwidth}
    \begin{tabularx}{\textwidth}{|l|L|}
    \hline
        \textbf{Caso de uso} & \textbf{02 - Editar un bot} \\ \hline
    \hline
        Tipo & Primario \\ \hline
        Propósito & Editar los parámetros y comandos de un bot. \\ \hline
        Referencias & \hyperref[sec:hu03]{HU-03}\\ \hline
        Precondición & El usuario provee al sistema de los nuevos parámetros del bot, o de los comandos que quiere modificar. \\ \hline
        Postcondición & El bot queda modificado. \\ \hline
    \end{tabularx}
    \end{adjustbox}
    \caption{Caso de uso 02. Editar un bot.}
\end{table}

\begin{table}[H]
    \centering
    \def\arraystretch{1.25}
    \begin{adjustbox}{max width=\textwidth}
    \begin{tabularx}{\textwidth}{|l|L|}
    \hline
        \textbf{Caso de uso} & \textbf{03 - Eliminar un bot} \\ \hline
    \hline
        Tipo & Primario \\ \hline
        Propósito & Eliminar un bot cuando no es necesario. \\ \hline
        Referencias & \hyperref[sec:hu04]{HU-04} \\ \hline
        Precondición & El usuario provee al sistema de los datos necesarios del bot que quiere eliminar. \\ \hline
        Postcondición & El bot queda eliminado. \\ \hline
    \end{tabularx}
    \end{adjustbox}
    \caption{Caso de uso 03. Eliminar un bot.}
\end{table}

\begin{table}[H]
    \centering
    \def\arraystretch{1.25}
    \begin{adjustbox}{max width=\textwidth}
    \begin{tabularx}{\textwidth}{|l|L|}
    \hline
        \textbf{Caso de uso} & \textbf{04 - Crear un comando} \\ \hline
    \hline
        Tipo & Primario \\ \hline
        Propósito & Crear un comando para después configurarlo en un bot. \\ \hline
        Referencias & \hyperref[sec:hu05]{HU-05} \\ \hline
        Precondición & El usuario provee al sistema de los datos necesarios para crear el comando. \\ \hline
        Postcondición & El comando es creado y queda disponible para ser configurado en bots. \\ \hline
    \end{tabularx}
    \end{adjustbox}
    \caption{Caso de uso 04. Crear un comando.}
\end{table}

\begin{table}[H]
    \centering
    \def\arraystretch{1.25}
    \begin{adjustbox}{max width=\textwidth}
    \begin{tabularx}{\textwidth}{|l|L|}
    \hline
        \textbf{Caso de uso} & \textbf{05 - Editar un comando} \\ \hline
    \hline
        Tipo & Primario \\ \hline
        Propósito & Editar los parámetros de un comando. \\ \hline
        Referencias & \hyperref[sec:hu07]{HU-07} \\ \hline
        Precondición & El usuario provee al sistema de los nuevos parámetros del bot, o de los comandos que quiere modificar. \\ \hline
        Postcondición & El bot queda modificado. \\ \hline
    \end{tabularx}
    \end{adjustbox}
    \caption{Caso de uso 05. Editar un comando.}
\end{table}

\begin{table}[H]
    \centering
    \def\arraystretch{1.25}
    \begin{adjustbox}{max width=\textwidth}
    \begin{tabularx}{\textwidth}{|l|L|}
    \hline
        \textbf{Caso de uso} & \textbf{06 - Eliminar un comando} \\ \hline
    \hline
        Tipo & Primario \\ \hline
        Propósito & Eliminar un comando cuando no es necesario. \\ \hline
        Referencias & \hyperref[sec:hu08]{HU-08} \\ \hline
        Precondición & El usuario provee al sistema de los datos necesarios del comando que quiere eliminar. \\ \hline
        Postcondición & El comando queda eliminado. \\ \hline
    \end{tabularx}
    \end{adjustbox}
    \caption{Caso de uso 06. Eliminar un comando.}
\end{table}

\begin{table}[H]
    \centering
    \def\arraystretch{1.25}
    \begin{adjustbox}{max width=\textwidth}
    \begin{tabularx}{\textwidth}{|l|L|}
    \hline
        \textbf{Caso de uso} & \textbf{07 - Desplegar un bot} \\ \hline
    \hline
        Tipo & Primario \\ \hline
        Propósito & Desplegar un bot para que esté disponible en los servidores de \textit{Discord}. \\ \hline
        Referencias & \hyperref[sec:hu10]{HU-10} \\ \hline
        Precondición & El usuario provee al sistema de los datos necesarios del bot que quiere desplegar. Al menos un \textit{worker} se encuentra activo. \\ \hline
        Postcondición & El bot es desplegado en uno de los workers y queda disponible en los servidores de \textit{Discord}. \\ \hline
    \end{tabularx}
    \end{adjustbox}
    \caption{Caso de uso 07. Desplegar un bot.}
\end{table}

\begin{table}[H]
    \centering
    \def\arraystretch{1.25}
    \begin{adjustbox}{max width=\textwidth}
    \begin{tabularx}{\textwidth}{|l|L|}
    \hline
        \textbf{Caso de uso} & \textbf{08 - Cancelar el despliegue de un bot} \\ \hline
    \hline
        Tipo & Primario \\ \hline
        Propósito & Parar la ejecución de un bot para que deje de estar disponible en los servidores de \textit{Discord}. \\ \hline
        Referencias & \hyperref[sec:hu11]{HU-11} \\ \hline
        Precondición & El usuario provee al sistema de los datos necesarios del bot del que quiere cancelar el despliegue. \\ \hline
        Postcondición & El bot deja de ejecutarse y deja de estar disponible en los servidores de \textit{Discord}. \\ \hline
    \end{tabularx}
    \end{adjustbox}
    \caption{Caso de uso 08. Cancelar el despliegue de un bot.}
\end{table}

\begin{table}[H]
    \centering
    \def\arraystretch{1.25}
    \begin{adjustbox}{max width=\textwidth}
    \begin{tabularx}{\textwidth}{|l|L|}
    \hline
        \textbf{Caso de uso} & \textbf{09 - Interfaz de usuario} \\ \hline
    \hline
        Tipo & Primario \\ \hline
        Propósito & Realizar las tareas de gestión de bots y comandos mediante una interfaz de usuario. \\ \hline
        Referencias & \hyperref[sec:hu12]{HU-12} \\ \hline
        Precondición & El sistema provee a la interfaz de usuario de toda la información referente a los bots y comandos. \\ \hline
        Postcondición & El usuario realiza las tareas de gestión de estos bots y comandos. \\ \hline
    \end{tabularx}
    \end{adjustbox}
    \caption{Caso de uso 09. Cancelar el despliegue de un bot.}
\end{table}


\section{Personas}

\subsection{Administrador}
\label{sec:personaAdmin}

\begin{table}[H]
    \centering
    \def\arraystretch{1.25}
    \begin{adjustbox}{max width=\textwidth}
    \begin{tabularx}{\textwidth}{|l|L|}
    \hline
        \textbf{Nombre} & \textbf{David Infante} \\ \hline
    \hline
        Rol & Administrador \\ \hline
        Descripción & · 24 años.\linebreak · Disfruta de las tardes con sus amigos en \textit{Discord} jugando a sus videojuegos favoritos. \\ \hline
        Intereses & · \textit{Discord}, ya que le parece una herramienta muy potente.\linebreak · Videojuegos, le encantan los \textit{shooters}.\linebreak · Automatización de tareas repetitivas, ya que odia hacer lo mismo continuamente.\linebreak · Programación, ya que disfruta creando software para facilitar su día a día.\linebreak · Monitorización, le gusta saber que todo el software que despliega funciona correctamente.\linebreak · Creación de servidores de juegos, para jugar con sus amigos y no tener que depender de servidores de terceros. \\ \hline
        Formación & Ingeniero informático.\linebreak\linebreak Tiene conocimientos avanzados en:\linebreak · Configuración de \textit{Discord}.\linebreak · Administración de sistemas.\linebreak · Despliegue de software y sistemas. \\ \hline
        Frustraciones & Tener que realizar tareas repetitivas. \\ \hline
        Necesidades & Un software o sistema que le permita programar las tareas repetitivas de monitorización y automatización que tanto odia. Usa mucho \textit{Discord}, por lo que piensa que sería útil que el software estuviera integrado con esa herramienta. \\ \hline
    \end{tabularx}
    \end{adjustbox}
    \caption{Persona 1. Administrador.}
\end{table}


\subsection{Usuario de \textit{Discord}}
\label{sec:personaUsuarioDiscord}
\begin{table}[H]
    \centering
    \def\arraystretch{1.25}
    \begin{adjustbox}{max width=\textwidth}
    \begin{tabularx}{\textwidth}{|l|L|}
    \hline
        \textbf{Nombre} & \textbf{Jorge Pulido} \\ \hline
    \hline
        Rol & Usuario de \textit{Discord} \\ \hline
        Descripción & · 25 años.\linebreak · Amigo de Jorge Cancho. No tiene conocimientos de programación ni de temas relacionados con la ingeniería o la informática. Le gusta disfrutar de las tardes con sus amigos en \textit{Discord}. \\ \hline
        Intereses & · Videojuegos, dedica la mayor parte de su tiempo a jugar con sus amigos.\linebreak · La facilidad de las cosas, no le gusta complicarse la vida.\linebreak · \textit{Discord}, le parece una herramienta muy útil, ya que la usa con sus amigos y para temas laborales. \\ \hline
        Formación & Magisterio de educación primaria. \\ \hline
        Frustraciones & No le gusta nada tener que indagar en detalles técnicos al jugar a videojuegos con sus amigos. Entiende que en ocasiones es necesaria alguna configuración para poder jugar (como acceder a un servidor), pero quiere que ese proceso sea lo más fácil posible. No le gusta tener que recordar esos detalles. \\ \hline
        Necesidades & Una herramienta que le permita acceder a esos detalles sin preocuparse de recordarlos o de consultarlos de manera extraña. \\ \hline
    \end{tabularx}
    \end{adjustbox}
    \caption{Persona 2. Usuario de \textit{Discord}.}
\end{table}

\subsection{Miembro del tribunal}
\label{sec:personaMiembroTribunal}
\begin{table}[H]
    \centering
    \def\arraystretch{1.25}
    \begin{adjustbox}{max width=\textwidth}
    \begin{tabularx}{\textwidth}{|l|L|}
    \hline
        \textbf{Nombre} & \textbf{Blanca Casado} \\ \hline
    \hline
        Rol & Miembro del tribunal \\ \hline
        Descripción & · 52 años.\linebreak · Su conocimiento en informática es muy elevado, pero no tiene tanta destreza con las distintas aplicaciones de mensajería instantánea que han surgido en los últimos años. \\ \hline
        Intereses & · Procesamiento en segundo plano.\linebreak · Redes neuronales.\linebreak · Desarrollo ágil. \\ \hline
        Formación & Catedrática en informática. \\ \hline
        Frustraciones & No le gusta enfrentarse a documentaciones poco precisas o de dudosa credibilidad. \\ \hline
        Necesidades & Una documentación y una presentación acorde a los criterios de evaluación de TFM que le permita evaluar al estudiante. \\ \hline
    \end{tabularx}
    \end{adjustbox}
    \caption{Persona 3. Miembro del tribunal.}
\end{table}

\section{Historias de usuario}

Para la creación de las historias de usuarios se ha usado la siguiente estructura.

\begin{table}[H]
    \centering
    \def\arraystretch{1.25}
    \begin{adjustbox}{max width=\textwidth}
    \begin{tabularx}{\textwidth}{|l|L|}
    \hline
        \textbf{Sección} & \textbf{Significado} \\ \hline
    \hline
        Resumen & Breve resumen de la historia de usuario. \\ \hline
        Meta & Qué se quiere conseguir. \\ \hline
        Beneficio & El beneficio de la historia de usuario. \\ \hline
        Perfil de usuario & Perfil del usuario que genera la historia de usuario. \\ \hline
        Escenario & Escenario de la historia de usuario. Se deben especificar detalles más concretos.\linebreak · Dado …\linebreak · Cuando …\linebreak · Entonces … \\ \hline
        Notas funcionales & Notas adicionales de carácter funcional que ayudan a comprender mejor el alcance de la historia de usuario. \\ \hline
        Notas técnicas & Notas adicionales de carácter técnico que ayudan a comprender mejor este tipo de detalles a la hora de desarrollar la historia de usuario. \\ \hline
        Dependencias & Posibles dependencias que tenga la historia de usuario. Éstas pueden ser otras historias de usuario, tareas que se estén llevando a cabo, etc. \\ \hline
        Tareas de seguimiento & Una vez analizada la historia de usuario, las tareas que se deben realizar a continuación. \\ \hline
        Criterio de aceptación & Criterio por el cual se va a determinar que la historia de usuario ha sido completada con éxito y por tanto finalizada. \\ \hline
    \end{tabularx}
    \end{adjustbox}
    \caption{Resumen historias de usuario.}
\end{table}

\bigskip

Por otro lado, se han creado las siguientes historias de usuario. Estas se encuentran en el \href{https://github.com/harvestcore/matroos}{repositorio} de \textit{GitHub} del proyecto, en la sección \href{https://github.com/harvestcore/matroos/labels/US}{\textit{Issues}}.

\begin{enumerate}
	\item \href{https://github.com/harvestcore/matroos/issues/1}{Crear diferentes bots de \textit{Discord}}
	\item \href{https://github.com/harvestcore/matroos/issues/2}{Consultar datos de un bot}
	\item \href{https://github.com/harvestcore/matroos/issues/3}{Editar un bot}
	\item \href{https://github.com/harvestcore/matroos/issues/4}{Eliminar un bot}
	\item \href{https://github.com/harvestcore/matroos/issues/5}{Crear diferentes comandos de \textit{Discord}}
	\item \href{https://github.com/harvestcore/matroos/issues/6}{Consultar datos de un comando}
	\item \href{https://github.com/harvestcore/matroos/issues/7}{Editar un comando}
	\item \href{https://github.com/harvestcore/matroos/issues/8}{Eliminar un comando}
	\item \href{https://github.com/harvestcore/matroos/issues/9}{Lanzar bots}
	\item \href{https://github.com/harvestcore/matroos/issues/10}{Cancelar ejecución de bots}
	\item \href{https://github.com/harvestcore/matroos/issues/11}{Consultar estado de los despliegues}
	\item \href{https://github.com/harvestcore/matroos/issues/25}{Interfaz de usuario}
	\item \href{https://github.com/harvestcore/matroos/issues/39}{Criterios de evaluación}
\end{enumerate}


\subsection{HU-01 - Crear diferentes bots de \textit{Discord}}
\label{sec:hu01}

\begin{table}[H]
    \centering
    \def\arraystretch{1.25}
    \begin{adjustbox}{max width=\textwidth}
    \begin{tabularx}{\textwidth}{|l|L|}
    \hline
        \textbf{Sección} & \textbf{Contenido} \\ \hline
    \hline
        Resumen & Como usuario administrador quiero crear distintos bots de \textit{Discord} en el sistema para poder configurarlos con los comandos que yo quiera para que Estos realicen las tareas que deseo al ejecutar los comandos en los canales donde he agregado los bots. \\ \hline
        Meta & Creación de bots para que más tarde puedan ser configurados y desplegados y para que puedan ejecutar los comandos configurados en ellos. \\ \hline
        Beneficio & De este modo, cada bot puede tener una funcionalidad específica configurada, sin tener que compartir un bot con funcionalidades de distintos ámbitos. \\ \hline
        Perfil de usuario & \hyperref[sec:personaAdmin]{Administrador} \\ \hline
        Escenario & · Dado: que quiero crear un bot de \textit{Discord} y configurar sus funcionalidades,\linebreak · Cuando: proveo al sistema de una \textit{key} de bot de \textit{Discord} y de un nombre para el bot,\linebreak · Entonces: el sistema crea un bot para que pueda empezar a configurarlo. \\ \hline
        Notas funcionales & · No se debe permitir la creación de bots con el mismo nombre.\linebreak · No se debe permitir la creación de bots con la misma \textit{key}. \\ \hline
        Notas técnicas & Elementos necesarios para la creación del bot:\linebreak · \verb|key: str|. Key del bot, proporcionada en el panel de control de \textit{Discord}.\linebreak · \verb|name: str|. Nombre del bot.\linebreak \linebreak Parámetros extra necesarios para crear un bot:\linebreak · \verb|id: Guid|. Identificador único para cada bot. \\ \hline
        Dependencias & – \\ \hline
        Tareas de seguimiento & – \\ \hline
        Criterio de aceptación & · Los bots se pueden crear correctamente.\linebreak · El usuario es avisado en caso de error al crear un bot.\linebreak · Tests unitarios y integración son creados dentro de lo posible. \\ \hline
    \end{tabularx}
    \end{adjustbox}
    \caption{HU-01. Crear diferentes bots de \textit{Discord}.}
\end{table}


\subsection{HU-02 - Consultar datos de un bot}
\label{sec:hu02}

\begin{table}[H]
    \centering
    \def\arraystretch{1.25}
    \begin{adjustbox}{max width=\textwidth}
    \begin{tabularx}{\textwidth}{|l|L|}
    \hline
        \textbf{Sección} & \textbf{Contenido} \\ \hline
    \hline
        Resumen & Como usuario administrador quiero consultar los detalles de los bots de \textit{Discord} creados en el sistema, para poder ver sus características y su configuración. \\ \hline
        Meta & Obtener todos los datos de un bot en concreto o de todos los bots creados en el sistema para consultar sus características y configuración. \\ \hline
        Beneficio & Consultar la configuración única y específica de cada uno de los bots. \\ \hline
        Perfil de usuario & \hyperref[sec:personaAdmin]{Administrador} \\ \hline
        Escenario & · Dado: que quiero consultar los detalles de los bots creados en el sistema,\linebreak · Cuando: hago una petición al sistema para ello,\linebreak · Entonces: el sistema me devuelve los detalles y datos de los bots. \\ \hline
        Notas funcionales & · Se debe permitir obtener los detalles de todos los bots.\linebreak · Se debe permitir obtener los detalles de un bot en específico. \\ \hline
        Notas técnicas & Parámetros necesarios para obtener los detalles de un bot específico:\linebreak · \verb|id: Guid|. Identificador único para cada bot. \\ \hline
        Dependencias & – \\ \hline
        Tareas de seguimiento & – \\ \hline
        Criterio de aceptación & · Todos los datos de los bots son devueltos.\linebreak · El usuario es avisado en caso de error al obtener los detalles de los bots.\linebreak · Tests unitarios y integración son creados dentro de lo posible. \\ \hline
    \end{tabularx}
    \end{adjustbox}
    \caption{HU-02. Consultar datos de un bot.}
\end{table}

\subsection{HU-03 - Editar un bot}
\label{sec:hu03}

\begin{table}[H]
    \centering
    \def\arraystretch{1.25}
    \begin{adjustbox}{max width=\textwidth}
    \begin{tabularx}{\textwidth}{|l|L|}
    \hline
        \textbf{Sección} & \textbf{Contenido} \\ \hline
    \hline
        Resumen & Como usuario administrador quiero poder editar los detalles de un bot de \textit{Discord}, para configurarle comandos nuevos, eliminar existentes o cambiar sus parámetros. \\ \hline
        Meta & Permitir la edición de los parámetros de los bots (comandos, nombre, key, etc). \\ \hline
        Beneficio & Ampliar, reducir o modificar las funcionalidades específicas de los bots. \\ \hline
        Perfil de usuario & \hyperref[sec:personaAdmin]{Administrador} \\ \hline
        Escenario & · Dado: que quiero modificar los datos de un bot,\linebreak · Cuando: proveo al sistema del identificador del bot a editar y los datos que se deben modificar,\linebreak · Entonces: el sistema modifica los datos del bot. \\ \hline
        Notas funcionales & · No se debe permitir la existencia de bots con el mismo nombre.\linebreak · No se debe permitir la existencia de bots con la misma \textit{key}.\linebreak · En caso de agregar un comando a un bot, el comando debe estar creado previamente en el sistema.\linebreak · No se debe permitir la existencia comandos iguales en un mismo bot.\linebreak · En caso de que el bot se encuentre desplegado, debe reiniciarse el despliegue una vez la edición finaliza. \\ \hline
        Notas técnicas & Parámetros necesarios para obtener los detalles de un bot específico y para modificarlo:\linebreak · \verb|id: Guid|. Identificador único para cada bot.\linebreak · key: str. Key del bot, proporcionada en el panel de control de \textit{Discord}.\linebreak · \verb|name: str|. Nombre del bot.\linebreak · \verb|command_id: Guid|. Identificador único para cada comando. \\ \hline
        Dependencias & \hyperref[sec:hu01]{HU-01} \\ \hline
        Tareas de seguimiento & – \\ \hline
        Criterio de aceptación & · El bot es modificado correctamente.\linebreak · El usuario es avisado en caso de error al modificar los datos del bot.\linebreak · Tests unitarios y integración son creados dentro de lo posible. \\ \hline
    \end{tabularx}
    \end{adjustbox}
    \caption{HU-03. Editar un bot.}
\end{table}

\subsection{HU-04 - Eliminar un bot}
\label{sec:hu04}

\begin{table}[H]
    \centering
    \def\arraystretch{1.25}
    \begin{adjustbox}{max width=\textwidth}
    \begin{tabularx}{\textwidth}{|l|L|}
    \hline
        \textbf{Sección} & \textbf{Contenido} \\ \hline
    \hline
        Resumen & Como usuario administrador quiero poder eliminar un bot de \textit{Discord}. \\ \hline
        Meta & Permitir el borrado de bots. \\ \hline
        Beneficio & Cuando ya no es necesario un bot, se puede eliminar. \\ \hline
        Perfil de usuario & \hyperref[sec:personaAdmin]{Administrador} \\ \hline
        Escenario & · Dado: que quiero eliminar un bot,\linebreak · Cuando: proveo al sistema del identificador del bot a eliminar,\linebreak · Entonces: el sistema elimina el bot y sus datos asociados. \\ \hline
        Notas funcionales & ·  En caso de que el bot se encuentre desplegado, éste despliegue debe cancelarse. \\ \hline
        Notas técnicas & Parámetros necesarios para eliminar un bot:\linebreak · \verb|id: Guid|. Identificador único para cada bot. \\ \hline
        Dependencias & \hyperref[sec:hu01]{HU-01} \\ \hline
        Tareas de seguimiento & – \\ \hline
        Criterio de aceptación & · El bot es eliminado correctamente.\linebreak · El usuario es avisado en caso de error al eliminar los datos del bot.\linebreak · Tests unitarios y integración son creados dentro de lo posible. \\ \hline
    \end{tabularx}
    \end{adjustbox}
    \caption{HU-04. Eliminar un bot.}
\end{table}


\subsection{HU-05 - Crear diferentes comandos de \textit{Discord}}
\label{sec:hu05}

\begin{table}[H]
    \centering
    \def\arraystretch{1.25}
    \begin{adjustbox}{max width=\textwidth}
    \begin{tabularx}{\textwidth}{|l|L|}
    \hline
        \textbf{Sección} & \textbf{Contenido} \\ \hline
    \hline
        Resumen & Como usuario administrador quiero crear distintos comandos para los bots de \textit{Discord} para que Estos realicen las tareas que deseo tras ejecutarlos en los canales de \textit{Discord}. \\ \hline
        Meta & Creación de distintos comandos que posteriormente puedan ser asignados a bots. \\ \hline
        Beneficio & De este modo, cada comando puede tener una tarea específica configurada y accesible (y ejecutable) dentro de un servidor de \textit{Discord}. \\ \hline
        Perfil de usuario & \hyperref[sec:personaAdmin]{Administrador} \\ \hline
        Escenario & · Dado: que quiero crear un comando de \textit{Discord} y configurar sus funcionalidades,\linebreak · Cuando: proveo al sistema de un nombre para el comando y de un prefijo,\linebreak · Entonces: el sistema crea un comando para que pueda empezar a configurarlo. \\ \hline
        Notas funcionales & · No se debe permitir la creación de comandos con el mismo nombre. \\ \hline
        Notas técnicas & Elementos necesarios para la creación del comando:\linebreak · \verb|name: str|. Nombre del comando.\linebreak · \verb|prefix: str|. Prefijo del comando.\linebreak · \verb|type: CommandType|. El tipo de comando.\linebreak · \verb|params: obj|. Parámetros del comando. \\ \hline
        Dependencias & – \\ \hline
        Tareas de seguimiento & – \\ \hline
        Criterio de aceptación & · Los comandos se crean correctamente.\linebreak · El usuario es avisado en caso de error al crear un comando.\linebreak · Tests unitarios y integración son creados dentro de lo posible. \\ \hline
    \end{tabularx}
    \end{adjustbox}
    \caption{HU-05. Crear diferentes comandos de \textit{Discord}.}
\end{table}

\subsection{HU-06 - Consultar datos de un comando}
\label{sec:hu06}

\begin{table}[H]
    \centering
    \def\arraystretch{1.25}
    \begin{adjustbox}{max width=\textwidth}
    \begin{tabularx}{\textwidth}{|l|L|}
    \hline
        \textbf{Sección} & \textbf{Contenido} \\ \hline
    \hline
        Resumen & Como usuario administrador quiero consultar los detalles de los comandos creados en el sistema. \\ \hline
        Meta & Obtener todos los datos de un comando en concreto o de todos los comandos creados en el sistema. \\ \hline
        Beneficio & Consultar la configuración única y específica de cada uno de los comandos. \\ \hline
        Perfil de usuario & \hyperref[sec:personaAdmin]{Administrador} \\ \hline
        Escenario & · Dado: que quiero consultar los detalles de los comandos creados en el sistema,\linebreak · Cuando: hago una petición al sistema para ello,\linebreak · Entonces: el sistema me devuelve los datos de los comandos. \\ \hline
        Notas funcionales & · Se debe permitir obtener los detalles de todos los comandos.\linebreak · Se debe permitir obtener los detalles de un comando en específico. \\ \hline
        Notas técnicas & Parámetros necesarios para obtener los detalles de un comando específico:\linebreak · \verb|id: Guid|. Identificador único para cada comando. \\ \hline
        Dependencias & \hyperref[sec:hu05]{HU-05} \\ \hline
        Tareas de seguimiento & – \\ \hline
        Criterio de aceptación & · Todos los datos de los comandos son devueltos.\linebreak · El usuario es avisado en caso de error al obtener los detalles de los comandos.\linebreak · Tests unitarios y integración son creados dentro de lo posible. \\ \hline
    \end{tabularx}
    \end{adjustbox}
    \caption{HU-06. Consultar datos de un comando.}
\end{table}

\subsection{HU-07 - Editar un comando}
\label{sec:hu07}

\begin{table}[H]
    \centering
    \def\arraystretch{1.25}
    \begin{adjustbox}{max width=\textwidth}
    \begin{tabularx}{\textwidth}{|l|L|}
    \hline
        \textbf{Sección} & \textbf{Contenido} \\ \hline
    \hline
        Resumen & Como usuario administrador quiero poder editar los detalles de un comando para . \\ \hline
        Meta & La edición de los parámetros de los comandos (nombre, prefijo, parámetros, etc). \\ \hline
        Beneficio & Modificar la configuración de la tarea que ejecuta dicho comando. \\ \hline
        Perfil de usuario & \hyperref[sec:personaAdmin]{Administrador} \\ \hline
        Escenario & · Dado: que quiero modificar los datos de un comando,\linebreak · Cuando: proveo al sistema del identificador del comando a editar y los datos que se deben modificar,\linebreak · Entonces: el sistema modifica los datos del comando. \\ \hline
        Notas funcionales & · No se debe permitir la existencia de comandos con el mismo nombre.\linebreak · En caso de que el comando esté siendo usado por un bot que se encuentre desplegado, este debe ser reiniciado. \\ \hline
        Notas técnicas & Parámetros necesarios para obtener los detalles de un bot específico y para modificarlo:\linebreak · \verb|id: Guid|. Identificador único para cada bot.\linebreak · \verb|prefix: str|. Prefijo del comando.\linebreak · \verb|params: dict|. Parámetros adicionales del comando. \\ \hline
        Dependencias & \hyperref[sec:hu05]{HU-05} \\ \hline
        Tareas de seguimiento & – \\ \hline
        Criterio de aceptación & · El comando es modificado correctamente.\linebreak · El usuario es avisado en caso de error al modificar los datos del comando.\linebreak · Tests unitarios y integración son creados dentro de lo posible. \\ \hline
    \end{tabularx}
    \end{adjustbox}
    \caption{HU-07. Editar un comando.}
\end{table}

\subsection{HU-08 - Eliminar un comando}
\label{sec:hu08}

\begin{table}[H]
    \centering
    \def\arraystretch{1.25}
    \begin{adjustbox}{max width=\textwidth}
    \begin{tabularx}{\textwidth}{|l|L|}
    \hline
        \textbf{Sección} & \textbf{Contenido} \\ \hline
    \hline
        Resumen & Como usuario administrador quiero poder eliminar un comando para que deje de estar disponible en el sistema. \\ \hline
        Meta & El borrado de comandos en el sistema. \\ \hline
        Beneficio & Cuando ya no es necesario un comando, se puede eliminar. \\ \hline
        Perfil de usuario & \hyperref[sec:personaAdmin]{Administrador} \\ \hline
        Escenario & · Dado: que quiero eliminar un comando,\linebreak · Cuando: proveo al sistema del identificador del comando a eliminar,\linebreak · Entonces: el sistema elimina el comando y sus datos asociados. \\ \hline
        Notas funcionales & · En caso de que el comando esté en uso por un bot desplegado, éste debe reiniciarse. \\ \hline
        Notas técnicas & Parámetros necesarios para eliminar un bot:\linebreak · \verb|id: Guid|. Identificador único para cada bot. \\ \hline
        Dependencias & \hyperref[sec:hu05]{HU-05} \\ \hline
        Tareas de seguimiento & – \\ \hline
        Criterio de aceptación & · El comando es eliminado correctamente.\linebreak · El usuario es avisado en caso de error al eliminar los datos del comando.\linebreak · Tests unitarios y integración son creados dentro de lo posible. \\ \hline
    \end{tabularx}
    \end{adjustbox}
    \caption{HU-08. Eliminar un comando.}
\end{table}

\subsection{HU-09 - Lanzar bots}
\label{sec:hu09}

\begin{table}[H]
    \centering
    \def\arraystretch{1.25}
    \begin{adjustbox}{max width=\textwidth}
    \begin{tabularx}{\textwidth}{|l|L|}
    \hline
        \textbf{Sección} & \textbf{Contenido} \\ \hline
    \hline
        Resumen & Como usuario administrador quiero poder ejecutar los bots de \textit{Discord} que he creado y configurado previamente en el sistema para poder hacer uso de los comandos de los que disponen en los servidores de \textit{Discord}. \\ \hline
        Meta & Desplegar bots en los workers para que puedan usarse desde los servidores de \textit{Discord}. \\ \hline
        Beneficio & La ejecución de los bots permite que los comandos estén disponibles en los servidores de \textit{Discord} para los usuarios (una vez los bots se agreguen a esos servidores). \\ \hline
        Perfil de usuario & \hyperref[sec:personaAdmin]{Administrador} \\ \hline
        Escenario & · Dado: que quiero desplegar un bot de \textit{Discord},\linebreak · Cuando: proveo al sistema de un identificador de bot de \textit{Discord} existente,\linebreak · Entonces: el sistema despliega el bot automáticamente, lo que me permite empezar a hacer uso de sus comandos. \\ \hline
        Notas funcionales & · No se debe permitir el despliegue de un bot varias veces.\linebreak · No se debe permitir el despliegue de un bot que no tiene comandos configurados. \\ \hline
        Notas técnicas & Elementos necesarios para el despliegue del bot:\linebreak · \verb|id: Guid|. Identificador único para cada bot. \\ \hline
        Dependencias & \hyperref[sec:hu01]{HU-01}, \hyperref[sec:hu03]{HU-03}, \hyperref[sec:hu05]{HU-05}, \hyperref[sec:hu07]{HU-07} \\ \hline
        Tareas de seguimiento & – \\ \hline
        Criterio de aceptación & · El bot queda desplegado correctamente.\linebreak · El usuario es avisado en caso de error al desplegar un bot.\linebreak · Tests unitarios y integración son creados dentro de lo posible. \\ \hline
    \end{tabularx}
    \end{adjustbox}
    \caption{HU-09. Lanzar bots.}
\end{table}

\subsection{HU-10 - Cancelar ejecución de bots}
\label{sec:hu10}

\begin{table}[H]
    \centering
    \def\arraystretch{1.25}
    \begin{adjustbox}{max width=\textwidth}
    \begin{tabularx}{\textwidth}{|l|L|}
    \hline
        \textbf{Sección} & \textbf{Contenido} \\ \hline
    \hline
        Resumen & Como usuario administrador quiero terminar la ejecución de un bot de \textit{Discord} en el sistema para que deje de estar disponible. \\ \hline
        Meta & Terminar la ejecución de bots en los workers para que dejen de poder usarse desde los servidores de \textit{Discord}. \\ \hline
        Beneficio & Cuando sea necesario realizar mantenimiento a un bot (o cuando ya no sea necesario que esté en activo), se puede cancelar su ejecución. \\ \hline
        Perfil de usuario & \hyperref[sec:personaAdmin]{Administrador} \\ \hline
        Escenario & · Dado: que quiero cancelar el despliegue de un bot para realizar algún tipo de tarea de mantenimiento,\linebreak · Cuando: proveo al sistema de un identificador de bot de \textit{Discord} que se encuentre desplegado,\linebreak · Entonces: el sistema termina la ejecución del bot. \\ \hline
        Notas funcionales & · No se debe permitir cancelar el despliegue de un bot que no se encuentra desplegado. \\ \hline
        Notas técnicas & Parámetros necesarios para cancelar el despliegue de un bot:\linebreak · \verb|id: Guid|. Identificador único para cada bot. \\ \hline
        Dependencias & \hyperref[sec:hu01]{HU-01}, \hyperref[sec:hu03]{HU-03}, \hyperref[sec:hu05]{HU-05}, \hyperref[sec:hu07]{HU-07}, \hyperref[sec:hu09]{HU-09} \\ \hline
        Tareas de seguimiento & – \\ \hline
        Criterio de aceptación & · El despliegue del bot es cancelado.\linebreak · El usuario es avisado en caso de error al cancelar el despliegue del bot.\linebreak · Tests unitarios y integración son creados dentro de lo posible. \\ \hline
    \end{tabularx}
    \end{adjustbox}
    \caption{HU-10. Cancelar ejecución de bots.}
\end{table}

\subsection{HU-11 - Consultar estado de los despliegues}
\label{sec:hu11}

\begin{table}[H]
    \centering
    \def\arraystretch{1.25}
    \begin{adjustbox}{max width=\textwidth}
    \begin{tabularx}{\textwidth}{|l|L|}
    \hline
        \textbf{Sección} & \textbf{Contenido} \\ \hline
    \hline
        Resumen & Como usuario administrador quiero conocer el estado de los despliegues de los bots de \textit{Discord} en el sistema. \\ \hline
        Meta & Obtener detalles de los workers y de los bots que se encuentran desplegados en los workers. \\ \hline
        Beneficio & Consultar cuales de los bots están activos y cuales no, para tareas de monitorización del sistema. \\ \hline
        Perfil de usuario & \hyperref[sec:personaAdmin]{Administrador} \\ \hline
        Escenario & · Dado: que quiero conocer el estado de los despliegues de los bots,\linebreak · Cuando: hago una petición al sistema para ello,\linebreak · Entonces: el sistema me devuelve los datos de los bots que se encuentran desplegados. \\ \hline
        Notas funcionales & · No se deben devolver datos de configuración del bot.\linebreak · Se debe devolver información de los workers que ejecutan los bots. \\ \hline
        Notas técnicas & Datos a devolver:\linebreak · \verb|workers: []Worker|. Todos los workers que hay disponibles en el sistema.\linebreak \linebreak Worker:\linebreak · \verb|id: Guid|. Identificador único para cada worker.\linebreak · \verb|uptime: DateTime|. El tiempo de actividad del worker.\linebreak · \verb|location: str|. La URL donde se encuentra el worker.\linebreak · \verb|bots: []Bot|. Los bots que están desplegados en el worker.\linebreak \linebreak Bot:\linebreak · \verb|id: Guid|. Identificador único para cada bot.\linebreak · \verb|uptime: DateTime|. El tiempo de actividad del bot.\linebreak · ... \\ \hline
        Dependencias & \hyperref[sec:hu01]{HU-01}, \hyperref[sec:hu03]{HU-03}, \hyperref[sec:hu05]{HU-05}, \hyperref[sec:hu07]{HU-07}, \hyperref[sec:hu09]{HU-09}, \hyperref[sec:hu10]{HU-10} \\ \hline
        Tareas de seguimiento & – \\ \hline
        Criterio de aceptación & · Los datos de los bots de \textit{Discord} que se encuentran desplegados son devueltos al usuario.\linebreak · El usuario es avisado en caso de error al obtener los datos de despliegue.\linebreak · Tests unitarios y integración son creados dentro de lo posible. \\ \hline
    \end{tabularx}
    \end{adjustbox}
    \caption{HU-11. Consultar estado de los despliegues.}
\end{table}

\subsection{HU-12 - Interfaz de usuario}
\label{sec:hu12}

\begin{table}[H]
    \centering
    \def\arraystretch{1.25}
    \begin{adjustbox}{max width=\textwidth}
    \begin{tabularx}{\textwidth}{|l|L|}
    \hline
        \textbf{Sección} & \textbf{Contenido} \\ \hline
    \hline
        Resumen & Como usuario administrador quiero disponer de una interfaz gráfica para poder crear, configurar, desplegar y conocer el estado de los bots de \textit{Discord} que hay en el sistema. \\ \hline
        Meta & Disponer de una interfaz gráfica que permita realizar las tareas de creación, configuración y despliegue de bots de \textit{Discord}. \\ \hline
        Beneficio & Realizar todas las tareas de gestión de comandos y bots de manera sencilla en una interfaz de usuario, en lugar de hacerlas mediante el uso de una API. \\ \hline
        Perfil de usuario & \hyperref[sec:personaAdmin]{Administrador} \\ \hline
        Escenario & · Dado: que quiero administrar los bots de \textit{Discord},\linebreak · Cuando: accedo a la interfaz gráfica,\linebreak · Entonces: el sistema me permite realizar todas las tareas de administración de bots y comandos. \\ \hline
        Notas funcionales & – \\ \hline
        Notas técnicas & – \\ \hline
        Dependencias & \hyperref[sec:hu01]{HU-01}, \hyperref[sec:hu02]{HU-02}, \hyperref[sec:hu03]{HU-03}, \hyperref[sec:hu04]{HU-04}, \hyperref[sec:hu05]{HU-05}, \hyperref[sec:hu06]{HU-06}, \hyperref[sec:hu07]{HU-07}, \hyperref[sec:hu08]{HU-08}, \hyperref[sec:hu09]{HU-09}, \hyperref[sec:hu10]{HU-10}, \hyperref[sec:hu11]{HU-11} \\ \hline
        Tareas de seguimiento & – \\ \hline
        Criterio de aceptación & · La interfaz permite realizar las tareas de gestión de bots y comandos.\linebreak · El usuario es avisado en caso de producirse algún error.\linebreak · Tests unitarios y integración son creados dentro de lo posible. \\ \hline
    \end{tabularx}
    \end{adjustbox}
    \caption{HU-12. Interfaz de usuario.}
\end{table}

\subsection{HU-13 - Criterios de evaluación}
\label{sec:hu13}

\begin{table}[H]
    \centering
    \def\arraystretch{1.25}
    \begin{adjustbox}{max width=\textwidth}
    \begin{tabularx}{\textwidth}{|l|L|}
    \hline
        \textbf{Sección} & \textbf{Contenido} \\ \hline
    \hline
        Resumen & Como miembro del tribunal, quisiera disponer de una documentación, una presentación y un informe acordes a los criterios de evaluación para comprobar que Estos se han cumplido correctamente. \\ \hline
        Meta & Disponer de una documentación que recoja claramente toda la información referente al desarrollo del TFM. \\ \hline
        Beneficio & De este modo es más sencillo evaluar todo el trabajo que el alumno ha realizado para desarrollar el TFM. \\ \hline
        Perfil de usuario & \hyperref[sec:personaMiembroTribunal]{Miembro del tribunal} \\ \hline
        Escenario & · Dado: que quiero evaluar el trabajo realizado por el alumno,\linebreak · Cuando: éste me de acceso a dicha documentación acorde a los criterios de evaluación,\linebreak · Entonces: podré evaluar el trabajo del alumno. \\ \hline
        Notas funcionales & Los criterios de evaluación son:\linebreak \linebreak El estudiante…\linebreak · Utiliza fuentes de información variadas, válidas y fiables y selecciona la relevante para el objetivo del trabajo.\linebreak · Toma decisiones adecuadas al contexto y propone soluciones utilizando el conocimiento adquirido.\linebreak · Detecta y analiza oportunidades para hacer nuevas propuestas.\linebreak · Propone soluciones adecuadas y justifica las decisiones tomadas para resolver problemas complejos.\linebreak · Utiliza recursos formales e informales para documentar adecuadamente el proceso de desarrollo: concepción, planificación, análisis, diseño, implementación, pruebas, etc.\linebreak · Muestra claridad y comprensión en la redacción,organizando la información adecuadamente y utilizando los recursos adecuados para el discurso escrito. Muestra claridad y comprensión en la expresión oral, organizando la información adecuadamente y utilizando los recursos adecuados para el discurso oral. \\ \hline
        Notas técnicas & – \\ \hline
        Dependencias & – \\ \hline
        Tareas de seguimiento & – \\ \hline
        Criterio de aceptación & · La documentación cumple con los criterios de evaluación. \\ \hline
    \end{tabularx}
    \end{adjustbox}
    \caption{HU-13. Criterios de evaluación.}
\end{table}
 
\section{User journeys}

En esta sección se enumeran los \textit{user journeys} tanto para el uso del sistema, como para la ampliación del repertorio de comandos.

\subsection{Uso del sistema}

\subsubsection{Crear un bot}

\begin{enumerate}
	\item El usuario crea una aplicación de \textit{Discord} en el \href{https://discord.com/developers/applications}{portal de desarrolladores} y obtiene un \textit{token} para un bot.
	\item El usuario provee al sistema de este \textit{token} y de un nombre.
	\item[!] Si el nombre o el \textit{token} se encuentran en uso por otro bot, el sistema cancela la creación.
	\item El bot es creado en el sistema, quedando disponible para ser configurado con comandos o para ser desplegado.
\end{enumerate}

\subsubsection{Modificar un bot}

Se puede modificar un bot de dos maneras:

\begin{itemize}
	\item Parámetros del bot.
	\begin{enumerate}
		\item El usuario provee al sistema de un nombre o \textit{token} distinto a los actuales.
		\item[!] Si el nombre o el \textit{token} corresponden a otro bot, el sistema cancela la modificación.
		\item[!] Si el bot se encuentra desplegado, el sistema cancela el despliegue previa modificación.
		\item El sistema modifica los detalles del bot.
	\end{enumerate}
	
	\item Comandos del bot.
	\begin{enumerate}
		\item El usuario provee al sistema de los identificadores de los comandos que quiere agregar o eliminar del bot.
		\item[!] Si el bot se encuentra desplegado, el sistema cancela el despliegue previa modificación.
		\item El sistema modifica los comandos del bot.
	\end{enumerate}
\end{itemize}

\subsubsection{Eliminar un bot}

\begin{enumerate}
	\item El usuario provee al sistema del identificador del bot que quiere eliminar.
	\item[!] Si el bot se encuentra desplegado, el sistema cancela el despliegue previa eliminación.
	\item El sistema elimina todos los datos asociados al bot, no pudiendo volver a usarse.
\end{enumerate}

\subsubsection{Crear un comando}

\begin{enumerate}
	\item El usuario provee al sistema de un nombre, un prefijo, un tipo de comando y de los parámetros necesarios para ese tipo de comando.
	\item[!] Si el nombre se encuentra en uso, el sistema cancela la creación del comando.
	\item El sistema crea el comando.
\end{enumerate}

\subsubsection{Modificar un comando}

\begin{enumerate}
	\item El usuario provee al sistema de los datos que quiere modificar de un comando, estos incluyen el nombre, prefijo y parámetros.
	\item[!] Si los datos que el usuario provee (nombre y prefijo) corresponden a otro comando, el sistema cancela la modificación.
	\item[!] Si el comando se encuentra en uso por un bot que se encuentra desplegado, el sistema cancela el despliegue previa modificación.
	\item El sistema modifica el comando.
\end{enumerate}

\subsubsection{Eliminar un comando}

\begin{enumerate}
	\item El usuario provee al sistema del identificador del comando que quiere eliminar.
	\item[!] Si el comando se encuentra en uso por un bot que se encuentra desplegado, el sistema cancela el despliegue previa eliminación.
	\item El sistema elimina todos los datos asociados al comando, no pudiendo volver a agregarse a un bot.
\end{enumerate}

\subsubsection{Desplegar un bot}

\begin{enumerate}
	\item El usuario provee al sistema del identificador del bot que quiere desplegar, además del identificador del \textit{worker} donde quiere desplegarlo.
	\item[!] Si el bot ya se encuentra desplegado, el sistema cancela el despliegue.
	\item[!] Si el bot o el \textit{worker} no existen, el sistema cancela el despliegue.
	\item El sistema despliega el bot en el \textit{worker}.
\end{enumerate}

\subsubsection{Cancelar despliegue de un bot}

\begin{enumerate}
	\item El usuario provee al sistema del identificador del bot del que quiere cancelar el despliegue.
	\item[!] Si el bot no se encuentra desplegado, el sistema cancela la operación.
	\item[!] Si el bot no existe, el sistema cancela la operación.
	\item El sistema cancela el despliegue del bot.
\end{enumerate}


\subsection{Ampliación del repertorio de comandos}

Para la realización de este \textit{user journey} es necesario el uso de C\#.

\begin{enumerate}
	\item El usuario define el nuevo tipo de comando.
	\item El usuario define los parámetros que necesita ese comando.
	\item El usuario implementa la funcionalidad asociada al comando, esto es el código que se ejecuta cuando el comando es invocado.
	\item El comando queda disponible para ser usado por el sistema.
	\item[!] Es necesario reiniciar el sistema para que pueda utilizarse el nuevo comando.
\end{enumerate}

\section{Modelo de negocio}

La solución propuesta se caracteriza por ser software libre, pero los costos de desarrollo e implementación nunca son nulos.

Debido a la muy probable falta de financiación al inicio del desarrollo del software, los objetivos iniciales se centrarían en obtener la renta mínima para poder continuar con el desarrollo del proyecto. A medida que se supere este primer obstáculo y el software esté mejor establecido, el modelo de financiación cambiaría para lograr un mayor valor de mercado y ganancias.

Como modelo de negocio, teniendo en cuenta que se opta por una solución compuesta por software libre, con el fin de sufragar todos estos gastos se podría optar por un modelo de consultoría. En este, se ofrecería soporte personalizado y desarrollo de características personalizadas para cada uno de los clientes que contratase el servicio. Otra posible fuente de ingresos podría ser el \textit{hosting} de bots mediante suscripciones mensuales, ofreciendo la herramienta y los bots como servicio.

\subsection{Sociedad Limitada Nueva Empresa}

Una \textit{SLNE} es una buena opción, ya que permite crear una pequeña empresa con pocos recursos iniciales con la que iniciar el desarrollo de forma profesional el desarrollo. Además tiene bastantes beneficios frente a otros modelos:

\begin{itemize}
	\item Construcción rápida.
	\item No necesita registro de socios.
	\item Fraccionado y aplazamiento de retenciones del \textit{IRPF} y otras deudas y pagos fraccionados.
	\item Se puede cambiar la denominación social de forma gratuita.
\end{itemize}

Los gastos para poder desarrollar un software de las características descritas de forma profesional no son desorbitados, pero tampoco son bajos. En cuanto a gastos derivados de la empresa y burocráticos serían (al menos) los siguientes:

\begin{itemize}
	\item 3000 euros. El capital mínimo a aportar para crear una \textit{SLNE}.
	\item 1000 euros. Estimación de los distintos gastos burocráticos.
	\item 550 euros. Gastos derivados con el desarrollo de la actividad laboral, como por ejemplo un local. Al año supone al menos 6600 euros.
\end{itemize}

Además, hay que tener en cuenta el salario del trabajador, que en este caso sería uno solo, para intentar abaratar costes. Los datos de empleo de 2022 en el sector de la Informática y Telecomunicaciones indican que el salario medio de una persona con aproximadamente 3 años de experiencia laboral (como es mi caso) se sitúa en 36500 euros brutos, lo que se traduce aproximadamente en 3050 euros brutos al mes. De nuevo, a fin de reducir los costes al inicio de la actividad laboral de esta empresa, se podría fijar un salario inferior, 28000 euros brutos al año.

A todas estas cifras habría que sumar todos los gastos relacionados con el desarrollo del software en sí, como pueden ser servicios de alojamiento del código, integración continua, copias de seguridad o sistemas y equipos informáticos. En una etapa inicial se podrían utilizar las versiones gratuitas de algunos estos servicios, pero en ciertos casos no sería posible ya que pueden ser necesarias otras características adicionales.

A continuación se muestra un posible presupuesto del gasto anual teniendo en cuenta todos los aspectos anterior mencionados.

\begin{table}[H]
    \centering
    \def\arraystretch{1.25}
    \begin{adjustbox}{max width=\textwidth}
    \begin{tabularx}{\textwidth}{|L|r|r|r|}
    \hline
        \textbf{Concepto} & \textbf{Euros/Ud} & \textbf{Cantidad} & \textbf{Total (Euros)} \\ \hline
    \hline
        Capital inicial (SLNE) & 3000 & 1 & 3000 \\ \hline
        Burocracia & 1000 & 1 & 1000 \\ \hline
        Derivados & 550 & 12 & 6600 \\ \hline
        Salario & 2333 & 12 & 28000 \\ \hline
        Servicios \textit{Cloud} & 150 & 12 & 1800 \\ \hline
        Sistemas informáticos & 3000 & 1 & 3000 \\ \hline
    \hline
        \multicolumn{3}{|r|}{\textbf{Total}} & \textbf{43400} \\ \hline
    \end{tabularx}
    \end{adjustbox}
    \caption{Presupuesto anual como \textit{SLNE}.}
\end{table}

Se puede observar que el primer año de vida de esta empresa (que hasta el momento sólo tiene un empleado) costaría más de 43000 euros, una cifra bastante alta. En el caso de que se quisiera incluir a un nuevo empleado, también desarrollador con experiencia similar, el coste adicional ascendería a aproximadamente 33000 euros.

\subsection{Autónomo}

Ser autónomo es otra posible opción para comenzar a desarrollar el software profesionalmente. En este caso los costes pueden ser algo inferiores, y además existen deducciones en el caso de ser una primera alta, pero no son bajos.

La siguiente tabla muestra el posible presupuesto del gasto anual:

\begin{table}[H]
    \centering
    \def\arraystretch{1.25}
    \begin{adjustbox}{max width=\textwidth}
    \begin{tabularx}{\textwidth}{|L|r|r|r|}
    \hline
        \textbf{Concepto} & \textbf{Euros/Ud} & \textbf{Cantidad} & \textbf{Total (Euros)} \\ \hline
    \hline
        Cuota mínima & 294 & 12 & 3528 \\ \hline
        Cuota máxima & 711 & 12 & 8532 \\ \hline
    \hline
        Burocracia & 1000 & 1 & 1000 \\ \hline
        Servicios \textit{Cloud} & 150 & 12 & 1800 \\ \hline
        Sistemas informáticos & 3000 & 1 & 3000 \\ \hline
    \hline
        \multicolumn{3}{|r|}{\textbf{Total (cuota mínima)}} & \textbf{9328} \\ \hline
        \multicolumn{3}{|r|}{\textbf{Total (cuota máxima)}} & \textbf{14332} \\ \hline
    \end{tabularx}
    \end{adjustbox}
    \caption{Presupuesto anual como \textit{autónomo}.}
\end{table}

Para este caso se ha mantenido el salario objetivo que se marcaba en la sección anterior, 28000 euros divididos en 12 pagas. Además, entra en juego la cuota de autónomos, que en 2022 sitúa su mínimo en 294 euros. Este aspecto es importante, ya que esta es la aportación por la que se cotiza. Si bien en los primeros meses podría ser interesante reducir al máximo los gastos, no es lo ideal a largo plazo. Otro aspecto importante es que se deberían pagar impuestos trimestrales, como el IVA, lo que incrementa los gastos.
